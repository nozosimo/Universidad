\documentclass[10pt,letterpaper]{article}
\usepackage[utf8]{inputenc}
\usepackage[spanish]{babel}
\usepackage{amsmath}
\usepackage{amsfonts}
\usepackage{amssymb}
\usepackage{graphicx}
\usepackage{subfiles}
\usepackage{enumitem}
\usepackage[affil-it]{authblk}
\usepackage{array}
\usepackage{tabularx}
\usepackage{multicol}
\usepackage{slashbox}
\usepackage{diagbox}
\usepackage{slashbox,multirow}
\usepackage{enumitem}
\usepackage{mathtools}
\usepackage{pstricks}
\usepackage{pgfplots}
\usepackage{tikz}
\usepackage{calc}
\usepackage{ifthen}
\usepackage{mathrsfs} %Contiene el Signo de Transformada de Laplace
\usepackage{empheq}
\usepackage{svg}
\usepackage{hyperref}
\usepackage[left=2cm,right=2cm,top=2cm,bottom=2cm]{geometry}
\author{Leonardo H. Añez Vladimirovna\\
\texttt{toborochi98@outlook.com}
}
\title{Fórmulas de Física III \texttt{(FIS200)}\\{\normalsize Facultad de Ingeniería en Ciencias de la Computación y Telecomunicaciones}\\{\normalsize Universidad Autónoma Gabriel René Moreno}}
 \pgfplotsset{compat=1.16}
\begin{document}
\maketitle
\section{Interacción Electrostática}
\subsection{Carga y Campo Eléctrico}

\begin{multicols}{3}
\subsubsection{Ley de Coulomb}
$$
F_e = k_e \dfrac{|q_1\cdot q_2|}{r^2}
$$
\columnbreak
\subsubsection{Lineas de Campo}
\begin{align*}
F=ma & \Leftrightarrow  F=q_0 \overrightarrow{E}\\
ma &=q_0 \overrightarrow{E} \\ \vspace{0.01cm}\\
\Aboxed{a&=\dfrac{q_0}{m}\overrightarrow{E}}
\end{align*}
\columnbreak
\subsubsection{Campo Eléctrico}
$$
\overrightarrow{E}=\dfrac{\overrightarrow{F_0}}{q_0}\widehat{r}=\dfrac{k\dfrac{q\cdot q_0}{r^2}}{q_0}\widehat{r}=\dfrac{kq\widehat{r}}{r^2}
$$
\end{multicols}
\subsection{Dipolos}
$$
\overrightarrow{\tau}=\overrightarrow{d} \times \overrightarrow{F}=dF\sin(\theta)=dqE\sin(\theta)
$$
\begin{multicols}{2}
\subsubsection{Fuerza y Torsión}
\begin{align*}
\tau &= q\cdot d\cdot E\cdot\sin(\theta) \\
\tau &= p\cdot E\cdot\sin(\theta)
\end{align*}

\columnbreak
\subsubsection{Energía Potencial}
\end{multicols}

\begin{multicols}{2}
\subsubsection{Momento Dipolar}
\columnbreak
\subsubsection{Energía Potencial Dipolar}
\end{multicols}
\subsection{Potencial Eléctrico}
\end{document}