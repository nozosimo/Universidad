\chapter{Ecuación Lineal de Primer Orden}
Recordando la definición de una ED de orden $n$:
$$a_n(x)\dfrac{d^ny}{dx^n}+a_{n-1}(x)\dfrac{d^{n-1}y}{dx^{n-1}}+\cdots + a_2(x)\dfrac{d^2y}{dx^2}+a_1(x)\dfrac{dy}{dx}+a_0(x)y=h(x)$$
tenemos, para 1er Orden:
\begin{equation}
a_1(x)\dfrac{dy}{dx}+a_0(x)y=h(x)
\end{equation}
Dividiendo la ED entre el coeficiente de $a_1(x)$:
$$\dfrac{dy}{dx}+\underbrace{\dfrac{a_0(x)}{a_1(x)}}_{P(x)}y=\underbrace{\dfrac{h(x)}{a_1(x)}}_{Q(x)}$$
Con esto tenemos:
\begin{equation}
\dfrac{dy}{dx}+P(x)y=Q(x)
\end{equation}
\subsection*{Método de Solución}
Partimos de:
$$\dfrac{d}{dx}\Bigg[\mu(x)\cdot y\Bigg]=\mu(x)\dfrac{dy}{dx}+\mu(x)P(x) \cdot y = \mu(x)P(x) $$

Tenemos:
\[
\dfrac{d}{dx}\Bigg[\mu(x)\cdot y\Bigg]=\mu(x)\dfrac{dy}{dx}+y  \overbrace{\dfrac{d\mu}{dx}}\tikzmark{0}=\mu(x)\dfrac{dy}{dx}+ \overbrace{\tikzmark{1}\mu(x)P(x)}\cdot y =  \mu(x)Q(x)
\]
%
\begin{tikzpicture}[remember picture, overlay, bend left=30,line width=0.25mm]
  \draw ([yshift=4ex]pic cs:0) to ([yshift=4ex]pic cs:1);
\end{tikzpicture}
A partir de esta igualdad obtendremos lo que se llama como \textit{Factor Integrando:}
$$\dfrac{d\mu}{dx}=\mu(x)P(x)$$
Para simplificar, escribiremos $\mu(x)$ simplemente como $\mu$, sobreentenderemos que esta en función de $x$:

\begin{align*}
\dfrac{d\mu}{dx} & = \mu \cdot P(x) \\{ }\\
\int\dfrac{1}{\mu} & = \int P(x) dx \\{ }\\
e^{\ln(\mu)} & = e^{\int P(x) dx}
\end{align*}
\begin{equation}
\mu  = e^{\int P(x) dx}
\end{equation}
Una vez hallado el FI, procedemos a reemplazar en la ecuación:
\begin{align*}
\dfrac{d}{dx}\Big[ \mu(x)\cdot y \Big] & = \mu(x)\cdot Q(x) \\
\dfrac{d}{dx}\Big[ e^{\int P(x)dx} y \Big] & = e^{\int P(x) dx} Q(x)
\end{align*}
Integrando:
\begin{align*}
e^{\int P(x) dx} \cdot y & = \int e^{\int P(x) dx} \cdot Q(x) dx
\end{align*}
$$ \boxed{y  = e^{-\int P(x) dx} \int e^{\int P(x) dx} \cdot Q(x) dx}$$
\section{Ecuación de Bernoulli}
Tiene la siguiente forma:
$$\dfrac{dy}{dx}+P(x)\cdot y =Q(x)\cdot y^n $$
Para poder tratar la ecuación, dividimos entre $y^n$:
$$y^{-n}\cdot\dfrac{dy}{dx}+P(x)\cdot y^{1-n} =Q(x)$$
\section{Ecuaciones Diferenciales Exactas}
La ecuación:
\begin{equation}
M(x,y)dx+N(x,y)dy=0
\end{equation}
Es exacta se se puede escribir en la forma:
$$\dfrac{d}{dx}\Big[ f(x,y) \Big] = 0$$
\subsubsection{Criterio}
La diferencia total:
\begin{equation}
\dfrac{\partial f}{\partial x}dx - \dfrac{\partial f}{\partial y}dy=0
\end{equation}

Para $(1\texttt{.}4)$ y $(1\texttt{.} 5)$ serán lo mismo:
\begin{equation}
\dfrac{\partial f}{\partial x}=M(x,y) \hspace{1cm} \wedge \hspace{1cm} \dfrac{\partial f}{\partial y}=N(x,y)
\end{equation}

$$\dfrac{\partial^2 f}{\partial x \partial y}=M(x,y) \hspace{1cm} \wedge \hspace{1cm} \dfrac{\partial^2 f}{\partial y\partial x}=N(x,y)$$
\subsubsection{Por Clairaut}
Si existen las segundas derivadas cruzadas y son continuas entonces estas son iguales:
$$\dfrac{\partial^2 f}{\partial y\partial x}=\dfrac{\partial^2 f}{\partial x\partial y}$$
\subsubsection{Solución}
\begin{align*}
\dfrac{\partial f}{\partial x} & = M(x,y)
\end{align*}
\begin{equation}
f(x,y)= \int M(x,y) dx + C(y)
\end{equation}
\begin{align*}
\dfrac{\partial f}{\partial y} & = \dfrac{\partial}{\partial y}\left[\int M(x,y)\right]  dx + C'(y) 
\end{align*}
Recordando $1\texttt{.}6$:
\subsection{Ecuación Diferencial Asociada}
El objetivo es obtener la ED de una familia de curvas, mediante el proceso de eliminación de constantes arbitrarias involucradas en la familia de curvas. Es importante recordar que ninguna ED contiene constantes arbitrarias por lo tanto debemos eliminar la constante que aparece en la familia de curvas.
\subsubsection{Procedimiento}
Dada la familia de curvas:
$$F(x,y,C_1,C_2,\ldots,C_n)=0$$
Para obtener la ED asociada a la familia de Curvas se deben utilizar las constantes en el sistema:
\begin{align*}
F(x,y,C_1,C_2,\ldots,C_n) & = 0 \\
\dfrac{d}{dx}\left[ F(x,y,C_1,C_2,\ldots,C_n)\right]  & = 0 \\
\dfrac{d^2}{dx^2}\left[ F(x,y,C_1,C_2,\ldots,C_n)\right]  & = 0 
\end{align*}
\begin{center}
$\vdots $
\end{center}
\begin{align*}
\dfrac{d^n}{dx^n}\left[ F(x,y,C_1,C_2,\ldots,C_n)\right]  & = 0 
\end{align*}
\section{Trayectorias Ortogonales}
\subsection{Determinación}
\subsubsection{Primer Paso}
Dada una familia de curvas con parámetro $C$ se encuentra su ED de la forma:
$$y'=f(x,y)$$
\subsubsection{Segundo Paso}
Se encuentra las trayectorias ortogonales resolviendo la ED:
$$y' = -\dfrac{1}{f(x,y)}$$