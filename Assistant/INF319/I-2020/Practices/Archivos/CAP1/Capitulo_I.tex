\section*{Demostrar Formalmente}

 Construir dos expresiones regulares $\alpha$ y $\beta$, tales que $|L(\alpha)|=3$ y $|L(\beta)|=4$, construyendo dos autómatas ($M_\alpha$ y $M_\beta$) respectivamente y demostrar \textbf{formalmente} que: $$L(M_\alpha)\cap L(M_\beta)=\Sigma^*$$
\subsection*{Respuesta}
Primeramente construimos las ER. $\alpha$ y $\beta$.

$$ 
\alpha = (a\cup b) \cup bb \hspace{1cm} \beta =  ((a\cup b) \cup aa)\cup bb  
$$
Evidentemente, la cardinalidad del lenguaje generado por cada una cumple con el enunciado, según lo vemos a continuación:\\
\noindent
\begin{minipage}[t]{.5\textwidth}
\raggedright

\begin{align*}
L(\alpha) &= L( (a\cup b)\cup bb) \\
		  &= L(a\cup b)\cup L(bb) \\
		  &= (L(a)\cup L(b))\cup L(b)L(b) \\
		  &= (\{ a\}\cup \{ b \})\cup \{b\}\{b\} \\
		  &= \{a,b \}\cup \{bb\} \\
		  &= \{a,b,bb \}
\end{align*}
\hspace{0.5cm}
\end{minipage}% <---------------- Note the use of "%"
\begin{minipage}[t]{.5\textwidth}
\raggedright
\begin{align*}
L(\beta)  &= L( ((a\cup b) \cup aa)\cup bb  )\\
		  &= L( (a\cup b) \cup aa)\cup L(bb) \\
		  &= (L( a\cup b) \cup L(aa))\cup L(bb) \\
		  &= (( L(a)\cup L(b)) \cup L(a)L(a))\cup L(b)L(b) \\
		  &= ((\{a\}\cup\{b\})\cup\{a\}\{a\})\cup \{b\}\{b\} \\
		  & = (\{a,b\}\cup\{aa\})\cup\{bb\} \\
		  & = \{a,b,aa\}\cup\{bb\} \\
		  & = \{a,b,aa,bb\}
\end{align*}
\hspace{0.5cm}
\end{minipage}

\begin{minipage}[t]{.5\textwidth}
\raggedright
\textbf{Cardinalidad} de $\alpha$:
$$
|L(\alpha)|=3
$$

\end{minipage}% <---------------- Note the use of "%"
\begin{minipage}[t]{.5\textwidth}
\raggedright
\textbf{Cardinalidad} de $\beta$:
$$
|L(\beta)|=4
$$

\end{minipage}
\vspace{0.5cm} \\
Hasta acá simplemente estamos proponiendo las ER. para demostrar su intersección, no necesariamente tienen que ser estas ER., pero podemos tratar siempre de hacerlas lo mas pequeñas posible.

\newpage
Como queremos demostrar que $L(M_\alpha)\cap L(M_\beta)=\Sigma^*$ tenemos que construir primeramente los AFDs $M_\alpha$ y $M_\beta$, para estos lenguajes finitos (podemos usar el método del árbol):


\begin{minipage}[t]{.5\textwidth}
$M_\alpha$
\centering
\tikzset{every loop/.style={min distance=10mm,in=240,out=300,looseness=10,line width=0.25mm}}
\begin{tikzpicture}[>={Triangle[width=1.5mm,length=1.5mm]},->,node distance=2cm,auto,state/.style={circle, draw, minimum size=0.5cm}]
\node[state,accepting] (q_0) at (0,-1.5) {${ }$};
\node[state,initial above,initial text=] (q_1) at (1.5,0) {${ }$};
\node[state,accepting] (q_2) at (3,-1.5) {${ }$};
\node[state] (q_3) at (0,-3.5) {${ }$};
\node[state,accepting] (q_4) at (3,-3.5) {${ }$};

\path[->,line width=0.25mm] (q_1) edge node[above] {$a$} (q_0);
\path[->,line width=0.25mm] (q_1) edge node[] {$b$} (q_2);
\path[->,line width=0.25mm] (q_0) edge node[left] {$a,b$} (q_3);
\path[->,line width=0.25mm] (q_2) edge node[right] {$b$} (q_4);
\path[->,line width=0.25mm] (q_4) edge node[below] {$a,b$} (q_3);
\path[->,line width=0.25mm] (q_2) edge node[above] {$a$} (q_3);
\path[->] (q_3) edge  [loop below] node {$a,b$} (q_3);

%\path[->,line width=0.25mm] (q_0) edge[bend right] node[swap] {$\lambda$} (q_2); 
%\path[->,line width=0.25mm] (q_0) edge[bend left] node[] {$1$} (q_1); 
%\path[->,line width=0.25mm] (q_2) edge[bend right] node[swap] {$1$} (q_1); 
%\tikzset{every loop/.style={min distance=10mm,in=120,out=60,looseness=10,line width=0.25mm}}

%\path[->] (q_2) edge  [loop above] node {$0$} (q_2);
%\tikzset{every loop/.style={min distance=10,in=340,out=10,looseness=10,line width=0.25mm}}
%\path[->] (q_1) edge  [loop right] node {$1$} (q_1);

\end{tikzpicture}


\end{minipage}% <---------------- Note the use of "%"
\begin{minipage}[t]{.5\textwidth}
$M_\beta$
\raggedright
\tikzset{every loop/.style={min distance=10mm,in=240,out=300,looseness=10,line width=0.25mm}}
\begin{tikzpicture}[>={Triangle[width=1.5mm,length=1.5mm]},->,node distance=2cm,auto,state/.style={circle, draw, minimum size=0.5cm}]
\node[state,accepting] (q_0) at (0,-1.5) {${ }$};
\node[state,initial above,initial text=] (q_1) at (1.5,0) {${ }$};
\node[state,accepting] (q_2) at (3,-1.5) {${ }$};
\node[state,accepting] (q_3) at (0,-3.5) {${ }$};
\node[state,accepting] (q_4) at (3,-3.5) {${ }$};
\node[state] (q_5) at (1.5,-5.5) {${ }$};

\path[->,line width=0.25mm] (q_1) edge node[above] {$a$} (q_0);
\path[->,line width=0.25mm] (q_1) edge node[] {$b$} (q_2);
\path[->,line width=0.25mm] (q_0) edge node[left] {$a$} (q_3);
\path[->,line width=0.25mm] (q_2) edge node[right] {$b$} (q_4);
\path[->,line width=0.25mm] (q_4) edge node[right] {$a,b$} (q_5);
\path[->,line width=0.25mm] (q_3) edge node[left] {$a,b$} (q_5);
\path[->] (q_5) edge  [loop below] node {$a,b$} (q_5);
\path[->,line width=0.25mm] (q_0) edge node[right] {$b$} (q_5);
\path[->,line width=0.25mm] (q_2) edge node[left] {$a$} (q_5);

%\path[->,line width=0.25mm] (q_0) edge[bend right] node[swap] {$\lambda$} (q_2); 
%\path[->,line width=0.25mm] (q_0) edge[bend left] node[] {$1$} (q_1); 
%\path[->,line width=0.25mm] (q_2) edge[bend right] node[swap] {$1$} (q_1); 
%\tikzset{every loop/.style={min distance=10mm,in=120,out=60,looseness=10,line width=0.25mm}}

%\path[->] (q_2) edge  [loop above] node {$0$} (q_2);
%\tikzset{every loop/.style={min distance=10,in=340,out=10,looseness=10,line width=0.25mm}}
%\path[->] (q_1) edge  [loop right] node {$1$} (q_1);

\end{tikzpicture}
\end{minipage}
\hspace{0.5cm}\\
Lo que haremos con estos autómatas es realizar la \textbf{intersección} como la conocemos: $$\Sigma^*-[(\Sigma^*-L(M_\alpha))\cup (\Sigma^*-L(M_\beta)) ]$$
Y vemos si es que el complemento de su \textbf{intersección} es $\emptyset$, si fuera el caso, significa que el lenguaje aceptado por el autómata sin el complemento es $\Sigma^*$ (por lo que quedaría demostrado).
\subsubsection*{Construyendo la Intersección}
Primero, realizamos el complemento de cada autómata $M_\alpha$ y $M_\beta$:

\begin{minipage}[t]{.5\textwidth}
$\overline{M_\alpha}$
\centering
\tikzset{every loop/.style={min distance=10mm,in=240,out=300,looseness=10,line width=0.25mm}}
\begin{tikzpicture}[>={Triangle[width=1.5mm,length=1.5mm]},->,node distance=2cm,auto,state/.style={circle, draw, minimum size=0.5cm}]
\node[state] (q_0) at (0,-1.5) {${ }$};
\node[state,accepting,initial above,initial text=] (q_1) at (1.5,0) {${ }$};
\node[state] (q_2) at (3,-1.5) {${ }$};
\node[state,accepting] (q_3) at (0,-3.5) {${ }$};
\node[state] (q_4) at (3,-3.5) {${ }$};

\path[->,line width=0.25mm] (q_1) edge node[above] {$a$} (q_0);
\path[->,line width=0.25mm] (q_1) edge node[] {$b$} (q_2);
\path[->,line width=0.25mm] (q_0) edge node[left] {$a,b$} (q_3);
\path[->,line width=0.25mm] (q_2) edge node[right] {$b$} (q_4);
\path[->,line width=0.25mm] (q_4) edge node[below] {$a,b$} (q_3);
\path[->,line width=0.25mm] (q_2) edge node[above] {$a$} (q_3);
\path[->] (q_3) edge  [loop below] node {$a,b$} (q_3);

%\path[->,line width=0.25mm] (q_0) edge[bend right] node[swap] {$\lambda$} (q_2); 
%\path[->,line width=0.25mm] (q_0) edge[bend left] node[] {$1$} (q_1); 
%\path[->,line width=0.25mm] (q_2) edge[bend right] node[swap] {$1$} (q_1); 
%\tikzset{every loop/.style={min distance=10mm,in=120,out=60,looseness=10,line width=0.25mm}}

%\path[->] (q_2) edge  [loop above] node {$0$} (q_2);
%\tikzset{every loop/.style={min distance=10,in=340,out=10,looseness=10,line width=0.25mm}}
%\path[->] (q_1) edge  [loop right] node {$1$} (q_1);

\end{tikzpicture}


\end{minipage}% <---------------- Note the use of "%"
\begin{minipage}[t]{.5\textwidth}
$\overline{M_\beta}$
\raggedright
\tikzset{every loop/.style={min distance=10mm,in=240,out=300,looseness=10,line width=0.25mm}}
\begin{tikzpicture}[>={Triangle[width=1.5mm,length=1.5mm]},->,node distance=2cm,auto,state/.style={circle, draw, minimum size=0.5cm}]
\node[state] (q_0) at (0,-1.5) {${ }$};
\node[state,accepting,initial above,initial text=] (q_1) at (1.5,0) {${ }$};
\node[state] (q_2) at (3,-1.5) {${ }$};
\node[state] (q_3) at (0,-3.5) {${ }$};
\node[state] (q_4) at (3,-3.5) {${ }$};
\node[state,accepting] (q_5) at (1.5,-5.5) {${ }$};

\path[->,line width=0.25mm] (q_1) edge node[above] {$a$} (q_0);
\path[->,line width=0.25mm] (q_1) edge node[] {$b$} (q_2);
\path[->,line width=0.25mm] (q_0) edge node[left] {$a$} (q_3);
\path[->,line width=0.25mm] (q_2) edge node[right] {$b$} (q_4);
\path[->,line width=0.25mm] (q_4) edge node[right] {$a,b$} (q_5);
\path[->,line width=0.25mm] (q_3) edge node[left] {$a,b$} (q_5);
\path[->] (q_5) edge  [loop below] node {$a,b$} (q_5);
\path[->,line width=0.25mm] (q_0) edge node[right] {$b$} (q_5);
\path[->,line width=0.25mm] (q_2) edge node[left] {$a$} (q_5);

%\path[->,line width=0.25mm] (q_0) edge[bend right] node[swap] {$\lambda$} (q_2); 
%\path[->,line width=0.25mm] (q_0) edge[bend left] node[] {$1$} (q_1); 
%\path[->,line width=0.25mm] (q_2) edge[bend right] node[swap] {$1$} (q_1); 
%\tikzset{every loop/.style={min distance=10mm,in=120,out=60,looseness=10,line width=0.25mm}}

%\path[->] (q_2) edge  [loop above] node {$0$} (q_2);
%\tikzset{every loop/.style={min distance=10,in=340,out=10,looseness=10,line width=0.25mm}}
%\path[->] (q_1) edge  [loop right] node {$1$} (q_1);

\end{tikzpicture}
\end{minipage}
\newpage
Luego, realizamos la \textbf{unión} de los complementos:

\begin{center}
\tikzset{every loop/.style={min distance=10mm,in=240,out=300,looseness=10,line width=0.25mm}}
\begin{tikzpicture}[>={Triangle[width=1.5mm,length=1.5mm]},->,node distance=2cm,auto,state/.style={circle, draw, minimum size=0.5cm}]
\node[state] (q_0) at (0,-1.5) {${ }$};
\node[state,accepting] (q_1) at (1.5,0) {${ }$};
\node[state] (q_2) at (3,-1.5) {${ }$};
\node[state,accepting] (q_3) at (0,-3.5) {${ }$};
\node[state] (q_4) at (3,-3.5) {${ }$};

\path[->,line width=0.25mm] (q_1) edge node[above] {$a$} (q_0);
\path[->,line width=0.25mm] (q_1) edge node[] {$b$} (q_2);
\path[->,line width=0.25mm] (q_0) edge node[left] {$a,b$} (q_3);
\path[->,line width=0.25mm] (q_2) edge node[right] {$b$} (q_4);
\path[->,line width=0.25mm] (q_4) edge node[below] {$a,b$} (q_3);
\path[->,line width=0.25mm] (q_2) edge node[above] {$a$} (q_3);
\path[->] (q_3) edge  [loop below] node {$a,b$} (q_3);



\node[state] (q_5) at (5,-1.5) {${ }$};
\node[state,accepting] (q_6) at (6.5,0) {${ }$};
\node[state] (q_7) at (8,-1.5) {${ }$};
\node[state] (q_8) at (5,-3.5) {${ }$};
\node[state] (q_9) at (8,-3.5) {${ }$};
\node[state,accepting] (q_10) at (6.5,-5.5) {${ }$};

\path[->,line width=0.25mm] (q_6) edge node[above] {$a$} (q_5);
\path[->,line width=0.25mm] (q_6) edge node[] {$b$} (q_7);
\path[->,line width=0.25mm] (q_5) edge node[left] {$a$} (q_8);
\path[->,line width=0.25mm] (q_7) edge node[right] {$b$} (q_9);
\path[->,line width=0.25mm] (q_9) edge node[right] {$a,b$} (q_10);
\path[->,line width=0.25mm] (q_8) edge node[left] {$a,b$} (q_10);
\path[->] (q_10) edge  [loop below] node {$a,b$} (q_10);
\path[->,line width=0.25mm] (q_5) edge node[right] {$b$} (q_10);
\path[->,line width=0.25mm] (q_7) edge node[left] {$a$} (q_10);


\node[state,initial above,initial text=] (q_11) at (4,1.5) {${ }$};
\path[->,line width=0.25mm] (q_11) edge node[above] {$\lambda$} (q_1);
\path[->,line width=0.25mm] (q_11) edge node[above] {$\lambda$} (q_6);

\end{tikzpicture}
\end{center}
Necesitamos ahora convertirlo a un AFD:

\begin{center}
\tikzset{every loop/.style={min distance=10mm,in=240,out=300,looseness=10,line width=0.25mm}}
\begin{tikzpicture}[>={Triangle[width=1.5mm,length=1.5mm]},->,node distance=2cm,auto,state/.style={circle, draw, minimum size=0.5cm}]
\node[state] (q_0) at (0,-1.5) {$q_3$};
\node[state,accepting] (q_1) at (1.5,0) {$q_1$};
\node[state] (q_2) at (3,-1.5) {$q_4$};
\node[state,accepting] (q_3) at (0,-3.5) {$q_7$};
\node[state] (q_4) at (3,-3.5) {$q_8$};

\path[->,line width=0.25mm] (q_1) edge node[above] {$a$} (q_0);
\path[->,line width=0.25mm] (q_1) edge node[] {$b$} (q_2);
\path[->,line width=0.25mm] (q_0) edge node[left] {$a,b$} (q_3);
\path[->,line width=0.25mm] (q_2) edge node[right] {$b$} (q_4);
\path[->,line width=0.25mm] (q_4) edge node[below] {$a,b$} (q_3);
\path[->,line width=0.25mm] (q_2) edge node[above] {$a$} (q_3);
\path[->] (q_3) edge  [loop below] node {$a,b$} (q_3);



\node[state] (q_5) at (5,-1.5) {$q_5$};
\node[state,accepting] (q_6) at (6.5,0) {$q_2$};
\node[state] (q_7) at (8,-1.5) {$q_6$};
\node[state] (q_8) at (5,-3.5) {$q_9$};
\node[state] (q_9) at (8,-3.5) {$q_{10}$};
\node[state,accepting] (q_10) at (6.5,-5.5) {$q_{11}$};

\path[->,line width=0.25mm] (q_6) edge node[above] {$a$} (q_5);
\path[->,line width=0.25mm] (q_6) edge node[] {$b$} (q_7);
\path[->,line width=0.25mm] (q_5) edge node[left] {$a$} (q_8);
\path[->,line width=0.25mm] (q_7) edge node[right] {$b$} (q_9);
\path[->,line width=0.25mm] (q_9) edge node[right] {$a,b$} (q_10);
\path[->,line width=0.25mm] (q_8) edge node[left] {$a,b$} (q_10);
\path[->] (q_10) edge  [loop below] node {$a,b$} (q_10);
\path[->,line width=0.25mm] (q_5) edge node[right] {$b$} (q_10);
\path[->,line width=0.25mm] (q_7) edge node[left] {$a$} (q_10);


\node[state,initial above,initial text=] (q_11) at (4,1.5) {$q_0$};
\path[->,line width=0.25mm] (q_11) edge node[above] {$\lambda$} (q_1);
\path[->,line width=0.25mm] (q_11) edge node[above] {$\lambda$} (q_6);

\end{tikzpicture}
\end{center}
Hasta este paso, solo etiquete los estados para poder realizar la tabla de transición en la conversión.
\newpage
Realizamos la tabla de transición:

\begin{center}
\begin{tabular}{|c|c|c|c|}
\hline 
 {\footnotesize  \shortstack{Estados}} & {\footnotesize \shortstack{Entrada $\lambda$}} & {\footnotesize \shortstack{Entrada $a$}} & {\footnotesize \shortstack{Entrada $b$}} \\ 
\hline 
$\rightarrow q_0$ & $q_0,q_1,q_2$ & $q_3,q_5$ & $q_4,q_6$ \\ 
\hline 
\underline{$q_1$} & $q_1$ & $q_3$ & $q_4$ \\ 
\hline 
\underline{$q_2$} & $q_2$ & $q_5$ & $q_6$ \\ 
\hline 
$q_3$ & $q_3$ & $q_7$ & $q_7$ \\ 
\hline 
$q_4$ & $q_4$ & $q_7$ & $q_8$ \\ 
\hline 
$q_5$ & $q_5$ & $q_9$& $q_{11}$ \\ 
\hline 
$q_6$ & $q_6$ &$q_{11}$ & $q_{10}$ \\ 
\hline 
\underline{$q_7$} & $q_7$ & $q_7$ & $q_7$ \\ 
\hline 
$q_8$ & $q_8$ & $q_7$ & $q_7$ \\ 
\hline 
$q_9$ & $q_9$ & $q_{11}$ & $q_{11}$ \\ 
\hline 
$q_{10}$ & $q_{10}$ & $q_{11}$ & $q_{11}$ \\ 
\hline 
\underline{$q_{11}$} & $q_{11}$ &  $q_{11}$ & $q_{11}$ \\ 
\hline 
\end{tabular} 
\end{center}
Finalmente realizamos el AFD con la tabla:
\begin{figure}[h]
\centering
\begin{tikzpicture}[>={Triangle[width=1.5mm,length=1.5mm]},->,node distance=2cm,auto]
\node[state,accepting,initial,initial text=,rectangle] (q_0) {$\{ q_0 ,q_1,q_2 \}$};
\node[state,rectangle] (q_1) at (3,2) {$\{ q_3,q_5 \}$};
\node[state,rectangle] (q_2) at (3,-2) {$\{ q_4,q_6\}$};
\node[state,accepting,rectangle] (q_3) at (6,2) {$\{ q_7,q_9\}$};
\node[state,rectangle] (q_4) at (6,-2) {$\{ q_8,q_{10}\}$};
\node[state,accepting,rectangle] (q_5) at (9,0){$\{ q_7,q_{11} \}$};

\path[->,line width=0.25mm] (q_0) edge node[,below,right,sloped,left,above] {$a$} (q_1);
\path[->,line width=0.25mm] (q_0) edge node[sloped] {$b$} (q_2);
\path[->,line width=0.25mm] (q_1) edge node[sloped] {$a$} (q_3);
\path[->,line width=0.25mm] (q_1) edge node[sloped] {$b$} (q_5);
\path[->,line width=0.25mm] (q_2) edge node[sloped] {$a$} (q_5);
\path[->,line width=0.25mm] (q_2) edge node[sloped,below] {$b$} (q_4);
\path[->,line width=0.25mm] (q_4) edge node[sloped,below] {$a,b$} (q_5);
\path[->,line width=0.25mm] (q_3) edge node[sloped,above] {$a,b$} (q_5);
\path[->,line width=0.25mm, loop above] (q_5) edge node[above] {$a,b$} (q_5);
%bend left
\end{tikzpicture}
\end{figure}
\\
Con este AFD lo unico que nos falta por hacer es el complemento para obtener la intersección que buscabamos: $L(M_\alpha)\cap L(M_\beta)$. Y ver si es que nos da $\Sigma^*$.
\\{ }\\
 Recordemos por un momento que doble complemento es el conjunto original: $$(A^C)^C=A$$
\noindent
Por lo que podriamos no realizar el complemento ya que obtendriamos el mismo autómata de arriba.
\\{ }\\
\textbf{Finalmente} teniendo el autómata, podemos afirmar que $L(M_\alpha)\cap L(M_\beta)\neq\Sigma^*$ ya que al ver el complemento de su intersección vemos que existen secuencias de aristas que nos llevan a un estado terminal a partir de $\{q_0,q_1,q_2\}$.
$$ \overline{ L(M_\alpha)\cap L(M_\beta)}\neq\emptyset$$
Por lo tanto:
$$ L(M_\alpha)\cap L(M_\beta) \neq \Sigma^* $$