\chapter{Variables Aleatorias}
Una variable aleatoria $x$ (desde ahora denotada por \textbf{v.a.}) es una función definida sobre el espacio muestral $S$ con valores en $\mathbb{R}$ que a cada elemento de $S$ (Punto muestral) hace corresponder un número real $x=X$.

$$x=X(w) \in Rec_X \subseteq \mathbb{R}$$ 
\subsubsection{Gráficamente}
%Pendiente
\subsubsection{Notación Conjuntista}
$$X = \lbrace (w,x)\ w\in S, x=X(w)\in \mathbb{R} \rbrace \subseteq S\times  \mathbb{R}$$
Donde:
\begin{itemize}
\item $S$: Conjunto Partida (Espacio Muestral).
\item $\mathbb{R}$: Conjunto de llegada.
\item $w$: Elemento de $S$ (Punto Muestral).
\item $x$: Valor de la \textbf{v.a.} $X$.
\item $Rec_X$: Recorrido de $X$.
\item $X$: Función \textbf{v.a.} (Conjunto de Pares Ordenados).
\end{itemize}
\subsubsection{Notaciones}
Las \textbf{v.a.} se denotan con letras mayúsculas tales como $X,Y$ o $Z$, y los valores correspondientes con letras minúsculas.
\section{Clasificación de Variables Aleatorias}
\begin{itemize}
\item \textbf{Discreta:} Cuyo recorrido es un conjunto finito o infinito numerable de valores:
$$
X \text{ es \textbf{v.a.} discreta} \Rightarrow
\begin{cases}
\text{Conjunto Finito de Valores} \\
\text{Conjunto Infinito Numerable de Valores}
\end{cases}
$$
\item \textbf{Contínua:} Es aquella cuyo recorrido es conjunto finito no numerable de valores, puede tomar cualquier valor en un intervalo o conjunto.
\end{itemize}
$\blacklozenge$ En general las \textbf{v.a.} \textit{discretas} representan datos que provienen del \textit{conteo} de número de elementos. Pueden ser número de titulados, número de estudiantes, etc. Mientras que las \textbf{v.a.} \textit{contínuas} representan mediciones, como longitud, capacidad,etc.
\section{Función de Probabilidad de una Variable Aleatoria}
\noindent También llamada función de cuantía o función de masa de probabilidad de una \textbf{v.a.}. \\${ }$\\
Se denomina función de probabilidad de una \textbf{v.a.} discreta $X$ a una función $p$ o $f$, cuyo valor es $p(x)$ o $P(X=x)0$ ya que a cada valor distinto de la \textbf{v.a.} discreta $X$ hace corresponder en un número entre los valores $[0,1]$ que es su probabilidad, de ahí el nombre de función de cuantía o función de probabilidad. Estos valores satisfacen las siguientes condiciones:
\begin{enumerate}
\item $P(x)\geq 0 \hspace{0.25cm};\hspace{0.25cm} \forall x\in \mathbb{R}$
\item $\displaystyle\sum_{x_i\in Rec_X}^{}p(x_i)=1$
\begin{itemize}
\item Si $Rec_X=\lbrace x_1,x_2,\ldots ,x_n \rbrace$ entonces la condición \textbf{(II)} es: $\displaystyle\sum_{i=1}^{n}p(x_i)=1$
\item Si $Rec_X=\lbrace x_1,x_2,\ldots ,x_n,\ldots \rbrace$ entonces la condición \textbf{(II)} es: $\displaystyle\sum_{i=1}^{\infty}p(x_i)=1$
\end{itemize}
\end{enumerate}
Si $A$ es un evento en el recorrido de la \textbf{v.a.} discreta $X$ entonces la probabilidad de $A$ es el número:
$$P(A)=\displaystyle\sum P(X=x)=\sum p(x)$$
\textbf{Nota:}
$$P(X=x)
\begin{cases}
p(x)\geq 0 ;\hspace{0.25cm} \forall x\in \mathbb{R} \\
f(x)\geq 0 ;\hspace{0.25cm} \forall x\in \mathbb{R}
\end{cases}
$$
La función de probabilidad de una \textbf{v.a.} discreta $X$ se puede expresar por:
\begin{itemize}
\item \textbf{Un Conjunto:} 
$$p=\lbrace (x,P(X))/x\in D_p \rbrace$$
\item \textbf{Una Tabla:}
\begin{center}
\begin{tabular}{|c|c|c|c|c|}
\hline 
$x_i$ & $x_1$ & $x_2$ & $\ldots$ & $x_n$ \\ 
\hline 
$p(x_i)$ & $p(x_1)$ & $p(x_2)$ & $\ldots$ & $p(x_n)$ \\ 
\hline 
\end{tabular} 
\end{center}
\item \textbf{Una Gráfica:}
\end{itemize}
\section{Función de Distribución Acumulada (FDA)}
El valor de la \textbf{FDA} de una \textbf{v.a.} discreta $X$, que es $F(x)$, viene dada por la sumatoria de las probabilidades, desde un valor mínimo $t$ hasta un valor específico $x$; esto es:
$$F(x)=P(X\leq x)=\displaystyle\sum_{t\leq x} P(t),\hspace{0.25cm} \forall x\in \mathbb{R}$$
\subsection{Representación Gráfica}
Valores $F(x)$ aumentan en \textit{saltos}, presentando entonces la forma de una escalera:
%Hacer la grafica
\subsection{Caso Continuo}
\subsubsection{Función de Densidad}
$f$ es función densidad, si $f(x)$ cumple las siguientes condiciones:
\begin{enumerate}[label=(\roman*)]
\item $f(x)\geq 0;\hspace{0.25cm} \forall x\in \mathbb{R}$
\item $\displaystyle\int_{-\infty}^{\infty}f(x) dx = 1$
\item $p(a\leq x \leq b)=\displaystyle\int_{a}^{b}f(x)dx$
\end{enumerate}
\subsubsection{FDA}
$$F(x)=P(X\leq x)=\displaystyle\int_{-\infty}^{x} f(x) dx$$
\subsection{Propiedades de la FDA}
\subsubsection{Caso Discreto}
\begin{enumerate}
\item $0\leq F(x) \leq 1;\hspace{0.25cm}\forall x\in\mathbb{R}$
\end{enumerate}
\subsubsection{Caso Continuo}