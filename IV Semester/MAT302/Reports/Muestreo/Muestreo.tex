\documentclass[10pt,letterpaper]{article}
\usepackage[utf8]{inputenc}
\usepackage[spanish]{babel}
\usepackage{amsmath}
\usepackage{amsfonts}
\usepackage{amssymb}
\usepackage{graphicx}
\usepackage[left=2cm,right=2cm,top=2cm,bottom=2cm]{geometry}
\author{Leonardo H. Añez Vladimirovna\footnote{\textbf{correo:} \texttt{toborochi98@outlook.com}} \\ \vspace{0.1cm} \\
Facultad de Ingeniería en Ciencias de la Computación y Telecomunicaciones\\
Universidad Autónoma Gabriél René Moreno}
\title{Técnicas de Muestreo}
\begin{document}
\maketitle
\section{Muestreo}
A veces, cuando se cuenta con una población y necesitamos realizar un estudio sobre esto, no podemos analizar todos los elementos, por lo que realizamos la selección de una muestra, es decir, escogemos una parte representativa de la población. Con lo que el muestreo es una técnica que nos permite determinar que parte de la población debe examinarse, para poder realizar inferencia sobre esta.
\subsection{Errores al Realizar el Muestreo}
El error de muestreo se refiere a las diferencias entre la muestra y población que existe debido a las observaciones realizadas al seleccionar la muestra. El error de muestreo se produce cuando los investigadores toman una muestra aleatoria en lugar de observar a cada sujeto individual que comprende una población. 
Cuando trabajamos con grandes poblaciones, este proceso se convierte en la única opción, por lo que el error de muestreo es extremadamente difícil de evitar. Y esto ocurre a dos razones principales:
\begin{enumerate}
\item Hacer  conclusiones  muy  generales 
a  partir  de  la  observación  de  sólo  una  parte  de la Población, se denomina error de muestreo
\item Hacer  conclusiones hacia una Población mucho más grandes de la que originalmente 
se tomo la muestra. Error de Inferencia.
\end{enumerate}
Independienemente de la situación el Error de Muestreo es algo que en la mayoria de los casos es imposible de evitar. Para ello podemos tomar medidas para estimar y reducir el error de muestreo.\\${ }$\\
Es por eso que mayormente tendremos un margen de error que podremos apreciar con los resultados de la muestra y como es solo una estimación, existe una pequeña posibilidad de que el margen de error sea realmente mayor que el que estamos dispuestos a tolerar.
\subsubsection{Reduciendo el Error de Muestreo}
En general, el error de muestreo se reduce a medida que aumenta el tamaño de la muestra. Porque la muestra representa con mayor precisión a la población. Además, el error de muestreo se controlará si además de aumentando el tamaño de la muestra, la muestra se elige al azar de la
población.
\section{Métodos de Muestreo}
Antes de describir procedimientos de muestreo, necesitamos definir algunos términos. El termino \textit{población}, describe a todos los miembros que cumplen con ciertos criterios y a un miembro de este conjunto lo llamamos \textit{elemento}. Cuando algunos elementos son seleccionados nos referimos a esto como \textit{muestra}, cuando seleccionamos a todos los elementos lo llamamos \textit{censo}.
\subsection{Probabilístico}
\subsection{No Probabilístico}

\begin{thebibliography}{9}
\bibitem{SamplingTec} 
Sampling Techniques, University of Central Arkansas, 2013. \\\texttt{http://uca.edu/psychology/files/2013/08/Ch7-Sampling-Techniques.pdf}
\bibitem{SampErr}
Sedgwick, Philip. (2012). What is sampling error?. BMJ. 344. e4285. 10.1136/bmj.e4285. 
\bibitem{sonora}
Material, Departamento de Matemáticas, Universidad de Sonora.\\\texttt{http://www.estadistica.mat.uson.mx/Material/elmuestreo.pdf}

\end{thebibliography}
\end{document}