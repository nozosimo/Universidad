\chapter{Aproximaciones y Errores}
\section{Errores}
\begin{itemize}
\item \textbf{Errores de Redondeo:} Se debe a que el computador solo puede representar cantidades con un \textit{número finito }de dígitos.
\item \textbf{Errores de Truncamiento:} Representa la diferencia entre la formulación  matemática exacta de un problema y la aproximación dada por un método numérico.
\item \textbf{Cifras Significativas:} Se refiere a la confiabilidad de un valor numérico. Es el número de dígitos mas un dígito estimado que se puede usar con confianza.\\${ }$\\
$\blacklozenge$ Los ceros no siempre son cifras significativas ya que pueden usarse solo para ubicar el punto decimal. 
\end{itemize}
\subsubsection{Ejemplo:}
Los siguientes números tienen cuatro cifras significativas y en notación científica se escriben de la siguiente forma:
\begin{align*}
0.00001845 &= 1.845 \times 10^{-5} \\
0.0001845  &= 1.845 \times 10^{-4} \\
0.001845   &= 1.845 \times 10^{-3}
\end{align*}
Las cifras significativas, se cuentan de \textit{izquierda a derecha} a partir del primer dígito distinto de cero.

\section{Exactitud y Precisión}
\begin{itemize}
\item \textbf{Exactitud:} Se refiere a la aproximación de un número o medida al valor verdadero que se supone presenta.
\item \textbf{Precisión:} Se refiere a:
\begin{itemize}
\item Al número de cifras significativas que representa una cantidad.
\item La extensión en las lecturas repetidas, de un instrumento que mide alguna propiedad física.
\end{itemize}
\end{itemize}
\section{Definición de Error}
\subsection{Error Verdadero $(E_v)$}
Es la diferencia simple entre el valor Verdadero $(V_v)$ y el Valor Aproximado $(V_a)$:
$$E_v=V_v - V_a$$
Esta definición de error no toma en consideración la magnitud del valor que se está evaluando.
\subsection{Error Relativo Verdaderos}
Para introducir la magnitud del valor que ese está midiendo se normaliza el error $(E_r)$ al valor verdadero $V_v$:
$$E_r = \dfrac{E_v}{V_v}\cdot 100\% = \dfrac{V_v-V_a}{V_v}\cdot 100\% $$
En aplicaciones reales el valor verdadero $(V_v)$ únicamente se conocerá cuando se trate de funciones que tienen solución analítica. En general, el valor verdadero no se lo conoce. Se utiliza para ello la mejor estimación posible.
\\${ }$\\
Los métodos numéricos que utilizan sistemas iterativos, el error se normaliza, respecto de los valores aproximados. De esta manera, el error relativo aproximado se calcula por:
$$E_a = \dfrac{V_{actual}-V_{anterior}}{V_{actual}}\cdot 100\% $$
donde el $V_{actual}$ representa el último valor calculado y $V_{anterior}$ es el valor anterior en dos iteraciones sucesivas.
\\${ }$\\
El signo del error en general, no es significativo, siendo de mayor importancia acotar el valor absoluto del error $|E_a|$ menor a una cierta tolerancia. $E_s$, que en términos de la cantidad $n$ de cifras significativas utilizadas en el cálculo del error es:
$$E_s =0.5 \cdot 10^{2-n} \%$$
Donde:
\begin{itemize}
\item $n:$ Número de Cifras Significativas
\end{itemize}
Esta formula garantiza que el valor calculado tendrá $n$ cifras significativas iguales al valor verdadero.
\section{Series}
Las series son sucesiones de términos (en general infinitos) que se utilizan para representar funciones. El uso de series facilita el tratamiento de aquellas expresiones que son muy complicadas.
\subsubsection{Serie de Taylor}
$$f(x)=\displaystyle\sum_{n=0}^{\infty} \dfrac{ f^{(n)}(a) \cdot (x-a)^n}{n!} $$
Donde:
\begin{itemize}
\item $f^{(n)}(a):$ Derivada $n$-ésima de $f(x)$ evaluada en $a$.
\item $n!:$ Factorial de $n$.
\item $a:$ Entorno reducido de $x$.
\end{itemize}
\subsubsection{Serie de MacLaurin}
Es un caso particular de la serie de Taylor para $a=0$.
$$f(x)=\displaystyle\sum_{n=0}^{\infty} \dfrac{ f^{(n)}(0) \cdot (x)^n}{n!} $$
\section{Reglas de Redondeo}