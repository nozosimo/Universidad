\chapter{Raíces de Ecuaciones}
Se entiende por \textit{raíz} de una ecuación $f(x)$, al valor de $x$ que satisface:
\begin{multicols}{2}
\textbf{a)} $f(x)=0$
\columnbreak
\begin{flushleft}
\textbf{b)} $f(x)=p$
\end{flushleft}
\end{multicols}
$\blacklozenge$ $\xi:$ Letra griega (Xi) utilizada para representar el valor verdadero \textit{(exacto)} de la raíz.
\\${ }$\\
El caso mas conocido de cálculo de raíces en la fórmula cuadrática:
$$x_{1,2}=\dfrac{-b \pm \sqrt{b^2 - 4ac}}{2a}$$
donde $x_1$ y $x_2$ son las raíces del polinomio de segundo grado:
$$f(x)=ax^2+bx+c$$
Para polinomios de mayor grado no  existen fórmulas sencillas tampoco para las funciones trascendentes (logaritmos, trigonométricas, exponenciales o para combinaciones de todas ellas).
\\${ }$\\
Por ello se recurre a los métodos numéricos donde se verán algoritmos generales para resolver este tipo de problemas.
\section{Métodos de Intervalo}
\subsection{Método Gráfico}
\subsection{Método de Bisección}
\subsection{Método de Regla Falsa}
\subsection{Método de Regla Falsa Mejorada}
\section{Métodos Abiertos}
\subsection{Método de la Secante}
\subsection{Método de Newton-Raphson}