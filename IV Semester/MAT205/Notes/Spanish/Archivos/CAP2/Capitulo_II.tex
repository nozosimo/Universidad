\chapter{Raíces de Ecuaciones}
Se entiende por \textit{raíz} de una ecuación $f(x)$, al valor de $x$ que satisface:
\begin{multicols}{2}
\textbf{a)} $f(x)=0$
\columnbreak
\begin{flushleft}
\textbf{b)} $f(x)=p$
\end{flushleft}
\end{multicols}
$\blacklozenge$ $\xi:$ Letra griega (Xi) utilizada para representar el valor verdadero \textit{(exacto)} de la raíz.
\\${ }$\\
El caso mas conocido de cálculo de raíces en la fórmula cuadrática:
$$x_{1,2}=\dfrac{-b \pm \sqrt{b^2 - 4ac}}{2a}$$
donde $x_1$ y $x_2$ son las raíces del polinomio de segundo grado:
$$f(x)=ax^2+bx+c$$
Para polinomios de mayor grado no  existen fórmulas sencillas tampoco para las funciones trascendentes (logaritmos, trigonométricas, exponenciales o para combinaciones de todas ellas).
\\${ }$\\
Por ello se recurre a los métodos numéricos donde se verán algoritmos generales para resolver este tipo de problemas.
\section{Métodos de Intervalo}
\subsection{Método Gráfico}
Este método nos da una aproximación inicial a la de las raíces de $f(x)$, consiste en representar la función en una gráfica en el plano cartesiano. Para ello se generan pares ordenados $(x,f(x))$ que representan puntos en el plano y que al unirlos con una curva apropiada se obtiene el gráfico de $f(x)$.
\subsubsection{Teorema de Bolzano}
Se dice que $f(x)$ tiene por lo menos una raíz $\xi$ en el intervalo $(a,b)$ si se cumple que:
$$f(a)\cdot f(b) < 0$$
\subsection{Método de Bisección}
Consiste en dividir en dos partes un intervalo. Para una función $f(x)$ conocida, y el intervalo $(a,b)$ que cumple con el \textit{Teorema de Bolzano.}
\\${ }$\\
El método consiste en aproximarse a la raíz $\xi$ subdividiendo el intervalo $(a,b)$ en dos.
\subsubsection{Método de Bisección}
\begin{algorithm}[H]
  Con: 
  $a_0=a$ ,
  $b_0=b$\\
  \For{$i=0,1,\ldots$, hasta que se satisfaga}{
  \vspace{0.2cm}  
  Calcular: 
  \fbox{$x_{i+1}=\dfrac{a_i + b_i}{2}$}\\
  \vspace{0.2cm} 
  
  \uIf{$f(a_i)\cdot f(x_{i+1})<0$}{
   \vspace{0.2cm}
   $a_{i+1}=a_i$\\
   $b_{i+1}=x_{i+1}$
   }
   \uElseIf{$f(a_i)\cdot f(x_{i+1})>0$}{
   \vspace{0.2cm}
   $a_{i+1}=x_{i+1}$\\
   $b_{i+1}=b_i$
  }
 }
 \caption{Método de Bisección}
\end{algorithm}

\subsection{Método de Regla Falsa}
Este método parte del supuesto que la raíz $\xi$ está mas próxima del extremo del intervalo $(a,b)$ donde $|f(a)|$ o $|f(b)|$ es menor.
\\${ }$\\
Por ello en vez de calcular el valor medio de $a$ y $b$ como aproximación a la raíz $\xi$ se calculará la media ponderada. Se debe entonces incorporar los valores $f(a)$ y $f(b)$ en la ecuación de iteración.
\salto
Para deducir la ecuación del método de la Regla Falsa, se traza inicialmente una recta uniendo los  extremos de $f(a)$ y $f(b)$. Donde esta recta intersecta al eje $x$ (falsa posición) será la primera aproximación a la raíz $x_1$:
$$x_1 = a + \triangle_{0x}$$
Por relación de Triángulos:
\begin{align*}
\dfrac{\triangle_{0x}}{-f(a)} &= \dfrac{b-a}{f(b)+[-f(a)]}
\end{align*}

\subsubsection{Método de la Regla Falsa}
\begin{algorithm}[ht]
  Con: 
  $a_0=a$ ,
  $b_0=b$\\
  \For{$i=0,1,\ldots$, hasta que se satisfaga}{
  \vspace{0.2cm}  
  Calcular: 
  \fbox{$x_{i+1}=\dfrac{a_i\cdot f(b_i) - b_i \cdot f(a_i)}{f(b_i)-f(a_i)}$}\\
  \vspace{0.2cm} 
  
  \uIf{$f(a_i)\cdot f(x_{i+1})<0$}{
   \vspace{0.2cm}
   $a_{i+1}=a_i$\\
   $b_{i+1}=x_{i+1}$ \\
   $f(b_{i+1})=f(x_{i+1})$
   }
   \uElseIf{$f(a_i)\cdot f(x_{i+1})>0$}{
   \vspace{0.2cm}
   $a_{i+1}=x_{i+1}$\\
   $b_{i+1}=b_i$ \\
   $f(a_{i+1})=f(x_{i+1})$
  }
 }
 \caption{Método de la Regla Falsa}
\end{algorithm}
\subsection{Método de Regla Falsa Mejorada}
\section{Métodos Abiertos}
\subsection{Método de la Secante}
\subsection{Método de Newton-Raphson}