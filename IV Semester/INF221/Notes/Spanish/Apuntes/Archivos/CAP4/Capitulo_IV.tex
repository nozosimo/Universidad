\chapter{Sistemas de Ecuaciones Diferenciales}
\section{Sistemas no Homogéneos}
Una solución particular $X_p$ en un intervalo $I$ es cualquier vector libre de parámetros arbitrarios cuyas entradas son funciones que satisfacen \textbf{T.4}
\subsubsection{T.5 Solución General, Sistemas no Homogéneos}
Sea $X_p$ una solución del sistema no homogéneo del \textbf{T.4} en $I$ y sea: 
$$X_c=C_1 X_1 + C_2 X_2+\cdots +C_n X_n$$
La solución general en el mismo intervalo del sistema homogéneo asociado \textbf{T.5}, entonces la solución general del sistema no homogéneo, en el intervalo, es:
$$X=X_c + X_p$$
\section{Sistemas Lineales no Homogéneos}
\subsection{Autovalores Reales y Distintos}
Cuando la Matriz $A$ tiene $n$ valores distintos:
$$\lambda_1,\lambda_2,\lambda_3,\ldots ,\lambda_n$$
Entonces un conjunto de $m$ autovectores propios:
$$k_1,k_2,k_3,\ldots , k_m$$
Se puede encontrar:
$$X_1 = K_1 e^{\lambda_1 \cdot t}$$
\subsection{Autovalores Repetidos}