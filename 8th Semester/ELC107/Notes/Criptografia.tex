\documentclass[10pt,a4paper]{book}
\usepackage[utf8]{inputenc}
\usepackage[spanish]{babel}
\usepackage{amsmath}
\usepackage{amsfonts}
\usepackage{amssymb}
\usepackage{graphicx}
\usepackage{subfiles}
\usepackage{enumitem}
\usepackage[affil-it]{authblk}
\usepackage{array}
\usepackage{tabularx}
\usepackage{multicol}
\usepackage{slashbox}
\usepackage{diagbox}
\usepackage{slashbox,multirow}
\usepackage{enumitem}
\usepackage{mathtools}
\usepackage{pstricks}
\usepackage{pgfplots}
\usepackage{babel}
\usepackage{tikz}
\usetikzlibrary{babel}
\usetikzlibrary{calc,trees,positioning,arrows,chains,shapes.geometric,%
    decorations.pathreplacing,decorations.pathmorphing,shapes,%
    matrix,shapes.symbols}
\usepackage{calc}
\usepackage{ifthen}
\usepackage{mathrsfs} %Contiene el Signo de Transformada de Laplace
\usepackage{empheq}
\usepackage{svg}
\usepackage{hyperref}
\hypersetup{colorlinks=true,allcolors=gray}
\usepackage{hypcap}
\DeclarePairedDelimiter{\Ex}{[}{]}
\newcommand{\E}{\Ex}

\pgfplotsset{compat=1.16}

\newcommand\blfootnote[1]{%
  \begingroup
  \renewcommand\thefootnote{}\footnote{#1}%
  \addtocounter{footnote}{-1}%
  \endgroup
}

\newcolumntype{C}[1]{>{\centering\arraybackslash}p{#1}}
\usepackage[left=2cm,right=2cm,top=2cm,bottom=2cm]{geometry}

\pgfmathdeclarefunction{gauss}{2}{%
  \pgfmathparse{1/(#2*sqrt(2*pi))*exp(-((x-#1)^2)/(2*#2^2))}%
}

\newcommand{\varstackrel}[3][T]{\stackrel{\raisebox{0.5ex}{\clap{\scriptsize#2}}}{#3}}

\tikzset{
>=stealth',
  punktchain/.style={
    rectangle, 
    rounded corners, 
    % fill=black!10,
    draw=black, very thick,
    text width=10em, 
    minimum height=3em, 
    text centered, 
    on chain},
  line/.style={draw, thick, <-},
  element/.style={
    tape,
    top color=white,
    bottom color=blue!50!black!60!,
    minimum width=8em,
    draw=blue!40!black!90, very thick,
    text width=10em, 
    minimum height=3.5em, 
    text centered, 
    on chain},
  every join/.style={->, thick,shorten >=1pt},
  decoration={brace},
  tuborg/.style={decorate},
  tubnode/.style={midway, right=2pt},
}

\begin{document}
\subfile{Archivos/Introduccion}
\chapter{Conceptos Básicos}
\section{Seguridad Informatica o Seguridad de la Informacion}

\subsection*{No es lo mismo seguridad informática que seguridad de la información.}

Hoy en día  esta demostrado que la información es uno de los activos mas importantes para una empresa u organización y  por otra parte, ésta se enfrenta a una variada y cada vez más a diferentes amenazas. Asi mismo, tendremos que entender que nuestra información personal también es importante. Por lo tanto, la conclusión lógica a la que se llega tras un elemental primer análisis de los riesgos a los que se enfrenta dicha información, es que debemos protegerla en sus cuatro estados posibles, esto es cuando dicha información se crea, cuando se transmite, cuando se almacena y cuando se destruye, es decir durante todo su ciclo de vida.
\\{ }\\
\textbf{Vamos a pensar un poco.}
¿Por qué crees que tiene sentido proteger a la información cuando ésta se destruye? Pon algún ejemplo.
\\{ }\\
Hasta finales del siglo XX y muy especialmente en entornos no universitarios, era bastante común apreciar una confusión ante la definición y los alcances de la seguridad informática y la seguridad de la información, una situación que hacía suponer a mucha gente que ambos términos eran sinónimos y que, por tanto, significaban lo mismo. Como a fecha de hoy se sigue observando dicha confusión y dado que la diferencia entre ambas es ya muy marcada, resulta recomendable dedicar esta primera lección para aclarar conceptos, observar similitudes y resaltar sus diferencias. No obstante, ante la duda de usar un término u otro, tal como veremos en esta lección, hoy en día lo más correcto sería referirnos a todo esto como seguridad de la información, al ser este último término más amplio que el de seguridad informática.

\subsection*{Seguridad informática}
Cuando nos referimos a seguridad informática, estamos centrando nuestra atención sólo en los aspectos de seguridad que inciden o tienen que ver directamente con la informática; es decir, en los medios informáticos en los que se genera, gestiona, almacena o destruye esta información, pero sin profundizar en aspectos sistémicos de la gestión de esa seguridad. Ateniéndonos entonces en primera instancia a las temáticas propias de la seguridad informática, entendida ésta según lo indicado en el párrafo anterior, que por lo demás es como se conocía a esta especialidad en sus inicios desde el nacimiento de la computación, podríamos representarla en 7 grandes apartados, cada uno de ellos con entidad suficiente como para convertirse en una especialidad tecnológica:
\begin{enumerate}
\item Protección y seguridad de los datos
\item Criptografía
\item Seguridad y fortificación de redes
\item Seguridad en aplicaciones informáticas,programas y bases de datos.
\item Gestión de seguridad en equipos y sistemas informáticos
\item Informática forense
\item Ciberdelito, ciberseguridad
\end{enumerate}
\subsection*{Seguridad de la información}
Cuando además de lo anterior tenemos en cuenta aquellos aspectos sistémicos de la gestión de la seguridad, como podrían ser por ejemplo en una empresa u organización las políticas y planes de seguridad con su respectivo análisis y gestión del riesgo, la continuidad del negocio, la adecuación al entorno legal y a las normativas internacionales, entonces es más propio hablar de seguridad de la información. Esto es debido a que en la actualidad existen otras temáticas muy importantes que están relacionadas con la seguridad de la información y la protección de los datos pero que, a diferencia de los apartados vistos anteriormente, su entorno no está físicamente tan cerca de los equipos informáticos y de las redes, sino de la gestión, de la gerencia y del buen gobierno de la empresa. Entre estos otros temas más propios de un entorno empresarial, se pueden contemplar también 7 grandes apartados, a saber:

\begin{enumerate}

\item Gestión de la seguridad de la información

\item Asesoría y auditoría de la seguridad

\item Análisis y gestión de riesgos

\item Continuidad de negocio

\item Buen gobierno

\item Comercio electrónico

\item Legislación relacionada con seguridad

\end{enumerate}

\end{document}