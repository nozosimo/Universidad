\chapter{Preliminares Formales}
\section{Conjuntos}
\subsection{Conjunto Finito e Infinito}
\subsubsection{Equivalencia} Dado $A$ y $B$ (conjuntos) los llamamos \textit{equivalentes} si existe una biyección: $f:A\rightarrow B$
\subsubsection{Conjunto Finito}
Un conjunto $A$ es finito si es equivalente a $\{1,2,3,\ldots ,n\}$ para algún $n\in\mathbb{N}$.
\subsubsection{Conjunto Infinito}
Un conjunto es infinito si no es finito. Si no es equivalente a $\{1,2,3,\ldots , n\}$ es decir no hay biyección. Sin embargo no todos los conjuntos finitos son equivalentes.
\begin{itemize}
\item \textbf{Conjunto Contablemente Infinito:}  Se dice que un conjunto es contablemente infinito si es equivalente con $\mathbb{N}$.
\item \textbf{Conjunto Contable:} Es contable si es finito o contablemente infinito.
\item \textbf{Conjunto Incontable:} Se dice que es incontable si no es contable.  
\end{itemize}

\subsubsection{Principio de las Casillas}
Si $A$ y $B$ son conjuntos finitos no vacíos y $|A|>|B|$ entonces no existe una función inyectiva de: $A\rightarrow B$.
\section{Preliminares}
\subsection{Alfabeto}
Un alfabeto $\Sigma$ es cualquier conjunto finito no vacío. 
\subsubsection{Ejemplo(s)}
\begin{align*}
\Sigma_1 =& \{ Leo, Martha \} \\
\Sigma_2 =& \{0,1,2,3,\ldots , 13\} \\
\Sigma_3 =& \{ a,b \} \\
\Sigma_4 =& \{ R,G,B,A\}
\end{align*}
\subsection{Palabra}
Una palabra sobre $\Sigma$ es una sucesión finita de símbolos de $\Sigma$. Es decir:
$$(\sigma_1,\sigma_2,\ldots,\sigma_n);\sigma \in \Sigma \text{\hspace{0.5cm}  ó  \hspace{0.5cm}} \sigma_1\sigma_2\sigma_3\ldots\sigma_n ; \sigma\in\Sigma$$
\subsubsection{Ejemplo(s)}
\noindent
\begin{multicols}{4}
\noindent
\begin{align*}
\textbf{Sobre } \Sigma_1& \\
w_1 =& LeoLeo \\
w_2 =& MarthaLeoMartha
\end{align*} 
\columnbreak
\begin{align*}
\textbf{Sobre } \Sigma_2& \\
w_1 =& 1111110 \\
w_2 =& 11235813
\end{align*}
\columnbreak
\begin{align*}
\textbf{Sobre } \Sigma_3& \\
w_1 =& bababababa \\
w_2 =& abba
\end{align*}
\columnbreak
\begin{align*}
\textbf{Sobre } \Sigma_4& \\
w_1 =& ABGR \\
w_2 =& RRRA
\end{align*}
\end{multicols}
Denotamos por $\Sigma^*$ el conjunto de todas las palabras sobre $\Sigma$.
\subsubsection{Longitud de una Palabra}
Sea $w$ una palabra sobre $\Sigma$, es decir $w=\sigma_1\sigma_2\ldots\sigma_n ; \sigma\in\Sigma$. La longitud de $w$ es $n$ y se denota por: $|w|=n$.
\subsubsection{Palabra vacía}
Es la sucesión vacía de símbolos de $\Sigma$ y se denota por: $\lambda$.
\subsection{Notaciones}
\begin{itemize}
\item $\Sigma^+ = \{ w\in\Sigma^* / |w|>0\}$
\item $\Sigma^k = \{ w\in\Sigma^* / |w|=k\}$
\item $\Sigma^0 = \{ w\in\Sigma^* / |w|=0\}=\{\lambda\}$
\item $\Sigma^1 = \{ w\in\Sigma^* / |w|=1\}=\Sigma$
\item $\Sigma^* = \Sigma^+ \cup \{\lambda\}$
\item $\Sigma^+ = \Sigma^* - \{\lambda\}$
\end{itemize}
\subsection{Cantidad de Ocurrencias}
Sea $w\in\Sigma^*$, denotamos por $|w|_\sigma$ al número de ocurrencias del símbolo $\sigma$ en la palabra $w$.
\subsubsection{Ejemplo(s)}
$$\Sigma = \{ a,b \} $$

\begin{itemize}
\item $\Sigma^{*}= \{\lambda, a,b,aa,bb,ab,ba,aaa,\ldots\}$
\item $\Sigma^0=\{\lambda\}$
\item $\Sigma_1 =\Sigma = \{a,b\}$
\item $\Sigma^2 = \{ ab,aa,ba,bb \}$
\end{itemize}
\subsection{Concatenación}
Sea $u,v\in\Sigma^*$ tal que $u=\sigma_1\sigma_2\ldots\sigma_n, v=\epsilon_1\epsilon_2\ldots\epsilon_n$. La concatenación de $u$ y $v$ se define por:
$$uv=\sigma_1\sigma_2\ldots\sigma_n \epsilon_1\epsilon_2\ldots\epsilon_n$$
\subsubsection{Definición de Recurrencia}
$$|\text{ }|:\Sigma^*\rightarrow\mathbb{N}$$
$$
\begin{cases}
|\lambda |=0 \\
|wa| = |w| + 1
\end{cases}
$$
\subsubsection{Ejemplo(s)}
\begin{align*}
u=abab & \text{ } &uv=ababbba \\
v=bba  & \text{ } & vu=bbaabab
\end{align*}
\subsubsection{Propiedades}
\begin{itemize}
\item $uv\neq vu$
\item $(uv)w=u(vw)$
\item $u\lambda=\lambda u=u$
\item $|uv|=|u|+|v|$
\item $|uv|_a = |u|_a + |v|_a$
\end{itemize}
\subsection{Inversa}
Si $w=\sigma_1,\sigma_2,\ldots ,\sigma_n \in\Sigma^n$ entonces $w'=\sigma_n,\sigma_{n-1},\ldots ,\sigma_1$ se llama inversa o transpuesta de $w$.
\subsubsection*{Definición de Recurrencia}
$$':\Sigma^*\rightarrow\Sigma^*$$
$$
\begin{cases}
\lambda'=\lambda \\
(wa)' = aw'
\end{cases}
$$
\subsection{Potencia de una Palabra}
$$w^n =  \underbrace{ww\ldots w}_{n-veces}$$
\subsubsection*{Definición de Recurrencia}
$$^n:\Sigma^*\rightarrow\Sigma^*$$
$$ w^1 = w $$
$$
\begin{cases}
w^0=\lambda \\
w^{n+1} = ww^{n}
\end{cases}
$$
\subsubsection{Propiedades}
\begin{itemize}
\item $|w^n| = n|w|$
\item $w^m w^n = w^{m+n}$
\item $(w^n)^m = w^{mn}$
\item $\lambda^n = \lambda$
\end{itemize}
\subsection{Principio de Inducción para $\Sigma^*$}
Sea $L$ un conjunto de palabras sobre $\Sigma$ con las propiedades:
\renewcommand{\labelenumi}{\theenumi}
\renewcommand{\theenumi}{\textbf{\roman{enumi}.)}}%
\begin{enumerate}
\item $\lambda \in L$
\item $w\in L \wedge a\in\Sigma \Rightarrow wa \in L$ \\${ }$\\
\textbf{Crecimiento por Izquierda} \\${ }$\\
\textbf{{\footnotesize ii')}} $w\in L \wedge a\in\Sigma \Rightarrow aw \in L$
\end{enumerate}
\subsubsection{Entonces}
$L=\Sigma^*$, (es decir, todas las palabras sobre $\Sigma$ están en $L$.)
\subsection{Prefijos, Sufijos y Subpalabras}
Sean $v,z \in \Sigma^*$:
\begin{itemize}
\item Se dice que $v$ es prefijo de $z$ \textbf{ssi} existe $w\in\Sigma^*$ tal que: $z=vw$. Y se escribe: ``$v \text{\textit{ pref }} z$'' .
\item Se dice que $v$ es sufijo de $z$ \textbf{ssi} existe $u\in\Sigma^*$ tal que: $z=uv$. Y se escribe: ``$v \text{\textit{ suf }} z$'' .
\item Se dice que $v$ es subpalabra de $z$ \textbf{ssi} existe $u_1,u_2\in\Sigma^*$ tal que: $z=u_1 \vee u_2$. Y se escribe: ``$v \text{\textit{ subp}} z$'' .
\end{itemize}

\begin{center}
\fbox{\begin{minipage}{28em}
Sean $x,w \in \Sigma^*$:
\begin{itemize}
\item $x$ es \textbf{prefijo} de $w$ si $w=xy$, si $\exists y \in \Sigma^* / w = xy$
\item $x$ es \textbf{sufijo} de $w$ si $w=yx$, si $\exists y \in \Sigma^* / w = yx$
\item $x$ es \textbf{subcadena} de $w$ si $w=yx$, si $\exists z,y \in \Sigma^* / w = zxy$
\end{itemize}
\end{minipage}}
\end{center}

\subsubsection*{Ejemplo(s)}
\begin{itemize}
\item $\Sigma = \{ a,b \}$ , $z = babaab$
\end{itemize}

\begin{multicols}{3}
\centering
\textbf{Prefijo}
\begin{align*}
& u_1 = ba      & {\color{red}w} & {\color{red}= baab} \\
& u_2 = bab     & {\color{red}w} & {\color{red}= aab} \\
& u_3 = baba    & {\color{red}w} & {\color{red}= ab} \\
& u_4 = babaa   & {\color{red}w} & {\color{red}= b} \\
& u_5 = babaab  & {\color{red}w} & {\color{red}= \lambda}
\end{align*}

\columnbreak
\textbf{Sufijo}
\begin{align*}
& v_1 = aab   & {\color{red}u} & {\color{red}= bab} \\
& v_2 = abaab & {\color{red}u} & {\color{red}= b} \\
& v_3 = ab    & {\color{red}u} & {\color{red}= baba} \\
& v_4 = baab  & {\color{red}u} & {\color{red}= ba} \\
& v_5 = b     & {\color{red}u} & {\color{red}= babaa}
\end{align*}

\columnbreak

\textbf{Subpalabra}
\begin{align*}
& s_1 = aa   & {\color{red}u_1} & {\color{red}= bab} & {\color{red}u_2=}&{\color{red}b} \\
& s_2 = baa  & {\color{red}u_1} & {\color{red}= ba}  & {\color{red}u_2=}&{\color{red}b}\\
& s_3 = abaa & {\color{red}u_1} & {\color{red}= b}   & {\color{red}u_2=}&{\color{red}b}\\
& s_4 = ba   & {\color{red}u_1} & {\color{red}= ba}  & {\color{red}u_2=}&{\color{red}ab}\\
& s_5 = b    & {\color{red}u_1} & {\color{red}= \lambda} & {\color{red}u_2=}&{\color{red}abaab}
\end{align*}
\end{multicols}




\subsection{Lenguajes}
Un lenguaje sobre $\Sigma$ es un subconjunto de $\Sigma^*$
\subsubsection{Operaciones}
Recordemos que ya conocemos otras operaciones (Unión, Intersección, Diferencia y Complemento), para esta materia tenemos las siguientes:
\begin{itemize}
\item Concatenación \\ ${ }$ \\
Sea $A,B \subseteq \Sigma^*$ \\
$$AB =\{ w\in\Sigma^* / w=xy, x\in A, y\in B \}$$
\item Transposición \\ ${ }$ \\
Sea $A\subseteq \Sigma^*$ \\
$$A'=\{ w'\in \Sigma^* /w\in A\} $$
\item Estrella de Kleene \\ ${ }$ \\
Sea $A\subseteq \Sigma^*$ \\
$$A^* = \{w\in\Sigma^* / w=w_1 w_2 \ldots w_n \text{ para algún } k\in\mathbb{N} \text{ y para algunas }w_1,w_2,\ldots,w_k \in A\}$$
\end{itemize}
\subsection{Orden Lexicográfico}
Cualquier lenguaje infinito puede ser ordenado lexicograficamente en consecuencia se puede poner en biyección con el conjunto $\mathbb{N}$.
\begin{enumerate}
\item Se presupone un orden en los símbolos del alfabeto.
\item Entre dos palabras de distinta longitud la de menor longitud precede a la de mayor longitud.
\item Entre 2 palabras de la misma longitud se factoriza ambas hasta el primer símbolo diferente exclusivo. La precedencia se decide comparando la segunda parte de ambas factorizaciones, en base al primer símbolo de la segunda parte.
\end{enumerate}

\subsubsection*{Ejemplo(s)}

\begin{itemize}
\item $\Sigma = \{ a,b,c \}$, $L=\{ a,b,ab,ac,acb,abc,c,aa,cba \}$
\end{itemize}

\begin{figure}[ht!]
 \centering
 \begin{tikzpicture}[ele/.style={fill=black,circle,minimum width=.8pt,inner sep=1pt},every fit/.style={ellipse,draw,inner sep=-2pt}]
 
  \node[,label=center:$\mathbb{N}$] (a0) at (0,5.25) {};   
  \node[,label=center:$1$] (a1) at (0,4) {};    
  \node[,label=center:$2$] (a2) at (0,3.5) {};    
  \node[,label=center:$3$] (a3) at (0,3) {};
  \node[,label=center:$4$] (a4) at (0,2.5) {};
  \node[,label=center:$5$] (a5) at (0,2) {};
  \node[,label=center:$6$] (a6) at (0,1.5) {};
  \node[,label=center:$7$] (a7) at (0,1) {};
  \node[,label=center:$\vdots$] (a8) at (0,0.5) {};

  \node[,label=center:$L$] (b0) at (3.5,5.25) {};   
  \node[,,label=center:$\text{ }a$] (b1) at (3.5,4) {};
  \node[,,label=center:$\text{ }b$] (b2) at (3.5,3.5) {};
  \node[,,label=center:$\text{ }c$] (b3) at (3.5,3) {};
  \node[,,label=center:$\text{ }aa$] (b4) at (3.5,2.5) {};
  \node[,,label=center:$\text{ }ab$] (b5) at (3.5,2) {};
  \node[,,label=center:$\text{ }ac$] (b6) at (3.5,1.5) {};
  \node[,,label=center:$\text{ }abc$] (b7) at (3.5,1) {};
  \node[,,label=center:$\text{  }\vdots$] (b8) at (3.5,0.5) {};  

  \node[draw,fit= (a1) (a2) (a3) (a4) (a5) (a6) (a7) (a8),minimum width=2cm] {} ;
  \node[draw,fit= (b1) (b2) (b3) (b4) (b5) (b6) (b7) (b8),minimum width=2cm] {} ;  
  \draw[->,thick,shorten <=2pt,shorten >=2pt] (a1) -- (b1);
  \draw[->,thick,shorten <=2pt,shorten >=2] (a2) -- (b2);
  \draw[->,thick,shorten <=2pt,shorten >=2] (a3) -- (b3);
  \draw[->,thick,shorten <=2pt,shorten >=2] (a4) -- (b4);
  \draw[->,thick,shorten <=2pt,shorten >=2] (a5) -- (b5);
  \draw[->,thick,shorten <=2pt,shorten >=2] (a6) -- (b6);
  \draw[->,thick,shorten <=2pt,shorten >=2] (a7) -- (b7);
 \end{tikzpicture}
\end{figure}

\subsection{Representación Finita de Lenguajes}
Necesitamos tener dos consideraciones:
\begin{itemize}
\item Cualquier representación, debe ser en si misma una palabra. Es decir una sucesión finita de símbolos de un alfabeto $\Sigma$.
\item Lenguajes diferentes deben tener representaciones diferentes.
\end{itemize}
\begin{center}

\begin{figure}[!ht]
  \centering
  \begin{minipage}{0.3\textwidth}
    \begin{tikzpicture}
\node (start) [startstop] {Representación};
\node (io1) [process,below=0.5 of start] {$\Sigma$};
\node (box1) [process,below=0.5 of io1] {$ \Sigma^* $};
\node (branch1) [process,below=0.5 of box1] {$f: \mathbb{N} \rightarrow \Sigma^*$};
\node (box2) [align=center,process,below=0.7 of branch1] {\color{red} {$\Sigma^* $ es contablemente} \\ {\color{red} infinito} };
\draw [arrow] (start) -- (io1);
\draw [arrow] (io1) -- (box1);
\draw [arrow] (box1) -- coordinate[midway](m1)(branch1);
\draw [arrow] (branch1) -- coordinate[pos=0.4](m3)(box2);
\end{tikzpicture}
  \end{minipage}
  ${ }$
  \centering
  \begin{minipage}{0.2\textwidth}
    \begin{tikzpicture}
\node (start) [startstop] {Lenguaje};
\node (io1) [process,below=0.5 of start] {$L$};
\node (box1) [process,below=0.5 of io1] {$ 2^{\Sigma^*}$};
\node (branch1) [align=center,process,below=0.5 of box1] {{\color{red}$2^{\Sigma^*}$ es incontable} };
\draw [arrow] (start) -- (io1);
\draw [arrow] (io1) -- (box1);
\draw [arrow] (box1) -- coordinate[midway](m1)(branch1);
\end{tikzpicture}
  \end{minipage}
\end{figure}

\end{center}


\subsubsection{Primer resultado de la Teoría de la Computación}
No importa cuán poderosos sean los métodos que conocemos para representar lenguajes, sólo pueden ser representados un número contable de ellos (mientras que las representaciones sean finitas), inevitablemente, un número incontable de ellos quedará sin representación bajo cualquier esquema de representación finita. 
\subsection{Expresiones Regulares}
Las expresiones regulares (ER) sobre un alfabeto ($\Sigma$) son las palabras sobre el alfabeto $\Sigma \cup {\color{red}\{ ), (, \emptyset, \cup,*\}}$ tal que cumple lo siguiente:
\renewcommand{\labelenumi}{\theenumi}
\renewcommand{\theenumi}{\textbf{\arabic {enumi}.)}}%
\begin{enumerate}
\item {\color{red}$\emptyset$} y cada símbolo de $\Sigma$ es una ER.
\item Si $\alpha$ y $\beta$ son ER entonces {\color{red}(}$\alpha\beta${\color{red})} es una ER.
\item Si $\alpha$ y $\beta$ son ER entonces {\color{red}(}$\alpha{\color{red}\cup}\beta${\color{red})} es una ER.
\item Si $\alpha$ es una ER entonces $\alpha^{{\color{red}*}}$ es una ER.
\item Nada mas es una ER a menos que provenga de \textbf{(1.)} a \textbf{(4.)}
\end{enumerate}
\subsubsection*{Ejemplo(s)}
Para $\Sigma = \{ a,b \}$ podemos formar:
\begin{align*}
{{\color{red}\emptyset}}\\
a\\
b\\
{{\color{red}(}}ab{{\color{red})}}\\
{{\color{red}(}}a{\color{red}\cup}b{{\color{red})}}\\
{{\color{red}(}}{\color{red}(}ab{\color{red})}{\color{red}\cup}b{{\color{red})}}\\
{{\color{red}(}}ba{{\color{red})^*}}\\
{\color{red}((}ba{\color{red})^*\cup(}a{\color{red}\cup}b{\color{red}))^*}
\end{align*}
\subsubsection*{Lenguaje Regular}
Un lenguaje es regular ssi es generado por una expresión regular.
\subsubsection*{Ejemplo(s)}

\begin{itemize}
\item Dado: $\Sigma = \{ a,b \}$. Escribir por comprensión los lenguajes generados por las \textit{ER}:
\begin{enumerate}[label=\alph*)]
\item $\alpha = ab^*$
	\begin{align*}
		L(\alpha ) & = L(ab^* )=L(a)L(b^*) = L(a)L(b)^* \\
			     & = \{ a \} \{ b \}^* \\
			     & = \{ a \} \{ \lambda, b,bb,bbb,\ldots \} \\
			     & = \{ a,ab,abb,abb,\ldots\} \\
		L(\alpha ) & = \{ w\in\Sigma^* / w=ab^n , n\geq 0  \}
	\end{align*}
\item $\alpha = ba^* b$
	\begin{align*}
		L(\alpha ) & = L(ba^*b)\\
			     & = L(ba^* ) L(b) \\
			     & = L(b )L(a^*) L(b) \\
			     & = \{ b \} \{ \lambda ,a,aa,aaa,\ldots\} \{ b\} \\
			     & = \{ b,bab,baab,baaab,\ldots \} \\
		L(\alpha ) & = \{ w\in\Sigma^* / w=ba^n b, n\geq 0 \} 
	\end{align*}
\item $\alpha = (a \cup b )^* a$
\item $\alpha = (a \cup b)^* a (a \cup b)^*$
\item $\alpha = b^* a b^*$
	\begin{align*}
		L(\alpha ) & = L(b^* a b^*)\\
			     & = L(b^*)L(a) L(b^*)\\
			     & = \{b\}^* \{ a \} \{ b\}^* \\
			     & = \{\lambda,b,bb,bbb,\ldots \} \{ a\} \{\lambda,b,bb,bbb,\ldots \} \\
			     & = \{a,bab,bbabb,bbbabbb,\ldots \} \\
		L(\alpha ) & = \{ w\in\Sigma^* / w = b^n a b^n ,n \geq 0 \}
	\end{align*}
\end{enumerate}
\end{itemize}

\subsection{Módulos}
\subsubsection*{Definición}
Un módulo es una tripleta $D=(k,\Sigma,f)$ donde:
\begin{itemize}
\item $k$ es un conjunto finito no vacío, \textit{llamado conjunto de estados}
\item $\Sigma$ es un conjunto finito no vacío, \textit{llamado alfabeto}
\item $f:k\times\Sigma\rightarrow k$, \textit{llamado función de transición}
\end{itemize}
\subsubsection*{Interpretación}
Un módulo se puede interpretar como un dispositivo que en determinados instantes de tiempo recibe señales (símbolos del alfabeto), que producen cambios en su configuración interna.
\begin{center}
\begin{tikzpicture}[>={Triangle[width=1.5mm,length=1.5mm]},->,node distance=2cm,auto]
\node[] (q_0) {${ }$};
\node[state,rectangle] (q_1) at (2,0) {$s\in k$};

\path[->,line width=0.25mm] (q_0) edge node[] {$\sigma\in\Sigma$} (q_1);
%\path[->] (q_0) edge node[swap] {$\sigma$} (q_1);
\end{tikzpicture}
\end{center}
\subsubsection*{Representación}
\begin{itemize}
\item \textbf{Tabla de Transición} \\

\begin{figure}[ht]
\centering
\begin{tabular}{|c|cccccc|}
    \hline
    \backslashbox{$k$}{\vspace{0.1pt}\\$\Sigma$} & $\sigma_1$ & $\sigma_1$ & $\cdots$ & $\sigma_j$ & $\cdots$ & $\sigma_m$\\ \hline
                $s_1$             &      &  &   &$\vdots$&&\\ 
                $s_2$             &      &  &  &$\vdots$&& \\ 
                $\vdots$             &      & &  & $\vdots$ &&\\ 
                $s_i$             &   $\cdots$   & $\cdots$  &$\cdots$  &$s_k$&&\\ 
                $\vdots$             &      &  &  &&& \\ 
                $s_n$             &      &  &  && &\\ \hline
\end{tabular} 
\caption{$s_k = f(s_i,\sigma_j)$}
\end{figure}
\item \textbf{Grafo} \\

\begin{figure}[ht]
\centering
\begin{tikzpicture}[>={Triangle[width=1.5mm,length=1.5mm]},->,node distance=2cm,auto]
\node[state] (q_0) {$s_i$};
\node[state] (q_1) at (3,0) {$s_j$};

\path[->,line width=0.25mm] (q_0) edge node[] {$\sigma$} (q_1);
%\path[->] (q_0) edge node[swap] {$\sigma$} (q_1);
\end{tikzpicture}
\caption{ssi: $f(s_i,\sigma)=s_j$}
\end{figure}
\end{itemize}
\subsubsection{Comportamiento Dinámico}
Sea $D=(k,\sigma,f)$ un módulo: \\
\begin{center}
\begin{tikzpicture}[>={Triangle[width=1.5mm,length=1.5mm]},->,node distance=2cm,auto]
\node[] (s_0) {$s_0$};
\node[] (s_1) at (2,0) {$s_1$};
\node[] (s_2) at (4,0) {$s_2$};
\node[] (sus) at (6,0) {$\cdots$};
\node[] (s_k) at (8,0) {$s_k$};

\node[] (t_0) at (0,1) {$t_0$};
\node[] (t_1) at (2,1) {$t_1$};
\node[] (t_2) at (4,1) {$t_2$};
\node[] (tsus) at (6,1) {$\cdots$};
\node[] (t_k) at (8,1) {$t_k$};

\path[->,line width=0.25mm] (s_0) edge node[] {$\sigma_0$} (s_1);
\path[->,line width=0.25mm] (s_1) edge node[] {$\sigma_1$} (s_2);
\path[->,line width=0.25mm] (s_2) edge node[] {$\sigma_2$} (sus);
\path[->,line width=0.25mm] (sus) edge node[] {$\sigma_{k-1}$} (s_k);
%\path[->] (q_0) edge node[swap] {$\sigma$} (q_1);
\end{tikzpicture}
\end{center}

\begin{align*}
s_{i+1} &= f(s_i, \sigma_i ) \text{ Procesa Símbolo}\\
s_{k  } &= \hat{f}(s_0, \sigma_0\sigma_1\sigma_2 \ldots \sigma_{k-1} ) \text{ Procesa Palabras}
\end{align*}

\subsubsection{Función Estado Terminal}
Sea $D=(k,\sigma,f)$ un módulo: \\${ }$\\
Una función de Estado Terminal del módulo $D$ es una única función:
$$
\widehat{f}:k\times\Sigma \rightarrow k \text{ tal que } \forall s\in k, w\in\Sigma^* , \sigma\in\Sigma
$$
$$
\begin{cases}
\widehat{f}(s,\lambda)=s \\
\widehat{f}(s,\sigma w)= \widehat{f}\left[ f(s,\sigma),w\right]
\end{cases}
$$
\subsubsection{$\diamond$ Notas}
\begin{itemize}
\item $w=\lambda$
$$
{\color{red}\widehat{f}(s,\sigma)}=\widehat{f}(s,\sigma\lambda)=\widehat{f}\left[ f(s,\sigma),\lambda\right]={\color{red}f(s,\sigma)}
$$
\item $\forall w\in\Sigma^*$
$$
f:k \rightarrow k \text{ tal que: } f_w(s)=\widehat{f}(s,w)
s
$$
\end{itemize}
\subsection{Alcanzabilidad}
Sea $D =(k,\Sigma,f)$ un módulo y sean $s,t\in K$.
\begin{itemize}
\item Decimos que $t$ es alcanzable a partir de $s$ \textbf{ssi} existe $w\in\Sigma^*$ tal que:
$$
	\hat{f}(s,w)=t
$$
y se escribe 

\begin{center}
\begin{tikzpicture}[>={Triangle[width=1.5mm,length=1.5mm]},->,node distance=2cm,auto]
\node[] (s_0) {$s$};
\node[] (s_1) at (1,0) {$t$};

\path[->,line width=0.25mm] (s_0) edge node[] {${}$} (s_1);

\end{tikzpicture}
\end{center}
\item Sea $k\in\mathbb{N}$. Decimos que $t$ es $k$-Alcanzable a partir de $s$ \textbf{ssi} existe $w\in\Sigma^*$ tal que $$|w|=k \wedge \hat{f}(s,w)=t$$  y se escribe:
\begin{center}
\begin{tikzpicture}[>={Triangle[width=1.5mm,length=1.5mm]},->,node distance=2cm,auto]
\node[] (s_0) {$s$};
\node[] (s_1) at (1,0) {$t$};

\path[->,line width=0.25mm] (s_0) edge node[] {${k}$} (s_1);

\end{tikzpicture}
\end{center}

\end{itemize}
\subsection{Máquinas}
\subsubsection{Definición}
Una máquina es una quíntupla $M=(k,\Sigma,\Delta,f,g)$ donde:
\begin{itemize}
\item $k$ es un conjunto finito no vacío, \textit{llamado conjunto de estados}
\item $\Sigma$ es un conjunto finito no vacío, \textit{llamado alfabeto de entrada}
\item $\Delta$ es un conjunto finito no vacío, \textit{llamado alfabeto de salida}
\item $f:k\times\Sigma\rightarrow k$, \textit{llamado función de transición}
\item $g:k\times\Sigma\rightarrow \Delta$, \textit{llamado función de salida}
\end{itemize}
\subsubsection{Interpretación}
Una máquina se puede interpretar como un dispositivo que en determinados instantes de tiempo recibe señales (símbolos de entrada) que producen cambios en su configuración interna y emiten señales (símbolos de salida).
\begin{center}
\begin{tikzpicture}[>={Triangle[width=1.5mm,length=1.5mm]},->,node distance=2cm,auto]
\node[] (q_0) {${ }$};
\node[state,rectangle] (q_1) at (2,0) {$s\in k$};
\node[] (q_2) at (4,0) {${ }$};

\path[->,line width=0.25mm] (q_0) edge node[] {$\sigma\in\Sigma$} (q_1);
\path[->,line width=0.25mm] (q_1) edge node[] {$\delta\in\Delta$} (q_2);
%\path[->] (q_0) edge node[swap] {$\sigma$} (q_1);
\end{tikzpicture}
\end{center}
\subsubsection{Representación}
\begin{itemize}
\item \textbf{Tabla de Transición} \\
\begin{figure}[h]
\centering
\begin{tabular}{|c|cccccc|}
    \hline
    \backslashbox{$k$}{\vspace{0.1pt}\\$\Sigma$} & $\sigma_1$ & $\sigma_1$ & $\cdots$ & $\sigma_j$ & $\cdots$ & $\sigma_m$\\ \hline
                $s_1$             &      &  &   &$\vdots$&&\\ 
                $s_2$             &      &  &  &$\vdots$&& \\ 
                $\vdots$             &      & &  & $\vdots$ &&\\ 
                $s_i$             &   $\cdots$   & $\cdots$  &$\cdots$  &$s_k/\delta_k$&&\\ 
                $\vdots$             &      &  &  &&& \\ 
                $s_n$             &      &  &  && &\\ \hline
\end{tabular} 
\caption{$s_k = f(s_i,\sigma_j)\hspace{0.5cm}g(s_1,\sigma_j)=\delta_k$}
\end{figure}
\item \textbf{Grafo} \\

\begin{figure}[h]
\centering
\begin{tikzpicture}[>={Triangle[width=1.5mm,length=1.5mm]},->,node distance=2cm,auto]
\node[state] (q_0) {$s_i$};
\node[state] (q_1) at (3,0) {$s_j$};

\path[->,line width=0.25mm] (q_0) edge node[] {$\sigma/\delta$} (q_1);
%\path[->] (q_0) edge node[swap] {$\sigma$} (q_1);
\end{tikzpicture}
\caption{ssi: $f(s_i,\sigma)=s_j \hspace{0.5cm} g(s_i,\sigma)=\delta$ }
\end{figure}
\end{itemize}
\subsubsection{Comportamiento Dinámico}
Sea $M=(k,\Sigma,\Delta,f,g)$ una máquina: \\
\begin{center}
\begin{tikzpicture}[>={Triangle[width=1.5mm,length=1.5mm]},->,node distance=2cm,auto]
\node[] (s_0) {$s_0$};
\node[] (s_1) at (2,0) {$s_1$};
\node[] (s_2) at (4,0) {$s_2$};
\node[] (sus) at (6,0) {$\cdots$};
\node[] (s_k) at (8,0) {$s_k$};

\node[] (t_0) at (0,1) {$t_0$};
\node[] (t_1) at (2,1) {$t_1$};
\node[] (t_2) at (4,1) {$t_2$};
\node[] (tsus) at (6,1) {$\cdots$};
\node[] (t_k) at (8,1) {$t_k$};

\path[->,line width=0.25mm] (s_0) edge node[] {$\sigma_0/\delta_0$} (s_1);
\path[->,line width=0.25mm] (s_1) edge node[] {$\sigma_1/\delta_1$} (s_2);
\path[->,line width=0.25mm] (s_2) edge node[] {$\sigma_2/\delta_2$} (sus);
\path[->,line width=0.25mm] (sus) edge node[] {$\sigma_{k-1}/\delta_{k-1}$} (s_k);
%\path[->] (q_0) edge node[swap] {$\sigma$} (q_1);
\end{tikzpicture}
\end{center}

\begin{align*}
\hat{f}(s_0,\sigma_0\sigma_2\ldots\sigma_{k-1}) & =s_k \\
      y(s_i,\sigma_i) & =\delta_i \\
      \overline{y}(s_0,\sigma_0\sigma_1\ldots\sigma_k)&=\delta_0\delta_1\ldots\delta_{k-1}
\end{align*}

\subsubsection{Función Estado Terminal}
Sea $M=(k,,\Sigma,\Delta,f,g)$ una máquina.
\begin{enumerate}
\item Una función de Estado Terminal de $M$ es la función de estado terminal:
$$
\widehat{f}: k\times\Sigma^* \rightarrow k \text{ del módulo } (k,\Sigma,f)
$$
\item Una función palabra de salida de $M$ es una única función:
$$
\overline{g}:k\times\Sigma^*\rightarrow\Delta^* \text{ tal que }\forall s\in k,\sigma\in\Sigma, w\in\Sigma^*
$$
$$
\begin{cases}
\overline{g}(s,\lambda)&=\lambda \\
\overline{g}(s,\sigma w)&= g(s,\sigma)\overline{g} \big[ f(s,\sigma),w \big]
\end{cases}
$$
\end{enumerate}
\subsubsection{$\diamond$ Notas}
\begin{itemize}
\item $w=\lambda$
$$
{\color{red}\overline{g}(s,\sigma)}=
\overline{g}(s,\sigma\lambda)=
g(s,\sigma)\underbrace{\overline{g} \big[ f(s,\sigma),\lambda \big]}_{\lambda}=g(s,\sigma)\lambda =
{\color{red}g(s,\sigma)}
$$
\item $\forall s\in k$
$$
g_s:\Sigma^*\rightarrow\Delta^* \text{ tal que } g_s(w)=\overline{g}(s,w)
$$
\end{itemize}
\subsection{Síntesis, Análisis y Verificación}
\begin{center}
\begin{tabular}{|c|c|c|}
\hline 
 & \texttt{I/O} & Máquina \\ 
\hline 
Síntesis & \checkmark & ? \\ 
\hline 
Análisis & ? & \checkmark \\ 
\hline 
Verificación & \checkmark & \checkmark \\ 
\hline 
\end{tabular} 
\end{center}