\chapter{Autómatas}
\section{Autómata Finito Determinístico (AFD)}
\subsection{Definición}
Un Autómata Finito Determinístico (AFD) es una quintupla $M=(k,\Sigma,f,s_0,F)$ donde:
\begin{itemize}
\item $k$ conjunto finito no vacio, \textit{conjunto de estados}
\item $\Sigma$ conjunto finito no vacio, \textit{Alfabeto}
\item $f:k\times\Sigma\rightarrow k$ , \textit{Funcion de transicion}
\item $s_0\in k$, \textit{Estado inicial}
\item $F\subseteq k$, \textit{Conjunto de estados finales}
\end{itemize}
\subsection{Intepretación}
\subsection{Representación}
\subsection{Configuración}
Sea $M=(k,\Sigma,\delta,s_0,F)$ un AFD. \\ $ { } $ \\ 
Una configuración de $M$ es un elemento de $k\times\Sigma^*$
\subsection{Relación }
\section{Autómata Finito no Determinístico (AFN)}
\subsection{Definición}
Un autómata Finito no Deterministico (AFN) es una quintupla $M=(k,\Sigma,\Delta,s,F)$ donde:
\section{Equivalencia entre una AFD y un AFN}
\subsubsection{Teorema}
Para cada AFN existe un AFD equivalente.
\subsubsection{$\bigstar$ Prueba}
Sea $M=(k,\Sigma,\Delta,s,F)$ un AFN
\renewcommand{\labelenumi}{\theenumi}
\renewcommand{\theenumi}{\textbf{\roman{enumi}.)}}
\begin{enumerate}
\item Construimos $M' = (k',\Sigma,\Delta',s',F')$ eliminando todas las aristas de $M$ que:
$$
(q,u,q')\in \Delta \hspace{0.5cm}\wedge\hspace{0.5cm} |u|>1
$$
Si $u=\sigma_1\sigma_2\ldots\sigma_k, k>1$ entonces añadimos $p_1,p_2,\ldots,p_{k-1}$ estados y las nuevas transiciones:
$$
(q,\sigma_1,p_1),(p_1,\sigma_2,p_2),\ldots,(p_{k-1},\sigma_k,q')
$$
a $\Delta$ para $u$ tal que $|u|>1$.
\item Construimos $M''=(k'',\Sigma,\delta'',s'',F'')$
\end{enumerate}