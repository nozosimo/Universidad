\documentclass[10pt,a4paper]{book}
\usepackage[utf8]{inputenc}
\usepackage[spanish]{babel}
\usepackage{amsmath}
\usepackage{amsfonts}
\usepackage{amssymb}
\usepackage{graphicx}
\usepackage{subfiles}
\usepackage{enumitem}
\usepackage[affil-it]{authblk}
\usepackage{array}
\usepackage{tabularx}
\usepackage{multicol}
\usepackage{slashbox}
\usepackage{diagbox}
\usepackage{slashbox,multirow}
\usepackage{enumitem}
\usepackage{mathtools}
\usepackage{pstricks}
\usepackage{pgfplots}
\usepackage{babel}
\usepackage{tikz}
\usetikzlibrary{babel}
\usetikzlibrary{positioning}
\usepackage{calc}
\usepackage{ifthen}
\usepackage{mathrsfs} %Contiene el Signo de Transformada de Laplace
\usepackage{empheq}
\usepackage{svg}
\usepackage{hyperref}
\hypersetup{colorlinks=true,allcolors=gray}
\usepackage{hypcap}
\DeclarePairedDelimiter{\Ex}{[}{]}
\newcommand{\E}{\Ex}

\pgfplotsset{compat=1.16}

\newcommand\blfootnote[1]{%
  \begingroup
  \renewcommand\thefootnote{}\footnote{#1}%
  \addtocounter{footnote}{-1}%
  \endgroup
}

\newcolumntype{C}[1]{>{\centering\arraybackslash}p{#1}}
\usepackage[left=2cm,right=2cm,top=2cm,bottom=2cm]{geometry}

\pgfmathdeclarefunction{gauss}{2}{%
  \pgfmathparse{1/(#2*sqrt(2*pi))*exp(-((x-#1)^2)/(2*#2^2))}%
}

\newcommand{\varstackrel}[3][T]{\stackrel{\raisebox{0.5ex}{\clap{\scriptsize#2}}}{#3}}

\begin{document}
\subfile{Archivos/Introduccion}
\begin{tikzpicture}
\matrix [column sep=7mm, row sep=5mm] {
  \node (p1) [draw, shape=rectangle,align=center] {Definición \\ del Problema}; & \\
  \node (p2) [draw, shape=rectangle] {Acopio de Datos}; \\
  \node (p3) [draw, shape=rectangle] {Formulación de un Mod
  elo Matemático}; \\
  \node (p4) [draw, shape=rectangle] {Resolución del Modelo Matemático}; \\
  \node (p5) [draw, shape=rectangle] {Interpretación de Resultados}; \\
  \node (p6) [draw, shape=rectangle] {Ejecución}; \\
};
\draw[->, thick] (p1) -- (p2);
\draw[->, thick] (p2) -- (p3);
\draw[->, thick] (p3) -- (p4);
\draw[->, thick] (p4) -- (p5);
\draw[->, thick] (p5) -- (p6);
\end{tikzpicture}
\chapter{Programación Lineal}
\section{Modelo Matemático}
Compuesto por:
\begin{itemize}
\item \textbf{Variables:} $x_1,x_2,\ldots ,x_n$.
\item \textbf{Función Objetivo:} Es aquella que buscamos \textit{maximizar} o \textit{minimizar}\footnote{$c_{ij}=$ costos (si minimizamos) ó ganancias (si maximizamos)}. 
$$z = c_1 x_1 + c_2 x_2 + \cdots +c_n x_n$$
\item \textbf{Restricciones:}
\begin{align*}
a_1 x_1 + a_2 x_2 + \cdots + a_n x_n \geq o \leq A\\
b_1 x_1 + b_2 x_2 + \cdots + b_n x_n \geq o \leq B
\end{align*}
\end{itemize}
\subsection{Métodos de Solución}
\subsubsection{Método Gráfico}
Es mas didáctico, solo lo usamos con dos variables y cuando ambos valores son positivos solo usamos el primer cuadrante del plano cartesiano.
\subsubsection{Método SIMPLEX}
\end{document}