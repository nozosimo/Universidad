\documentclass[10pt,a4paper]{book}
\usepackage[utf8]{inputenc}
\usepackage[spanish]{babel}
\usepackage{amsmath}
\usepackage{amsfonts}
\usepackage{amssymb}
\usepackage{graphicx}
\usepackage{subfiles}
\usepackage{enumitem}
\usepackage[affil-it]{authblk}
\usepackage{array}
\usepackage{tabularx}
\usepackage{multicol}
\usepackage{slashbox}
\usepackage{diagbox}
\usepackage{slashbox,multirow}
\usepackage{enumitem}
\usepackage{mathtools}
\usepackage{pstricks}
\usepackage{pgfplots}
\usepackage{babel}
\usepackage{tikz}
\usetikzlibrary{babel}
\usetikzlibrary{calc,trees,positioning,arrows,chains,shapes.geometric,%
    decorations.pathreplacing,decorations.pathmorphing,shapes,%
    matrix,shapes.symbols}
\usepackage{calc}
\usepackage{ifthen}
\usepackage{mathrsfs} %Contiene el Signo de Transformada de Laplace
\usepackage{empheq}
\usepackage{svg}
\usepackage{hyperref}
\hypersetup{colorlinks=true,allcolors=gray}
\usepackage{hypcap}
\DeclarePairedDelimiter{\Ex}{[}{]}
\newcommand{\E}{\Ex}

\pgfplotsset{compat=1.16}

\newcommand\blfootnote[1]{%
  \begingroup
  \renewcommand\thefootnote{}\footnote{#1}%
  \addtocounter{footnote}{-1}%
  \endgroup
}

\newcolumntype{C}[1]{>{\centering\arraybackslash}p{#1}}
\usepackage[left=2cm,right=2cm,top=2cm,bottom=2cm]{geometry}

\pgfmathdeclarefunction{gauss}{2}{%
  \pgfmathparse{1/(#2*sqrt(2*pi))*exp(-((x-#1)^2)/(2*#2^2))}%
}

\newcommand{\varstackrel}[3][T]{\stackrel{\raisebox{0.5ex}{\clap{\scriptsize#2}}}{#3}}

\tikzset{
>=stealth',
  punktchain/.style={
    rectangle, 
    rounded corners, 
    % fill=black!10,
    draw=black, very thick,
    text width=10em, 
    minimum height=3em, 
    text centered, 
    on chain},
  line/.style={draw, thick, <-},
  element/.style={
    tape,
    top color=white,
    bottom color=blue!50!black!60!,
    minimum width=8em,
    draw=blue!40!black!90, very thick,
    text width=10em, 
    minimum height=3.5em, 
    text centered, 
    on chain},
  every join/.style={->, thick,shorten >=1pt},
  decoration={brace},
  tuborg/.style={decorate},
  tubnode/.style={midway, right=2pt},
}

\begin{document}
\subfile{Archivos/Introduccion}
\chapter{Programación Lineal}
En Investigación Operativa se sigue el siguiente cuadro a la hora de trabajar:
\begin{center}
\scalebox{0.6}{
\begin{tikzpicture}
  [node distance=.5cm,
  start chain=going right,]
     \node[punktchain, join] (intro) {Definición \\ del Problema};
     \node[punktchain, join] (probf)      {Acopio de \\ Datos};
     \node[punktchain, join] (investeringer)      {Formulación de un \\ Modelo Matemático};
     \node[punktchain, join] (perfekt) {Resolución del \\ Modelo Matemático};
     \node[punktchain, join] (emperi) {Interpretación de \\ los resultados};
     \node[punktchain, join] (asym) [punktchain ]  {Ejecución};
  \end{tikzpicture}
  }
\end{center}

\section{Modelo Matemático}
Compuesto por:
\begin{itemize}
\item \textbf{Variables:} $x_1,x_2,\ldots ,x_n$.
\item \textbf{Función Objetivo:} Es aquella que buscamos \textit{maximizar} o \textit{minimizar}\footnote{$c_{ij}=$ costos (si minimizamos) ó ganancias (si maximizamos)}. 
$$z = c_1 x_1 + c_2 x_2 + \cdots +c_n x_n$$
\item \textbf{Restricciones:}
\begin{align*}
a_1 x_1 + a_2 x_2 + \cdots + a_n x_n \geq o \leq A\\
b_1 x_1 + b_2 x_2 + \cdots + b_n x_n \geq o \leq B
\end{align*}
\end{itemize}
\subsection{Métodos de Solución}
\subsubsection{Método Gráfico}
Es mas didáctico, solo lo usamos con dos variables y cuando ambos valores son positivos solo usamos el primer cuadrante del plano cartesiano.
\subsubsection{Método SIMPLEX}
\subsection{Soluciones}
\begin{itemize}
\item \textbf{Solución Básica:}
\item \textbf{Solución Factible:}
\item \textbf{Solución Básica Factible:}
\item \textbf{Solución Optima:}
\end{itemize}
\end{document}