\documentclass[10pt,letterpaper]{article}
\usepackage[utf8]{inputenc}
\usepackage[spanish]{babel}
\usepackage{amsmath}
\usepackage{amsfonts}
\usepackage{amssymb}
\usepackage{graphicx}\usepackage{listings}
\usepackage{color}
\usepackage{hyperref} 
\usepackage{multicol}
\usepackage{xcolor}
\usepackage[left=2cm,right=2cm,top=1cm,bottom=2cm]{geometry}
\title{ \texttt{\normalsize Tema\# 3} \\ Lab. 3: Más sobre ciclos básicos en Prolog \vspace{-1cm} }

\definecolor{codegreen}{rgb}{0,0.6,0}
\definecolor{codegray}{rgb}{0.5,0.5,0.5}
\definecolor{codepurple}{rgb}{0.58,0,0.82}
\definecolor{backcolour}{rgb}{0.98,0.98,0.98}

\lstdefinestyle{mystyle}{
    backgroundcolor=\color{backcolour},   
    commentstyle=\color{codegreen},
    keywordstyle=\color{magenta},
    numberstyle=\tiny\color{codegray},
    stringstyle=\color{codepurple},
    basicstyle={\small\ttfamily},
    breakatwhitespace=false,         
    breaklines=true,                 
    captionpos=b,                    
    keepspaces=true,                 
    numbers=left,                    
    numbersep=5pt,                  
    showspaces=false,                
    showstringspaces=false,
    showtabs=false,                  
    tabsize=2
}
 
\lstset{style=mystyle}

\begin{document}
\maketitle
\subsection*{Integrantes}
\begin{itemize}
\item Leonardo Henry Añez Vladmirovna
\item Gerson Oliva Rojas
\item Pedro Luis Caricari Torrejón
\item Erick Edwing Vidal Céspedes
\end{itemize}
\subsection*{Porcentaje Completado: 100\%}
\subsection*{Comentario(s):}
\subsection*{Source Code:}
1. \texttt{mostrarDivisoresDesc(n)} : Procedimiento que muestra los divisores del entero n. Muestra desde el n a 1. \\

\texttt{Python (Iterativo)}
\lstinputlisting[language=Python]{p1a.py}

\texttt{Python (Recursivo)}
\lstinputlisting[language=Python]{p1b.py}

\texttt{Prolog}
\lstinputlisting[language=Prolog]{p1c.pl}
${}$\\
\pagebreak 

2. \texttt{mostrarDivisoresComunes(n, m)} : Procedimiento que muestra los divisores comunes de los entero n y m. \\

3. \texttt{mostrarDivisoresPares(n)} : Procedimiento que muestra los divisores pares de n. \\

4. \texttt{mostrarDivisoresImpares(n)} :  Procedimiento que muestra los divisores impares de n. \\

5. \texttt{mostrarDivisores(n, a, b)} :  Procedimiento que muestra los divisores de n, comprendidos entre a y b inclusive. \\

6. \texttt{primo(n) } : Función lógica que devuelve True, si el entero n es número primo. \\

\texttt{Python (Iterativo)}
\lstinputlisting[language=Python]{p6a.py}

\texttt{Python (Recursivo)}
\lstinputlisting[language=Python]{p6b.py}

\texttt{Prolog}
\lstinputlisting[language=Prolog]{p6c.pl}

7. \texttt{proximoPrimo(n)} : Función que devuelve, el siguiente primo después de n. Si n es primo, devuelve n. \\

\pagebreak

14. \texttt{mostrarFib(n)}: Procedimiento que muestra los primeros n términos de la secuencia de Fibonacci. \\
\texttt{Python (Iterativo)}
\lstinputlisting[language=Python]{p14a.py}

\texttt{Python (Recursivo)}
\lstinputlisting[language=Python]{p14b.py}

\end{document}