\documentclass[10pt,letterpaper]{article}
\usepackage[utf8]{inputenc}
\usepackage[spanish]{babel}
\usepackage{amsmath}
\usepackage{amsfonts}
\usepackage{amssymb}
\usepackage{graphicx}\usepackage{listings}
\usepackage{color}
\usepackage{hyperref} 

\definecolor{mygreen}{rgb}{0,0.6,0}
\definecolor{mygray}{rgb}{0.5,0.5,0.5}
\definecolor{miplomo}{rgb}{0.98,0.98,0.98}
\definecolor{mymauve}{rgb}{0.58,0,0.82}
\usepackage[left=2cm,right=2cm,top=2cm,bottom=2cm]{geometry}
\title{Tarea\# 1: Propuesta de Bases de Conocimiento}
\lstset{ 
  backgroundcolor=\color{miplomo},   % choose the background color; you must add \usepackage{color} or \usepackage{xcolor}; should come as last argument
  basicstyle=\footnotesize,        % the size of the fonts that are used for the code
  breakatwhitespace=false,         % sets if automatic breaks should only happen at whitespace
  breaklines=true,                 % sets automatic line breaking
  captionpos=b,                    % sets the caption-position to bottom
  commentstyle=\color{mygreen},    % comment style
  deletekeywords={...},            % if you want to delete keywords from the given language
  escapeinside={\%*}{*)},          % if you want to add LaTeX within your code
  extendedchars=true,              % lets you use non-ASCII characters; for 8-bits encodings only, does not work with UTF-8
  firstnumber=1,                % start line enumeration with line 1000
  frame=single,	                   % adds a frame around the code
  keepspaces=true,                 % keeps spaces in text, useful for keeping indentation of code (possibly needs columns=flexible)
  keywordstyle=\color{blue},       % keyword style
  language=Octave,                 % the language of the code
  morekeywords={*,...},            % if you want to add more keywords to the set
  numbers=left,                    % where to put the line-numbers; possible values are (none, left, right)
  numbersep=5pt,                   % how far the line-numbers are from the code
  numberstyle=\tiny\color{mygray}, % the style that is used for the line-numbers
  rulecolor=\color{white},         % if not set, the frame-color may be changed on line-breaks within not-black text (e.g. comments (green here))
  showspaces=false,                % show spaces everywhere adding particular underscores; it overrides 'showstringspaces'
  showstringspaces=false,          % underline spaces within strings only
  showtabs=false,                  % show tabs within strings adding particular underscores
  stepnumber=1,                    % the step between two line-numbers. If it's 1, each line will be numbered
  stringstyle=\color{mymauve},     % string literal style
  tabsize=2,	                   % sets default tabsize to 2 spaces
  title=\lstname                   % show the filename of files included with \lstinputlisting; also try caption instead of title
}


\begin{document}
\maketitle
\subsection*{Integrantes}
\begin{itemize}
\item Leonardo Henry Añez Vladmirovna
\item Gerson Oliva Rojas
\item Pedro Luis Caricari Torrejón
\item Erick Edwing Vidal Céspedes
\end{itemize}
\textbf{Porcentaje Completado:} 100\% \\
\textbf{Comentario(s):} En este trabajo hemos reforzado el conocimiento aprendido en clases, haciendo uso de las reglas de proposicionales, como ser la negación, (\texttt{not}), conjunción (\texttt{,}) y disyunción (\texttt{;}). Ademas de utilizar reglas secundarias para facilitar el uso de otras. Ademas (aunque no es parte del avance por ahora), intentamos usar listas para tratar un problema de otra manera, para ello utilizamos la explicación que da Muhammad Awais Shaikh, en su repositorio para realizar una regla que pregunta por un elemento en una lista (ver referencias).

\lstinputlisting[language=Prolog]{biblioteca.pl}
\lstinputlisting[language=Prolog]{escuela.pl}
\lstinputlisting[language=Prolog]{pokemon.pl}
\lstinputlisting[language=Prolog]{this_war_of_mine.pl}
\lstinputlisting[language=Prolog]{computadores.pl}

\section*{Referencias}
\begin{itemize}
\item Asignatura Nº3 NOSWEGO (Universidad Estatal de Nueva York, Departamento de Ciencias de la Computación \url{http://cs.oswego.edu/~speel/csc366/Website/Assignment%203/pokemonEnglish.pdf}
\item Repositorio (Github, Muhammad Awais): Prolog \url{https://github.com/muhammadawaisshaikh/prolog}
\item \url{https://this-war-of-mine.fandom.com/wiki/This_War_of_Mine_Wiki}
\end{itemize}
\end{document}