\documentclass[12pt,letterpaper]{article}
\usepackage[utf8]{inputenc}
\usepackage[spanish]{babel}
\usepackage{amsmath}
\usepackage{amsfonts}
\usepackage{amssymb}
\usepackage{graphicx}
\usepackage[left=2cm,right=2cm,top=2cm,bottom=2cm]{geometry}
\author{
{\Large \textbf{Grupo Nº 5}}\\${ }$\\
\textbf{Integrantes:}\\
\begin{tabular}{|c|c|}
\hline 
Nombre & Registro \\ 
\hline 
Leonardo H. Añez Vladimirovna & 217002498 \\ \hline
Leonardo H. Añez Vladimirovna & 217002498 \\ \hline
Leonardo H. Añez Vladimirovna & 217002498 \\ \hline
Leonardo H. Añez Vladimirovna & 217002498 \\ \hline
Leonardo H. Añez Vladimirovna & 217002498 \\ \hline
Leonardo H. Añez Vladimirovna & 217002498 \\ \hline
Leonardo H. Añez Vladimirovna & 217002498 \\ \hline
Leonardo H. Añez Vladimirovna & 217002498 \\ 
\hline 
\end{tabular} 
}
\title{
{\large \texttt{ADM100 - GRUPO SC}}\\ {\normalsize \textbf{Docente:} Oscar Flores} \\ \vspace{3cm}
{\Huge El Control}
}

%\setcounter{secnumdepth}{0} % sections are level 1

\begin{document}
\maketitle

\newpage

\section{Introducción}
El control ha sido definido bajo dos grandes perspectivas, una perspectiva limitada y una perspectiva amplia. Desde la perspectiva limitada, el control se concibe como la verificación a posteriori de los resultados conseguidos en el seguimiento de los objetivos planteados y el control de gastos invertido en el proceso realizado por los niveles directivos donde la estandarización en términos cuantitativos, forma parte central de la acción de control.
\section{Concepto}
El control es una etapa primordial en la administración, pues, aunque una empresa cuente con magníficos planes, una estructura organizacional adecuada y una dirección eficiente, el ejecutivo no podrá verificar cuál es la situación real de la organización si no existe un mecanismo que se cerciore e informe si los hechos van de acuerdo con los objetivos.\\${ }$\\
El concepto de control es muy general y puede ser utilizado en el contexto organizacional para evaluar el desempeño general frente a un plan estratégico.
A fin de incentivar que cada uno establezca una definición propia del concepto se revisara algunos planteamientos de varios autores estudiosos del tema:
\begin{itemize}
\item \textbf{Henry Fayol:} El control consiste en verificar si todo ocurre de conformidad con el plan adoptado, con las instrucciones emitidas y con los principios establecidos. Tiene como fin señalar las debilidades y errores a fin de rectificarlos e impedir que se produzcan nuevamente.
\item \textbf{Robert B. Buchele:} El proceso de medir los actuales resultados en relación con los planes, diagnosticando la razón de las desviaciones y tomando las medidas correctivas necesarias.
\item \textbf{George R. Terry:} El proceso para determinar lo que se está llevando a cabo, valorización y, si es necesario, aplicando medidas correctivas, de manera que la ejecución se desarrolle de acuerdo con lo planeado.
\end{itemize}
\section{Importancia del Control}
Una de las razones más evidentes de la importancia del control es porque hasta el mejor de los planes se puede desviar. El control se emplea para:
\begin{itemize}
\item \textbf{Crear mejor calidad:} Las fallas del proceso se detectan y el proceso se corrige para eliminar errores.
\item \textbf{Enfrentar el cambio:} Este forma parte ineludible del ambiente de cualquier organización. Los mercados cambian, la competencia en todo el mundo ofrece productos o servicios nuevos que captan la atención del público. Surgen materiales y tecnologías nuevas. Se aprueban o enmiendan reglamentos gubernamentales. La función del control sirve a los gerentes para responder a las amenazas o las oportunidades de todo ello, porque les ayuda a detectar los cambios que están afectando los productos y los servicios de sus organizaciones.
\item \textbf{Producir ciclos más rápidos:} Una cosa es reconocer la demanda de los consumidores para un diseño, calidad, o tiempo de entregas mejorados, y otra muy distinta es acelerar los ciclos que implican el desarrollo y la entrega de esos productos y servicios nuevos a los clientes. Los clientes de la actualidad no solo esperan velocidad, sino también productos y servicios a su medida.
\end{itemize}
\subsection{Bases del Control}
\begin{itemize}
\item	Los objetivos son los programas que desea lograr la empresa, los que facilitarán alcanzar la meta de esta. Lo que hace necesaria la planificación y organización para fijar qué debe hacerse y cómo.
\item 	El hacer es poner en práctica el cómo se planificó y organizó la consecución de los objetivos. De éste hacer se desprende una información que proporciona detalles sobre lo que se está realizando, o sea, ella va a esclarecer cuáles son los hechos reales. Esta información debe ser clara, práctica y actualizada al evaluar.
\item	El evaluar que no es más que la interpretación y comparación de la información obtenida con los objetivos trazados, se puedan tomar decisiones acerca de que medidas deben ser necesarias tomar.
\item	La mejora es la puesta en práctica de las medidas que resolverán las desviaciones que hacen perder el equilibrio al sistema.
\end{itemize}
\subsection{Areas del Control}
El control actúa en todas las áreas y en todos los niveles de la empresa. Prácticamente todas las actividades de una empresa están bajo alguna forma de control o monitoreo. Las principales áreas de control en la empresa son:
\begin{itemize}
\item \textbf{Áreas de producción:} Si la empresa es industrial, el área de producción es aquella donde se fabrican los productos; si la empresa fuera prestadora de servicios, el área de producción es aquella donde se prestan los servicios.
\item \textbf{Control de producción:} El objetivo fundamental de este control es programar, coordinar e implantar todas las medidas tendientes a lograr un optima rendimiento en las unidades producidas, e indicar el modo, tiempo y lugar más idóneos para lograr las metas de producción, cumpliendo así con todas las necesidades del departamento de ventas. 
\end{itemize}
\section{Reglas del Proceso de Control}
\begin{enumerate}
\item	\textit{Establecimiento de los medios de control.}
\item	\textit{Operaciones de recolección de datos.}
\item	\textit{Interpretación y valoración de los resultados.}
\item	\textit{Utilización de los mismos resultados.}
\end{enumerate}
\subsection{Elementos del Control}
El control es un proceso cíclico y repetitivo. Está compuesto de cuatro elementos que se suceden:
\begin{itemize}
\item \textbf{Establecimiento de estándares:} Es la primera etapa del control, que establece los estándares o criterios de evaluación o comparación. Un estándar es una norma o un criterio que sirve de base para la evaluación o comparación de alguna cosa. Existen cuatro tipos de estándares; los cuales se presentan a continuación:
\begin{itemize}
\item \textit{Estándares de cantidad:} Como volumen de producción, cantidad de existencias, cantidad de materiales primas, números de horas, entre otros.
\item \textit{Estándares de calidad:} Como control de materia prima recibida, control de calidad de producción, especificaciones del producto, entre otros.
\item \textit{Estándares de tiempo:} Como tiempo estándar para producir un determinado producto, tiempo medio de existencias de un productos determinado, entre otros.
\item \textit{Estándares de costos:} Como costos de producción, costos de administración, costos de ventas, entre otros.
\end{itemize}
\item \textbf{Evaluación del desempeño:} Es la segunda etapa del control, que tiene como fin evaluar lo que se está haciendo.
\item \textbf{Comparación del desempeño con el estándar establecido:} Es la tercera etapa del control, que compara el desempeño con lo que fue establecido como estándar, para verificar si hay desvío o variación, esto es, algún error o falla con relación al desempeño esperado.
\item \textbf{Acción correctiva:} Es la cuarta y última etapa del control que busca corregir el desempeño para adecuarlo al estándar esperado. La acción correctiva es siempre una medida de corrección y adecuación de algún desvío o variación con relación al estándar esperado.
\end{itemize}
\end{document}
