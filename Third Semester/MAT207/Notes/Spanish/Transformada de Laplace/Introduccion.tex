% Partes Incluidas en este Archivo
% Portada
% Notas del Autor
% Tabla de Contenidos


% Caratula 
\title{Transformada de Laplace}

\author{Leonardo H. Añez Vladimirovna}
\affil{Universidad Autónoma Gabriél René Moreno,\\Facultad de Ingeniería en Ciencias de la Computación y Telecomunicaciones, \\Santa Cruz de la Sierra, Bolivia}
\date{\today}

\maketitle

Estos apuntes fueron realizados durante mis clases en la materia \texttt{MAT207 (Ecuaciones Diferenciales)}, en el período \texttt{I-2018} en la Facultad de Ingeniería en Ciencias de la Computación y Telecomunicaciones. 
\\ \vspace{0.5cm} \\
Para cualquier cambio, observación y/o sugerencia pueden enviarme un mensaje al siguiente correo:
\begin{center}
 \texttt{toborochi98@outlook.com}
\end{center}

\subsection*{Definición} Sea $F(t)$ definida para $t>0$. Entonces la Integral:

$$\mathscr{L} \lbrace F(t)\rbrace = f(s) = \displaystyle\int_{0}^{\infty} e^{-st}\cdot F(t) dt$$
Es la \textit{Transformada de Laplace} siempre que la Integral Converja. Veamos un ejemplo: 
\subsubsection*{Ejemplo}
Se tiene $F(t)=1$, hallar $\mathscr{L} \lbrace F(t)\rbrace$, comenzamos reemplazando en la integral:
$$\mathscr{L} \lbrace F(t)\rbrace=\displaystyle\int_{0}^{\infty} e^{-st}\cdot 1 dt$$
Expresamos la integral en forma de límite:
$$\displaystyle\int_{0}^{\infty} e^{-st} dt = \lim_{b\to\infty} \displaystyle\int_{0}^{b} e^{-st} dt$$
Realizamos la Integral:
$$\lim_{b\to\infty} \displaystyle\int_{0}^{b} e^{-st} dt = \lim_{b\to\infty}\left[ \left( -\dfrac{1}{s}\cdot e^{-st} \right)\Bigg|_0^b \right] $$
Evaluando la Integral en $b$ y $0$:
$$\lim_{b\to\infty}\left[  -\dfrac{1}{s}\cdot \left( e^{-sb} - e^0\right) \right] = \lim_{b\to\infty}\left[  -\dfrac{1}{s}\cdot \left( \dfrac{1}{e^{sb}} - 1\right) \right] $$
Aplicamos el límite:
$$\lim_{b\to\infty}\left[  -\dfrac{1}{s}\cdot \left( \dfrac{1}{e^{sb}} - 1\right) \right]=-\dfrac{1}{s}\cdot(0-1)= \fbox{$\dfrac{1}{s}$}$$
Por lo tanto:
$$\mathscr{L} \lbrace 1\rbrace=f(s)=\dfrac{1}{s}$$
\pagebreak
\subsection*{Transformada de Laplace de algunas Funciones Elementales}
De igual manera que el ejemplo del anterior punto, estas Transformadas se pueden obtener:
\begin{multicols}{2}
\begin{center}
$F(t)$
\end{center}
\begin{enumerate}[1)]
\item $1$
\item $t$ 
\item $t^n$
\item $e^{at}$
\item $sen(at)$
\item $cos(at)$
\item $senh(at)$
\item $cosh(at)$
\columnbreak
$$\mathscr{L} \lbrace F(t) \rbrace=f(s)$$
\item[] $\dfrac{1}{s} \hspace{0.25cm};\hspace{0.5cm}s>0$
\item[] $\dfrac{1}{s^2} \hspace{0.25cm};\hspace{0.5cm}s>0$
\item[] $\dfrac{n!}{s^{n+1}} \hspace{0.25cm};\hspace{0.5cm}s>0$
\item[] $\dfrac{1}{s-a}$
\item[] $\dfrac{a}{s^2+a^2}$
\item[] $\dfrac{s}{s^2+a^2}$
\item[] $\dfrac{a}{s^2-a^2} \hspace{0.25cm};\hspace{0.5cm}s>|a|$
\item[] $\dfrac{s}{s^2-a^2} \hspace{0.25cm};\hspace{0.5cm}s>|a|$
\end{enumerate}
\end{multicols}
\subsection*{Propiedades}
\subsubsection*{Teorema}
Si $C_1,C_2$ son constantes y $F_1(t),F_2(t)$ funciones con transformadas de Laplace: $f_1(s),f_2(s)$ respectivamente, entonces:
$$\mathscr{L} \lbrace C_1 F_1(t)+C_2 F_2(t) \rbrace = C_1\mathscr{L}\lbrace F_1(t)\rbrace + C_2\mathscr{L}\lbrace F_2(t)\rbrace$$
Esto es:
$$\mathscr{L} \lbrace C_1 F_1(t)+C_2 F_2(t) \rbrace = C_1 f_1(s) +  C_2 f_2(s)$$
\subsection*{Transformadas de Laplace de la Derivadas}
\subsubsection*{T.1.}
Si $\mathscr{L} \lbrace F(t) \rbrace = f(s)$ entonces:
$$\mathscr{L} \lbrace F'(t) \rbrace = s\cdot f(s) - F(0)$$
Asumiendo:
\begin{center}
$e^{-st} \to 0$ cuando $t\to\infty$
\end{center}
\subsubsection*{T.2.}
Si $\mathscr{L} \lbrace F(t) \rbrace = f(s)$ entonces:
$$\mathscr{L} \lbrace F''(t) \rbrace = s^2\cdot f(s) - s\cdot F(0)-F'(0)$$
\subsubsection*{T.3.}
Si $\mathscr{L} \lbrace F(t) \rbrace = f(s)$ entonces:
$$\mathscr{L} \lbrace F'''(t) \rbrace = s^3\cdot f(s) - s^2 \cdot F(0)-s\cdot F'(0)-F''(0)$$
\subsection*{Generalizando}
$$\mathscr{L} \lbrace F(t) \rbrace = s^n\cdot f(s)-s^{n-1}F(0)-s^{n-2}F'(0)-\cdots -s\cdot F^{n-2}(0)-F^{n-1}(0)$$
\subsection*{Transformada Inversa}
Si $\mathscr{L}\left\lbrace F(t) \right\rbrace = f(s)$ entonces $f(s)$ es la transformada inversa de Laplace. Esto es:
$$\mathscr{L}^{-1} \lbrace f(s) \rbrace = F(t)$$
Para la inversa se pueden utilizar las transformadas normales, teniendo en cuenta que ahora trabajamos con las inversas.