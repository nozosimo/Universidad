\chapter{Conceptos Generales}
Se dice que una ecuación es Diferencial si contiene una función desconocida y una o mas de sus derivadas. Si las ecuaciones contienen derivadas de funciones que dependen de una sola variable independiente se tiene una \textit{Ecuación Diferencial Ordinaria.} Si la función depende de varias variables independientes se tienen \textit{Ecuaciones Diferenciales en Derivadas Parciales.}

\subsubsection{Teorema}
Si se tiene la Ecuación:
$$F(x,y,y',y'',\ldots,y^{(n)})=0$$
Si se logra conseguir una función $\alpha(x)$ tal que, al reemplazar en la Ecuación:
$$F(x,\alpha(x),\alpha'(x),\alpha''(x),\ldots,\alpha^{(n)}(x))=0$$
Entonces se dice que $\alpha(x)$ es \textbf{Solución} de la Ecuación Diferencial.
\subsubsection*{Familia de Curvas}
Una ED en su forma mas simple tiene la siguiente apariencia:
$$\dfrac{dy}{dx}=f(x)$$
Para resolver esta ED debemos realizar el proceso inverso de la derivada, es decir integrar:
$$\int\dfrac{dy}{dx}dx=\int f(x) dx\Rightarrow y=F(x)+C$$
Debe notarse que debido a $C$ esta solución es en realidad un número infinito de soluciones. A esto se le llama \textbf{Familia de Curvas}. Una solución particular de esta ecuación, es una sola curva contenida en la familia de curvas.
\subsection*{Orden, Grado y Linealidad de una Ecuación Diferencial}
\begin{itemize}
\item \textbf{Orden:} El Orden de una ED\footnote{ED = Ecuaciones Diferenciales (notación que usaremos de aquí en adelante)} es el orden de la derivada mas alta de la función desconocida (variable dependiente) que aparece en la ecuación.
\item \textbf{Grado:} El grado se expresa mediante el mayor exponente de la derivada de mayor orden.
\item \textbf{Linealidad:} Una ED Lineal de orden $n$ es una ecuación de la forma:
$$a_n(x)\dfrac{d^ny}{dx^n}+a_{n-1}(x)\dfrac{d^{n-1}y}{dx^{n-1}}+\cdots + a_2(x)\dfrac{d^2y}{dx^2}+a_1(x)\dfrac{dy}{dx}+a_0(x)y=h(x)$$
Cuyos coeficientes $a_0(x),a_1(x),\ldots,a_n(x)$ y el segundo miembro $h(x)$ son continuos en un Intervalo $I$ en el que la $a_n(x)\neq 0$. Si $h(x)=0$ la Ecuación se llama \textit{Ecuación Diferencial de Orden $n$ Homogénea.}
\end{itemize}
\section{Formas de Resolución}
\subsection{Ecuaciones con Variables Separables}
Una ED de 1er Orden $\frac{dy}{dx}=f(x,y)$ es separable, si la función $f(x,y)$ se puede escribir como un producto de una función de $x$ y una función de $y$.
$$\dfrac{dy}{dx}=f(x,y)\Rightarrow \dfrac{dy}{dx}=f(x)g(y)$$
\section{Ecuaciones Homogéneas}
Una función $M(x,y)$ es Homogénea de grado $n$ si la suma de las potencias de $x$ y $y$ en cada termino es $n$. Esto es:
$$M(\lambda x,\lambda y)=\lambda^nM(x,y)$$
Una ED Homogénea es una ED de la forma:
$$\dfrac{dy}{dx}=\dfrac{M(x,y)}{N(x,y)}$$
Donde $M$ y $N$ son funciones homogéneas del mismo grado.
\subsection{Método de Solución}
Las ED Homogéneas se resuelven haciendo la sustitución:
$$y=Vx \hspace{1.5cm} V=\dfrac{y}{x}$$
donde $V$ es una función de $x$ por lo tanto:
$$\dfrac{dy}{dx}=V+x\dfrac{dV}{dx}$$
El miembro que corresponde a la función se puede expresar en términos de $\frac{y}{x}$. Esto se hace dividiendo todo entre $x^n$. Donde $n$ es el grado de: $N$ y $M$.
\chapter{Ecuación Lineal de Primer Orden}
Recordando la definición de una ED de orden $n$:
$$a_n(x)\dfrac{d^ny}{dx^n}+a_{n-1}(x)\dfrac{d^{n-1}y}{dx^{n-1}}+\cdots + a_2(x)\dfrac{d^2y}{dx^2}+a_1(x)\dfrac{dy}{dx}+a_0(x)y=h(x)$$
tenemos, para 1er Orden:
\begin{equation}
a_1(x)\dfrac{dy}{dx}+a_0(x)y=h(x)
\end{equation}
Dividiendo la ED entre el coeficiente de $a_1(x)$:
$$\dfrac{dy}{dx}+\underbrace{\dfrac{a_0(x)}{a_1(x)}}_{P(x)}y=\underbrace{\dfrac{h(x)}{a_1(x)}}_{Q(x)}$$
Con esto tenemos:
\begin{equation}
\dfrac{dy}{dx}+P(x)y=Q(x)
\end{equation}
\subsection*{Método de Solución}
Partimos de:
$$\dfrac{d}{dx}\Bigg[\mu(x)\cdot y\Bigg]=\mu(x)\dfrac{dy}{dx}+\mu(x)P(x) \cdot y = \mu(x)P(x) $$

Tenemos:
\[
\dfrac{d}{dx}\Bigg[\mu(x)\cdot y\Bigg]=\mu(x)\dfrac{dy}{dx}+y  \overbrace{\dfrac{d\mu}{dx}}\tikzmark{0}=\mu(x)\dfrac{dy}{dx}+ \overbrace{\tikzmark{1}\mu(x)P(x)}\cdot y =  \mu(x)Q(x)
\]
%
\begin{tikzpicture}[remember picture, overlay, bend left=30,line width=0.25mm]
  \draw ([yshift=4ex]pic cs:0) to ([yshift=4ex]pic cs:1);
\end{tikzpicture}
A partir de esta igualdad obtendremos lo que se llama como \textit{Factor Integrando:}
$$\dfrac{d\mu}{dx}=\mu(x)P(x)$$
Para simplificar, escribiremos $\mu(x)$ simplemente como $\mu$, sobreentenderemos que esta en función de $x$:

\begin{align*}
\dfrac{d\mu}{dx} & = \mu \cdot P(x) \\{ }\\
\int\dfrac{1}{\mu} & = \int P(x) dx \\{ }\\
e^{\ln(\mu)} & = e^{\int P(x) dx}
\end{align*}
\begin{equation}
\mu  = e^{\int P(x) dx}
\end{equation}
Una vez hallado el FI, procedemos a reemplazar en la ecuación:
\begin{align*}
\dfrac{d}{dx}\Big[ \mu(x)\cdot y \Big] & = \mu(x)\cdot Q(x) \\
\dfrac{d}{dx}\Big[ e^{\int P(x)dx} y \Big] & = e^{\int P(x) dx} Q(x)
\end{align*}
Integrando:
\begin{align*}
e^{\int P(x) dx} \cdot y & = \int e^{\int P(x) dx} \cdot Q(x) dx
\end{align*}
$$ \boxed{y  = e^{-\int P(x) dx} \int e^{\int P(x) dx} \cdot Q(x) dx}$$
\section{Ecuación de Bernoulli}
Tiene la siguiente forma:
$$\dfrac{dy}{dx}+P(x)\cdot y =Q(x)\cdot y^n $$
Para poder tratar la ecuación, dividimos entre $y^n$:
$$y^{-n}\cdot\dfrac{dy}{dx}+P(x)\cdot y^{1-n} =Q(x)$$
\section{Ecuaciones Diferenciales Exactas}
La ecuación:
\begin{equation}
M(x,y)dx+N(x,y)dy=0
\end{equation}
Es exacta se se puede escribir en la forma:
$$\dfrac{d}{dx}\Big[ f(x,y) \Big] = 0$$
\subsubsection{Criterio}
La diferencia total:
\begin{equation}
\dfrac{\partial f}{\partial x}dx - \dfrac{\partial f}{\partial y}dy=0
\end{equation}

Para $(1\texttt{.}4)$ y $(1\texttt{.} 5)$ serán lo mismo:
\begin{equation}
\dfrac{\partial f}{\partial x}=M(x,y) \hspace{1cm} \wedge \hspace{1cm} \dfrac{\partial f}{\partial y}=N(x,y)
\end{equation}

$$\dfrac{\partial^2 f}{\partial x \partial y}=M(x,y) \hspace{1cm} \wedge \hspace{1cm} \dfrac{\partial^2 f}{\partial y\partial x}=N(x,y)$$
\subsubsection{Por Clairaut}
Si existen las segundas derivadas cruzadas y son continuas entonces estas son iguales:
$$\dfrac{\partial^2 f}{\partial y\partial x}=\dfrac{\partial^2 f}{\partial x\partial y}$$
\subsubsection{Solución}
\begin{align*}
\dfrac{\partial f}{\partial x} & = M(x,y)
\end{align*}
\begin{equation}
f(x,y)= \int M(x,y) dx + C(y)
\end{equation}
\begin{align*}
\dfrac{\partial f}{\partial y} & = \dfrac{\partial}{\partial y}\left[\int M(x,y)\right]  dx + C'(y) 
\end{align*}
Recordando $1\texttt{.}6$:
\subsection{Ecuación Diferencial Asociada}
El objetivo es obtener la ED de una familia de curvas, mediante el proceso de eliminación de constantes arbitrarias involucradas en la familia de curvas. Es importante recordar que ninguna ED contiene constantes arbitrarias por lo tanto debemos eliminar la constante que aparece en la familia de curvas.
\subsubsection{Procedimiento}
Dada la familia de curvas:
$$F(x,y,C_1,C_2,\ldots,C_n)=0$$
Para obtener la ED asociada a la familia de Curvas se deben utilizar las constantes en el sistema:
\begin{align*}
F(x,y,C_1,C_2,\ldots,C_n) & = 0 \\
\dfrac{d}{dx}\left[ F(x,y,C_1,C_2,\ldots,C_n)\right]  & = 0 \\
\dfrac{d^2}{dx^2}\left[ F(x,y,C_1,C_2,\ldots,C_n)\right]  & = 0 
\end{align*}
\begin{center}
$\vdots $
\end{center}
\begin{align*}
\dfrac{d^n}{dx^n}\left[ F(x,y,C_1,C_2,\ldots,C_n)\right]  & = 0 
\end{align*}
\section{Trayectorias Ortogonales}
\subsection{Determinación}
\subsubsection{Primer Paso}
Dada una familia de curvas con parámetro $C$ se encuentra su ED de la forma:
$$y'=f(x,y)$$
\subsubsection{Segundo Paso}
Se encuentra las trayectorias ortogonales resolviendo la ED:
$$y' = -\dfrac{1}{f(x,y)}$$
\chapter{Ecuaciones Diferenciales Homogéneas}
\section{Ecuaciones Diferenciales Homogéneas con Coeficientes Constantes}
\begin{equation}
ay'' + by' +cy = 0
\end{equation}
Donde $a,b$ y $c$ son constantes. \\${ }$\\
Ecuaciones de este tipo tienen dos soluciones independientes.
\subsection{Solución General}
En general, si $y=y_1(x)$ y $y=y_2(x)$ son ambas soluciones de $1\texttt{.}8$ entonces, cualquier combinación lineal de $y_1$ y $y_2$ es también solución de $1\texttt{.}8$. Esto quiere decir:
\begin{align*}
 ay_1''(x)+by_1'(x) +cy_1(x)=0 \\
ay_2''(x)+by_2'(x) +cy_2(x)=0
\end{align*}
Sea: $y=Ay_1(x)+By_2(x)$ una combinación lineal de $y_1$ y $y_2$.
\section{Ecuaciones Diferenciales Homogéneas con Coeficientes Constantes de Orden $n$}
$$a_ny^{(n)}+a_{n-1}y^{(n-1)}+\cdots + a_0 y = 0$$
Donde: $a_i$ son constantes y $a_n\neq 0$ \\${ }$\\
El procedimiento es similar al de orden 2. \\${ }$\\
Si $y=e^{mx}$ es una solución, entonces $m$ es una raíz de la Ecuación Característica.
$$\boxed{a_nm^n+a_{n-1}m^{n-1}+\cdots +a_1 m + a_0 = 0}$$
El polinomio de grado $n$ tiene $n$ raíces, pero no necesitan ser distinas, la multiplicidad de una raíz es el número de veces que se repite. La suma de todas las multiplicidades de todas las raíces distintas es igual al grado del polinomio.
\subsection{Procedimiento}
\chapter{Ecuaciones Diferenciales no Homogéneas}
\chapter{Sistemas de Ecuaciones Diferenciales}
\section{Sistemas no Homogéneos}
\chapter{Otros Métodos de Solución}
\section{Transformada de Laplace}