\section{Medidas de Dispersión Relativa}
Se utilizan para comparar dos o mas distribuciones en torno a la media.
\subsection{Coeficiente de Variación $(C_v)$}
Es una medida de dispersión relativa muy útil cuando:
\begin{itemize}
\item Los datos están en unidades diferentes.
\item Los datos están en las mismas unidades, pero las medias muy distantes.
\end{itemize}
\subsubsection{Cálculo del Coeficiente de Variación}
$$C_v=\dfrac{S}{\overline{x}} \hspace{1cm}\vee\hspace{1cm}  C_v=\dfrac{S}{\overline{x}}100\hspace{1cm}\textrm{Tanto Porciento} $$
\subsection{Momentos}
Son de uso frecuente los promedios de las series de potencias de las variables estos promedios reciben el nombre de \textit{Momentos.} Pueden definirse respecto a cualquier punto:
\begin{itemize}
\item Momentos Respecto a un punto
\item Momentos Respecto del Origen
\item Momentos Respecto a la Media o Momentos Centrales
\end{itemize}
\subsubsection{Momentos Respecto a un Punto $(M_{r,A})$}
\begin{multicols}{2}
\textbf{Datos Originales} 
\begin{center}
$M_{r,A}=\dfrac{\displaystyle\sum_{i=1}^{n}(x_i-A)^r}{n}$
\end{center}
\columnbreak
\textbf{Datos Agrupados}
$$M_{r,A}=\dfrac{\displaystyle\sum_{i=1}^{n}(y_i-A)^r\cdot f_i}{n}$$
\end{multicols}
\subsubsection{Momentos Respecto del Origen $(M_r')$}
\begin{multicols}{2}
\textbf{Datos Originales} 
\begin{center}
$M_{r}'=\dfrac{\displaystyle\sum_{i=1}^{n}(x_i)^r}{n}$
\end{center}
\columnbreak
\textbf{Datos Agrupados}
$$M_{r}'=\dfrac{\displaystyle\sum_{i=1}^{n}(y_i)^r\cdot f_i}{n}$$
\end{multicols}
\subsubsection{Momentos Respecto a la Media $(M_r)$}
\begin{multicols}{2}
\textbf{Datos Originales} 
\begin{center}
$M_{r}=\dfrac{\displaystyle\sum_{i=1}^{n}(x_i-\overline{x})^r}{n}$
\end{center}
\columnbreak
\textbf{Datos Agrupados}
$$M_{r}=\dfrac{\displaystyle\sum_{i=1}^{n}(y_i-\overline{y})^r\cdot f_i}{n}$$
\end{multicols}
