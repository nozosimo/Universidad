\section{Definición Clásica de Probabilidad}
$$P(A)=\dfrac{n(A)}{n(S)}=\dfrac{\textrm{\# de casos favorables del Evento $A$}}{\textrm{\# total de casos posibles}}$$
\subsection{Probabilidad Condicionada}
Sean $A$ y $B$ eventos tal que $P(B)>0$ la probabilidad condicional de que ocurra $A$ dado que ha ocurrido el evento $B$ se denota por: $P(A/B)$. Se define como:
$$P(A/B)=\dfrac{P(A\cap B)}{P(B)}$$
Si $B$ es un evento tal que $P(B)>0$, entonces $P(\bullet/B)$ satisface los siguientes axiomas:
\begin{enumerate}[label=(\roman*)]
\item $0\leq P(A/B)\leq 1$
\item $P(S/B)=1$
\item $P\left( \bigcup\limits_{i=1}^{n} A_i/B\right)=
\displaystyle\sum_{i=1}^{n}P(A_i/B) \hspace{1cm}A_i \textrm{son eventos mutuamente excluyentes}$
\end{enumerate}
\subsubsection{Propiedades}
Si $B$ es un evento tal que $P(B)>0$, entonces $P(\bullet/B)$ tiene las siguientes propiedades:\blfootnote{En $P(\bullet/B)$ el símbolo $\bullet$ significa \textit{cualesquiera}}
\begin{enumerate}
\item $P(\phi/B)=0$
\item $P(A^c/B)=1-P(A/B)\hspace{0.5cm} \vee \hspace{0.5cm} P(A/B)=1-P(A^c/B) $
\item $P((A\cup C)/B)=P(A/B)+P(C/B)-P((A\cap C)/B)$
\item $A\in C \Rightarrow P(A/B)\leq P(C/B)$
\end{enumerate}
\subsection{Eventos Dependientes}
Se dice que los eventos $A$ y $B$ son estadísticamente dependientes, si:

\begin{equation*}
\begin{rcases}
  P(A\cap B)=P(B)\cdot P(A/B)\\
  P(A\cap B)=P(A)\cdot P(B/A)
\end{rcases}
\text{Regla General de la Multiplicación}
\end{equation*}
\subsection{Eventos Independientes}
Se dice que los eventos $A$ y $B$ son estadísticamente independientes, si:
$$P(A\cap B)=P(A)\cdot P(B)$$
\subsubsection{Observaciones}
\begin{enumerate}
\item $\begin{rcases}
  P(A)\\
  P(B)
\end{rcases} \text{Probabilidades Marginales}$
\item $\begin{rcases}
  P(A/B)\\
  P(B/A)
\end{rcases} \text{Probabilidades Condicionales}$
\item $P(A/B)\neq P(B/A)$
\item $P(A \cap B) \hspace{0.5cm} \text{Probabilidades Condicionales}$
\item Decir \textit{'Eventos Independientes'} no es lo mismo que decir \textit{'Evento Mutuamente Excluyente'}.
\end{enumerate}
\subsubsection{Teorema}
Si $A$ y $B$ son eventos mutuamente independientes, entonces:
\begin{enumerate}
\item $P(A/B)=P(A)$
\item $P(B/A)=P(B)$
\item $A^c$ y $B$ son independientes.
\item $A$ y $B^c$ son independientes.
\item $A^c$ y $B^c$ son independientes.
\end{enumerate}