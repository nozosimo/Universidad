\section{Mediana $(M_e)$}
Es el valor de la variable que supera a no mas de la mitad de las observaciones y es superado por no mas de la mitad de las observaciones.
\subsection{Cálculo}
\subsubsection{Datos no Agrupados}
En el cálculo de la mediana para datos no agrupados hay que distinguir 2 observaciones:
\begin{itemize}
\item \textbf{Número de Observaciones Impares:} \\${ }$\\ Si 
$$x_1,x_2,x_3,\ldots,x_n$$
Son los valores de las variables ordenados creciente o decreciente entonces la mediana es el valor de la variable situado en el centro del conjunto de variables, esto es:
$$Me=x_{\frac{n+1}{2}}$$
\item \textbf{Número de Observaciones Par:} En este caso esta definido por:
$$Me=\dfrac{x_{\frac{n}{2}}+x_{\frac{n}{2}+1}}{2}$$
\end{itemize}
\subsubsection{Datos Agrupados}
\begin{itemize}
\item \textbf{Agrupados no en Intervalo de Clase:}
\item \textbf{Agrupados en Intervalo de Clase:}
$$Me=y_{j-1}'+C_j\cdot \dfrac{\frac{n}{2}-F_{j-1}}{f_j}$$
\end{itemize}
\subsubsection{Mediana en Distribuciones Simétricas}
\subsection{Ejemplos}