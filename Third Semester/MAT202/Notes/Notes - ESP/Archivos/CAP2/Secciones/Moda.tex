\section{Moda o Modo $(M_o)$}
La Moda de una Muestra:
$$x_1,x_2,x_3,\ldots,x_n$$
Es aquel valor de la variable que representa la mayor frecuencia, es decir, el valor de la variable que mas veces se repite. En términos sociológicos la moda es aquella de mayor aceptación popular, la moda es usada generalmente para representar y describir una tendencia en la economía.
\subsubsection{Observación}
La moda no siempre existe y no siempre es única.
Como ejemplo podemos tener los siguientes conjuntos:
\begin{enumerate}
\item $\{ 2,4,5,6,3,7,3,8,3\}$
\item $\{5,7,4,5,8,4,5,9,4\}$
\item $\{1,2,3,4,5,6,7\}$
\end{enumerate}
Para el primer conjunto, la moda es $M_o=3$ ya que es el dato que mas se repite de las observaciones. Del segundo ejemplo, la moda es $M_o=4$ y $M_o=5$, ya que ambos datos son los que mas se repiten. Finalmente, en el tercer ejemplo, no hay moda.
\subsection{Cálculo}
\subsubsection{En Intervalo de Clase}
\begin{itemize}
\item \textbf{De Amplitud Constante:}
$$M_o=y_{j-1}' + C_j\cdot\dfrac{f_j-f_{j-1}}{(f_j-f_{j-1})+(f_j-f_{j+1})}$$
\item \textbf{De Amplitud Variable:}
$$M_o=y_{j-1}' + C_j\cdot\dfrac{\dfrac{f_{j+1}}{a_{j+1}}}{\dfrac{f_{j+1}}{a_{j+1}}+\dfrac{f_{j-1}}{a_{j-1}}}$$
esto es:
$$M_o=y_{j-1}' + C_j\cdot\dfrac{h_{j+1}}{h_{j+1}+h_{j-1}}$$
\end{itemize}

\subsection{Ventajas y Desventajas}
\begin{multicols}{2}
\textbf{Ventajas}
\begin{itemize}
\item El valor de la moda es totalmente independiente de los valores extremos.
\item La moda se puede utilizar como una localización para datos cualitativos como cuantitativos.
\item La moda es posible calcular aun en distribuciones que tienen intervalo de clase abierto.
\end{itemize}
\columnbreak
\textbf{Desventajas}
\begin{itemize}
\item Es una medida inestable porque varía cuando se combina el intervalo de clase.
\item Su significación es limitada, cuando no se dispone de una gran número de valores.
\item La moda no se presta a manipulaciones algebraicos posteriores.
\item Cuando el conjunto de observaciones contiene mas de una moda es difícil interpretar y de comprender.
\end{itemize}
\end{multicols}