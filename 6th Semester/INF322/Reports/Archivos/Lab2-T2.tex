El precio medio ponderado (PMP) es un método de valoración en contabilidad que se obtiene al realizar un cálculo del valor medio de las existencias que había al inicio y de las entradas ponderadas según sus cantidades.

\section*{Fórmula del precio promedio ponderado (PPP)}

Para calcular el Precio Promedio Ponderado (PPP) es necesario conocer esta sencilla formula. Explicando los términos que la componen, tenemos:
\begin{itemize}
\item $\sum P_i$ = Suma del precio de cada bien.
\item $Q_i$ = Cantidad de bienes comprados
\item $\sum Q_i$ = Cantidad total de bienes
\end{itemize}
$$
	PPP = \dfrac{\sum P_i \cdot Q_i}{\sum Q_i}
$$
\section*{¿Qué supone el uso del PPP?}

Por las propias condiciones propias del modelo, la utilización del precio promedio ponderado supone un continuo control de las existencias para su correcto funcionamiento, debido a que es necesario tener los datos sobre las cantidades almacenadas y las diferentes entradas y salidas de forma totalmente controlada.

Además, este criterio de valoración de existencias es bastante confiable en periodos de tiempo de estabilidad en los precios, ya que reduce los efectos producidos por las posibles oscilaciones de estos.

\pagebreak
\section*{Implementacion}
\begin{lstlisting}[language=SQL]
USE preventas;
/*
	Funcion que calcula el PPP(Precio Promedio Ponderado)
*/
GO
CREATE FUNCTION PA_PPP(@cprd INT, @calm INT)
RETURNS DECIMAL(12,2)
AS
BEGIN
	DECLARE @suma_total DECIMAL(12,2)
	DECLARE @cant_total DECIMAL(12,2)
	DECLARE @cant DECIMAL(12,2)
	DECLARE @prec DECIMAL(12,2)
	DECLARE @prec_final DECIMAL(12,2)
	SET @suma_total = 0
	SET @cant_total = 0
	
	DECLARE c_cursor CURSOR FOR SELECT sumi.cant, sumi.prec
		FROM sumi
		WHERE sumi.calm = @calm AND sumi.cprd = @cprd
	
	OPEN c_cursor
	FETCH c_cursor INTO @cant, @prec
	WHILE @@FETCH_STATUS = 0
	BEGIN
		SET @suma_total = @suma_total + (@cant * @prec)
		SET @cant_total = @cant_total + @cant
		FETCH c_cursor INTO @cant, @prec
	END

	CLOSE c_cursor
	DEALLOCATE c_cursor
	
	SET @prec_final = @suma_total / @cant_total
	
	RETURN @prec_final
END
GO

\end{lstlisting}