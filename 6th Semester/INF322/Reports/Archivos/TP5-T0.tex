\begin{enumerate}
\item ¿Cual es la diferencia entre el Lenguaje SQL y el Lenguaje Transact SQL?

La diferencia es que con Transact SQL podemos realizar todas las cosas que podriamos hacer en un lenguaje de programacion ordinario, como bucles, if's, etc

\item ¿Que se puede programar con Transact SQL?

Con T-SQL se pueden programar las unidades de programa dentro del \textit{SGBD}, como ser:
\begin{itemize}
\item Scripts
\item Funciones
\item Procedimientos Almacenados
\item Triggers
\end{itemize}


\item Liste los tipos de datos que maneja Transact SQL

\subsubsection*{Enteros}
\begin{center}
\begin{tabular}{|c|c|c|}
\hline
\textbf{Tipos} & \textbf{Desde} & \textbf{Hasta} \\ \hline
\texttt{bigint}     & -9,223,372,036,854,775,808 & 9,223,372,036,854,775,807 \\ \hline
\texttt{int}        & -2,147,483,648             & 2,147,483,647             \\ \hline
\texttt{smallint}   & -32,768                    & 32,767                    \\ \hline
\texttt{tinyint}    & 0                          & 255                       \\ \hline
\texttt{bit}        & 0                          & 1                         \\ \hline
\texttt{decimal}    & $-10^{38} +1$              & $10^{38} +1$              \\ \hline
\texttt{numeric}    & $-10^{38} +1$              & $10^{38} +1$              \\ \hline
\texttt{money}      & -922,337,203,685,477.5808  & +922,337,203,685,477.5807 \\ \hline
\texttt{smallmoney} & -214,748.3648              & +214,748.3647             \\ \hline
\end{tabular}
\end{center}
\vspace{-0.5cm}
\subsubsection*{Flotantes}
\begin{center}
\begin{tabular}{|c|c|c|}
\hline
\textbf{Tipos} & \textbf{Desde} & \textbf{Hasta} \\ \hline
\texttt{Float}     & -1.79E + 308 & 1.79E + 308 \\ \hline
\texttt{Real}        & -3.40E + 38            & 3.40E + 38            \\ \hline
\end{tabular}
\end{center}
\vspace{-0.5cm}
\subsubsection*{Fechas}

\begin{center}
\begin{tabular}{|c|c|c|}
\hline
\textbf{Tipos} & \textbf{Desde} & \textbf{Hasta} \\ \hline
\texttt{datetime}     & Jan 1, 1753& Dec 31, 9999 \\ \hline
\texttt{smalldatetime}        & Jan 1, 1900            & Jun 6, 2079           \\ \hline
\texttt{date}        & Jan 1, 0001           & Dec 31, 9999           \\ \hline
\end{tabular}
\end{center}
\vspace{-0.5cm}
\subsubsection*{Caracteres y Strings}

\begin{center}
\begin{tabular}{|c|c|}
\hline
\textbf{Tipo}     & \textbf{Caracteres}  \\ \hline
\texttt{char}     & 8000  \\ \hline
\texttt{varchar}  & 8000  \\ \hline
\texttt{text}     & 2,147,483,647  \\ \hline
\end{tabular}
\end{center}
\vspace{-0.5cm}

\subsubsection*{Otros}

\begin{itemize}
\item \texttt{timestamp}
\item \texttt{xml}
\item \texttt{cursos}
\item \texttt{table}
\end{itemize}

\item ¿Como se declara una variable en Transact SQL? cite ejemplos:

Se declara siguiendo el siguiente modelo:

\begin{center}
\texttt{DECLARE @<nombre>}
\end{center}

Algunos ejemplos son los siguientes:
\begin{lstlisting}[language=SQL]
-- Declarando un string
DECLARE @nombre char(64)
-- Declarando un entero
DECLARE @x int
-- Declarando un flotante
DECLARE @pi float
\end{lstlisting}

\item ¿Como se asigna valor a una variable en Transact SQL? cite ejemplos:

Se declara siguiendo el siguiente modelo:

\begin{center}
\texttt{SET @<nombre> = <valor>}
\end{center}

Algunos ejemplos son los siguientes:
\begin{lstlisting}[language=SQL]
-- Asignando un string
SET @nombre = '8086'
-- Asignando un entero
SET @x = 256
-- Asignando un flotante
SET @pi = 3.14156
\end{lstlisting}

\item ¿Como asignar valores obtenidas de una consulta a una variable en Transact SQL? cite ejemplos:

\begin{lstlisting}[language=SQL]
DECLARE @codigo int,
	@nombre char(40),
	@ciudad char(2)

SELECT @codigo = calm, @nombre =noma, @ciudad = ciud
FROM alma
WHERE calm = 1

PRINT @codigo
PRINT @nombre
PRINT @ciudad
\end{lstlisting}

\item ¿Que es un \texttt{CURSOR} en Transact SQL?

Un cursor es una variable.\\

\item ¿Para que sirve el uso de \texttt{CURSOR} en Transact SQL?

Nos permite recorrer con un conjunto de resultados obtenido a través de una sentencia \texttt{SELECT} fila a fila. \\

\item ¿Como se declare un \texttt{CURSOR} en Transact SQL?

Siguiendo el siguiente esquema:
\begin{center}
\texttt{DECLARE @<nombre>  CURSOR FOR <sentenciasql>}
\end{center}

\item ¿Como se abre un \texttt{CURSOR} en Transact SQL?

\begin{center}
\texttt{OPEN @<nombre>}
\end{center}

\item ¿Para que sirve la instrucción \texttt{FETCH} en Transact SQL?

Recupera una fila específica de un cursor del servidor Transact-SQL.

\item ¿Para que sirve la variable  \texttt{@@FECT\_STATUS}, que valores devuelve?

Esta función devuelve el estado de la última instrucción \texttt{FETCH} del cursor emitida contra cualquier cursor abierto actualmente por la conexión.

\item ¿Como se cierra un \texttt{CURSOR} en Transact SQL?

\begin{flalign*}
\texttt{CLOSE $<$nombre$>$}  \\
\texttt{DEALLOCATE $<$nombre$>$} 
\end{flalign*}

\item Cite un ejemplo de como leer todas las filas de una tabla en Transact SQL

\begin{lstlisting}[language=SQL]
DECLARE @nombre CHAR(40)
	@color VARCHAR(15)

DECLARE CDATOS CURSOR
FOR SELECT nomp,colo FROM prod
OPEN CDATOS
FETCH CDATOS INTO @nombre,@color

WHILE (@@FETCH_STATUS=0)
  BEGIN
    PRINT @nombre + @color
    FETCH CDATOS INTO @nombre,@color
  END
CLOSE CDATOS
DEALLOCATE
\end{lstlisting}

\item Liste los principales operadores utilizados en Transact SQL

\begin{itemize}
\item \textbf{Asignación} $=$
\item \textbf{Aritméticos} $+,-,*,/,**,\%$
\item \textbf{Relacionales} $=,<>,!=,<,>,>=,<=,!>,!<$
\item \textbf{Lógicos} $\texttt{AND},\texttt{NOT},\texttt{OR},\&,|$
\item \textbf{Otros} \texttt{ALL, ANY,BETWEEN, EXISTS,IN,LIKE,NOT,SOME}
\end{itemize}

\item ¿Como funciona la estructura de control \texttt{IF THEN}? cite ejemplo:
\begin{flalign*}
\texttt{IF $<$expresion$>$}  \\
\texttt{BEGIN \hspace{1cm}} \\
\texttt{END \hspace{1.35cm}}    \\
\texttt{ELSE IF $<$expresion$>$}  \\
\texttt{BEGIN \hspace{1cm}} \\
\texttt{END \hspace{1.35cm}}  
\end{flalign*}

\begin{lstlisting}[language=SQL]
DECLARE @edad int = 5

IF @edad>=18
  BEGIN
	 PRINT 'Mayor de Edad'
  END
ELSE
  BEGIN
    PRINT 'Menor de Edad'
  END
\end{lstlisting}

\item ¿Como funciona la estructura de control \texttt{CASE}? cite ejemplo:

Tiene la siguiente estructura, donde se analizan diferentes casos para un determinado dato, y en cada caso devuelve cierto valor, en caso de no ser ninguno devuelve un valor por defecto.
\begin{flalign*}
\texttt{CASE $<$expresion$>$ \hspace{6cm}}  \\
\texttt{WHEN $<$ valor\_expresion $>$ THEN $<$ valor\_devuelto $>$} \\
\texttt{WHEN $<$ valor\_expresion $>$ THEN $<$ valor\_devuelto $>$} \\
\texttt{ELSE} \hspace{8.35cm} \\
\texttt{END \hspace{8.5cm}}  
\end{flalign*}
Ejemplo:

\begin{lstlisting}[language=SQL]
DECLARE @dato int
DECLARE @valor char(45)

SET @dato = 2

SET @valor =(
CASE @dato
	WHEN 1 THEN 'A'
	WHEN 2 THEN 'B'
	WHEN 3 THEN 'C'
	ELSE 'NOPE'
END)

PRINT @valor
\end{lstlisting}

\item ¿Como funciona la estructura de control \texttt{WHILE}? cite ejemplo:

Al igual que en otros lenguajes de programacion (C++,C,C\#,Java,etc), tiene la siguiente estructura:
\begin{flalign*}
\texttt{WHILE $<$expresion$>$}  \\
\texttt{BEGIN \hspace{2cm}} \\
\texttt{END \hspace{2.35cm}}   
\end{flalign*}
Ejemplo de un programa que itera hasta 10 e imprime el resultado:

\begin{lstlisting}[language=SQL]
DECLARE @i int
SET @i = 0

WHILE (@i < 10)
  BEGIN
    SET @i = @i + 1
    PRINT @i
  END
\end{lstlisting}

\item ¿Como se hace el control de errores en Transact SQL? cite ejemplo:

Se hace utilizando la estructura de control \texttt{try-catch}

\begin{lstlisting}[language=SQL]
DECLARE @d float
DECLARE @n float
DECLARE @r float

SET @d = 0
SET @n = 255

BEGIN TRY
	SET @r = @n/@d
END TRY
BEGIN CATCH
	PRINT 'Division entre 0 Imposible!'
END CATCH
\end{lstlisting}
En este caso, como la division entre 0 es imposible, no ejecuta lo que esta dentro del try, y va al catch.
\item ¿Para que sirve la variable \texttt{@@ERROR} y que valores devuelve?

La variable global de sistema \texttt{@@ERROR} almacena el número de error producido por la última sentencia Transact SQL ejecutada

\end{enumerate}
