\chapter{Matrices}
\section{Conceptos}
\subsection{Matriz}
Es un arreglo de números. Los números en el arreglo se denominan elementos de la matriz. Para denotarlas se utilizan letras mayúsculas.
\begin{itemize}
\item \textbf{Tamaño u Orden:} Se describe en términos como el número de filas y columnas de la matriz.
\item \textbf{Identificación de Elementos:} Se los identifica indicando la fila y la columna.
\end{itemize}
\section{Operaciones con Matrices}
\subsection{Propiedades de las Operaciones con Matrices}
\begin{itemize}
\item $A+B=B+A$
\item $A+(B+C)=(A+B)+C$
\item $A(BC)=(AB)C$
\item $A(B\pm C)=AB\pm AC$
\item $(B\pm A)C = BA\pm AC$
\item $a(B\pm C)= aB\pm aC$ \hspace{1cm} $a$ es un escalar.
\item $(a\pm b)C = aC\pm bC$ \hspace{1cm} $b$ es un escalar.
\item $(ab)C = a(bC) = b(aC)$
\item $a(BC)=(aB)C=B(aC)$
\item $(A^t)^t=A$
\item $(A\pm B)^t=A^t \pm B^t$
\item $(kA)^t=kA^t$
\item $(AB)^t = B^t A^t$
\end{itemize}
\subsection{Operaciones Elementales}
\begin{itemize}
\item \textbf{Permutaciones:} de una fila (o columna) con otra $P_{ij}$ o $P_{ij}^c$
\item \textbf{Multiplicación:} de una fila (o columna) por un escalar: $k\neq 0$. Cuya notación será: $M_{i(k)}$ o $M_{j(k)}^c$
\end{itemize}
\section{Matrices Cuadradas}
\subsection{Matriz Identidad}
$$I_{(n)}/[I]_{ij}=
\begin{cases} 
1 \text{ si } i=j \\
0 \text{ si } i\neq j
\end{cases}
$$
\subsubsection{Propiedades}
\begin{itemize}
\item Actúa como el elemento neutro en el Producto de Matrices.
$$
\text{Si } A_{(m\times n)} \Rightarrow
\begin{cases} 
AI_{(n)} = A\\
I_{(n)}A = A

\end{cases}
$$
\end{itemize}
\subsection{Matriz Simétrica}
\subsection{Matriz Antisimétrica}
\subsection{Matriz Triangular Superior}
\subsection{Matriz Triangular Inferior}
\subsection{Matriz Diagonal}
\subsubsection{Matriz Escalar}
Es una matriz diagonal donde $[A]_{ij}=k$, siendo $k$ un escalar cualquiera.
\subsection{Matriz Elemental}
Es una Matriz que proviene de la Identidad, al aplicarle una \textbf{única} operación elemental.
\subsection{Matriz no Singular}
Una matriz $A_{(n)}$ es no singular si existe una matriz $B(n)$ tal que:
$$AB=BA=I$$
$B$ es la Inversa de $A$ y se representa por $A^{-1}$:
$$AA^{-1}=A^{-1}A=I$$
Si $A$ no tiene inversa se la denomina \textit{Singular.}
\chapter{Sistemas de Ecuaciones Lineales}
\chapter{Función Determinante}
\chapter{Vectores}
\chapter{Espacios Vectoriales}