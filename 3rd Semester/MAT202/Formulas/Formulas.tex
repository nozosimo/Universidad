\documentclass[10pt,letterpaper]{article}
\usepackage[utf8]{inputenc}
\usepackage[spanish]{babel}
\usepackage{amsmath}
\usepackage{amsfonts}
\usepackage{amssymb}
\usepackage{graphicx}
\usepackage{subfiles}
\usepackage{enumitem}
\usepackage[affil-it]{authblk}
\usepackage{array}
\usepackage{tabularx}
\usepackage{multicol}
\usepackage{slashbox}
\usepackage{diagbox}
\usepackage{slashbox,multirow}
\usepackage{enumitem}
\usepackage{mathtools}
\usepackage{pstricks}
\usepackage{pgfplots}
\usepackage{enumitem}
\usepackage{tikz}
\usepackage{calc}
\usepackage{ifthen}
\usepackage{mathrsfs} %Contiene el Signo de Transformada de Laplace
\usepackage{empheq}
\usepackage{svg}
\usepackage{hyperref}
\usepackage[left=2cm,right=2cm,top=2cm,bottom=2cm]{geometry}
\author{Leonardo H. Añez Vladimirovna \footnote{Para cualquier cambio, observación y/o sugerencia pueden enviarme un mensaje al siguiente correo: \texttt{toborochi98@outlook.com}}}
\title{Fórmulas de Probabilidad y Estadística I \texttt{(MAT202)}\footnote{Esta es una recopilación de las formulas utilizadas en la materia, sin teoría.}\\{\normalsize Facultad de Ingeniería en Ciencias de la Computación y Telecomunicaciones}\\ \vspace{-0.025cm}\normalsize Universidad Autónoma Gabriel René Moreno}
 \pgfplotsset{compat=1.16}
\begin{document}
\maketitle
\begin{multicols}{4}
\begin{flushleft}
\textbf{Marca de Clase}\\
\vspace{0.25cm}
\noindent $y_i=\dfrac{y_{i-1}'+y_i'}{2}$
\end{flushleft}
\columnbreak
\begin{flushleft}
\textbf{Recorrido ($R$)}\\
\begin{itemize}
\item \textbf{Datos Originales:}\\ $R=y_k -y_i$
\item \textbf{Datos Agrupados:}\\ $R=y_k' -y_i'$
\end{itemize}
\end{flushleft}
\columnbreak
\begin{flushleft}
\textbf{Amplitud ($C$)}\\
\vspace{0.25cm}
$C_j=y_j'-y_{j-1}'$


\end{flushleft}
\columnbreak
\begin{flushleft}
\textbf{\# Intervalos}\\
\vspace{0.25cm}
$k=\dfrac{R}{C_j} = \begin{cases} \sqrt{n} & n> 25 \\ 5 & n\leq 25 \end{cases}$\\

\vspace{0.4cm}
$k=1+3.3log(n)$
\end{flushleft}
\end{multicols}

\begin{multicols}{3}
\begin{flushleft}
\textbf{Frec. Abs. Acumulada}\\
\vspace{0.25cm}
$F_i=\displaystyle\sum_{j=1}^{i}f_j$
\end{flushleft}
\columnbreak
\begin{flushleft}
\textbf{Frec. Rel. Simple}\\
\vspace{0.25cm}
$h_j=\dfrac{f_j}{n}$
\end{flushleft}
\columnbreak
\begin{flushleft}
\textbf{Frec. Rel. Acumulada}\\
\vspace{0.25cm}
$H_i=\dfrac{F_i}{n}=\displaystyle\sum_{j=1}^{i}h_j$
\end{flushleft}
\end{multicols}

\section{Media Aritmética Simple}
\begin{multicols}{2}
\subsubsection*{Datos Originales}
\begin{flushleft}
Notación:  $M(x)=\overline{x}=\dfrac{\displaystyle\sum_{i=1}^{n} x_i }{n}$
\end{flushleft}
\columnbreak
\subsubsection*{Datos Agrupados}
\begin{flushleft}
Notación:  $M(y)=\overline{y}=\dfrac{\displaystyle\sum_{i=1}^{k} y_i\cdot f_i }{n}$
\end{flushleft}
\end{multicols}

\subsection{Método Abreviado}


\begin{multicols}{2}
\noindent $1º$
  $$\overline{y}=O_t + \dfrac{\displaystyle\sum_{i=1}^{k} z_i'f_i}{n}$$
  \\
  $2º$
  $$\overline{y}=O_t + C\cdot\dfrac{\displaystyle\sum_{i=1}^{k} z_i''f_i}{n}$$
  Donde:
\begin{itemize}
\item $z_i'=y_i-O_t $
\item $z_i''=\dfrac{y_i-O_t}{C}=\dfrac{z_i'}{C}$
\end{itemize}
  
\columnbreak
\subsubsection*{Distribuciones Simétricas}
\begin{itemize}
\item $k$ impar:  $y_\frac{k+1}{2}$
\item $k$ par:    $y_\frac{k}{2}'$
\end{itemize}
\subsubsection*{Submuestras}
$$\overline{x}=\dfrac{\displaystyle\sum_{i=1}^{r}\overline{x_i}\cdot f_i}{\displaystyle\sum_{i=1}^{r}f_i};\displaystyle\sum_{i=1}^{r}f_i = n$$
\end{multicols}


\section{Media Aritmética Ponderada ($\overline{x_p}$)}
$$ \overline{x_p}=\dfrac{\displaystyle\sum_{i=1}^{n}x_i\cdot w_i }{\displaystyle\sum_{i=1}^{n}w_i} $$

\section{Mediana ($M_e$)}

\begin{multicols}{3}
\subsubsection*{Datos Originales}
\begin{itemize}
\item $n$ impar: $M_e=x_{\frac{n+1}{2}}$
\item $n$ par: $M_e = \dfrac{x_{\frac{n}{2}}+x_{\frac{n}{2}+1}}{2}$
\end{itemize}
\columnbreak
\subsubsection*{Agrupados en Int. de Clase}
\begin{flushleft}
$M_e = y_{j-1}' + C_j \cdot \dfrac{\dfrac{n}{2}-F_{j-1}}{f_j}$
\end{flushleft}
\columnbreak
\subsubsection*{Agrupados No en Int. de Clase}
\begin{flushleft}
$M_e = F_i$ Escogiendo hacia arriba.
\end{flushleft}
\end{multicols}

\subsubsection*{Distribuciones Simétricas}
\begin{multicols}{2}
\begin{flushleft}
$k$ impar: $M_e=y_{\frac{k+1}{2}}$
\end{flushleft}
\columnbreak
\begin{flushleft}
$k$ par: $M_e=y_{\frac{k}{2}}'$
\end{flushleft}
\end{multicols}

\section{Cuantiles}

\begin{align*}
\phantom{x} & \textbf{Datos Originales} & \textbf{Datos Agrupados} \\
 \textbf{Cuartil} \phantom{xxx}& Q_i=x_{\frac{i(n+1)}{4}}; \phantom{x} i=\{1,2,3\} &Q_i= y_{j-1}' + C_j \cdot \dfrac{i\cdot\dfrac{n}{4}-F_{j-1} }{f_j}\\
 \textbf{Decil} \phantom{xxx}& D_i=x_{\frac{i(n+1)}{10}}; \phantom{x} i=\{1,\ldots , 9\} & D_i= y_{j-1}' + C_j \cdot \dfrac{i\cdot\dfrac{n}{10}-F_{j-1} }{f_j} \\
 \textbf{Percentil} \phantom{xxx} & P_i=x_{\frac{i(n+1)}{100}}; \phantom{x} i=\{1,\ldots , 99\}  &P_i= y_{j-1}' + C_j \cdot \dfrac{i\cdot\dfrac{n}{100}-F_{j-1} }{f_j} 
\end{align*}
\subsubsection*{No Exacto}
$Q_i=D_i=P_i=x_i+(x_{i+1}-x_i)\cdot 3$
\subsubsection*{Recorrido Intercuantílico}
$RI=Q_3 -Q_1$
\section{Moda o Modo ($M_o$)}


\noindent\textbf{No en Int. de Clase}\\
$M_o=\text{MAX frecuencia}$

\subsubsection*{En Int. de Clase}
\begin{multicols}{2}
\begin{flushleft}
\begin{itemize}
\item \textbf{Amplitud Constante}
$M_o=y_{j-1}' + C_j\cdot \dfrac{f_j - f_{j-1}}{(f_j-f_{j-1})+(f_j-f{j+1})}$
\columnbreak
\item \textbf{Amplitud Variable}
$M_o=y_{j-1}' + C_j\cdot \dfrac{h_{j+1}}{h_{j+1}+h_{j-1}}$
\end{itemize}
\end{flushleft}
\end{multicols}

\section{Media Geométrica ($M_g$)}

\begin{multicols}{3}
\subsubsection*{Datos no Agrupados}
\begin{itemize}
\item $M_g=\sqrt[n]{\displaystyle\prod_{i=1}^{n} x_i}$
\item $M_g=antilog\left[ \dfrac{1}{n}\displaystyle\sum_{i=1}^{n}log(x_i) \right]$
\end{itemize}
\columnbreak
\subsubsection*{Datos Agrupados}
\begin{itemize}
\item $M_g=\sqrt[n]{\displaystyle\prod_{i=1}^{k} {y_i}^{f_i}}$
\item $M_g=antilog\left[ \dfrac{1}{n}\displaystyle\sum_{i=1}^{k}f_i\cdot log(y_i) \right]$
\end{itemize}
\columnbreak
\subsubsection*{Cálculo de Taza Media}
$P_n=P_o\cdot (1+i)^n$
\end{multicols}

\section{Media Armónica ($M_n$)}

\begin{multicols}{2}
\subsubsection*{Datos no Agrupados}
$M_n=\dfrac{n}{\displaystyle\sum_{i=1}^{n} \dfrac{1}{x_i}}$
\columnbreak
\subsubsection*{Datos Agrupados}
$M_n=\dfrac{n}{\displaystyle\sum_{i=1}^{k} \dfrac{1}{y_i}\cdot f_i}$
\end{multicols}

\section{Media Cuadrática ($\overline{x_c},M_c(x)$)}
\begin{multicols}{2}
\subsubsection*{Datos Originales}
$\overline{x}_c=\sqrt{\dfrac{\displaystyle\sum_{i=1}^{n}{x_i}^2}{n}}$
\columnbreak
\subsubsection*{Datos Agrupados}
$\overline{y}_c=\sqrt{\dfrac{\displaystyle\sum_{i=1}^{k}{y_i}^2\cdot f_i}{n}}$
\end{multicols}

\section{Desviación Media ($DM$)}

\begin{multicols}{2}
\subsubsection*{Para Datos no Agrupados}
$DM=\dfrac{\displaystyle\sum_{i=1}^{n}|x_i-\overline{x}|}{n}$
\columnbreak
\subsubsection*{Para Datos Agrupados}
$DM=\dfrac{\displaystyle\sum_{i=1}^{k}|y_i-\overline{y}|\cdot f_i}{n}$
\end{multicols}


\section{Varianza ($S^2,\widehat{S}^2,V(x)$)}

\begin{multicols}{3}
\subsubsection*{Datos Originales}
\begin{itemize}
\item $V(x)=\dfrac{\displaystyle\sum_{i=1}^{n}(x_i-\overline{x})^2}{n}; \text{ }n\geq 30$
\item $V(x)=\dfrac{\displaystyle\sum_{i=1}^{n}(x_i-\overline{x})^2}{n-1}; \text{ }n <30$
\end{itemize}
\columnbreak
\subsubsection*{Datos Agrupados}
\begin{itemize}
\item $V(y)=\dfrac{\displaystyle\sum_{i=1}^{k}(y_i-\overline{y})^2\cdot f_i}{n}$
\item $V(y)=\dfrac{\displaystyle\sum_{i=1}^{k}(y_i-\overline{y})^2\cdot f_i}{n-1}$
\end{itemize}
\columnbreak
\subsubsection*{Intervarianza (${S_b}^2,V(\overline{y}_h)$)}
${S_b}^2=\dfrac{\displaystyle\sum_{h=1}^{L}(\overline{y}_h-\overline{y})^2\cdot f_h}{n}$
\subsubsection*{Intravarianza (${S_w}^2,M({S_n}^2)$)}
${S_w}^2=\dfrac{\displaystyle\sum_{h=1}^{L}{S_h}^2\cdot f_h}{n}$
\end{multicols}
$\bigstar\text{   } S^2={S_b}^2+{S_w}^2$
\subsubsection*{Metodos Abreviados}
\begin{multicols}{2}
\begin{center}
$S^2 =\dfrac{\displaystyle\sum_{i=1}^{n}{x_i}^2}{n}$
\end{center}
\columnbreak
\begin{center}
$S^2 =\dfrac{\displaystyle\sum_{i=1}^{k}{y_i}\cdot f_i}{n}-\overline{y}$
\end{center}
\end{multicols}
\section{Desviación Estándar (Típica)}
$S=D(x)=\sigma=\sqrt{V(x)}$
\begin{itemize}
\item $D(ax)=|a|\cdot D(x)$
\item $D(ax\pm b)=|a|\cdot D(x)$
\end{itemize}
\section{Coeficiente de Variación ($CV$)}
\begin{multicols}{2}
\begin{center}
$CV=\dfrac{S}{\overline{x}} \phantom{xx}\vee\phantom{xx} 	CV=\dfrac{S}{\overline{x}}\cdot 100\%$
\end{center}
\columnbreak
\begin{center}
$CV=\dfrac{S}{\overline{y}} \phantom{xx}\vee\phantom{xx} 	CV=\dfrac{S}{\overline{y}}\cdot 100\%$
\end{center}
\end{multicols}
\section{Momentos}

\begin{multicols}{3}
\subsubsection*{Respecto a un Punto}
\begin{itemize}
\item \textbf{Datos no Agrupados}
$$M_{r,A}=\dfrac{\displaystyle\sum_{i=1}^{n}(x_i-A)^r}{n}$$
\item \textbf{Datos Agrupados}
$$M_{r,k}=\dfrac{\displaystyle\sum_{i=1}^{k}(y_i-A)^r\cdot f_i}{n}$$
\end{itemize}
\columnbreak
\subsubsection*{Respecto del Origen}
\begin{itemize}
\item \textbf{Datos Originales}
$$M_r'=\dfrac{\displaystyle\sum_{i=1}^{n}(x_i)^r}{n}$$
\item \textbf{Datos Agrupados}
$$M_r'=\dfrac{\displaystyle\sum_{i=1}^{k}(y_i)^r\cdot f_i}{n}$$
\end{itemize}
\columnbreak
\subsubsection*{Centrales}
\begin{itemize}
\item \textbf{Datos Originales}
$$M_r=\dfrac{\displaystyle\sum_{i=1}^{n}(x_i-\overline{x})^r}{n}$$
\item \textbf{Datos Agrupados}
$$M_r=\dfrac{\displaystyle\sum_{i=1}^{k}(y_i-\overline{y})^r\cdot f_i}{n}$$
\end{itemize}
\end{multicols}
\section{Coeficiente de Asimétria}
\begin{multicols}{4}
\subsubsection*{1er Coef. Person\footnote{Para distribuciones Unimodales.}}
$S_p=\dfrac{\overline{x}-M_o}{S}$
\columnbreak
\subsubsection*{2do Coef. Person}
$S_p=\dfrac{3(\overline{x}-M_e)}{S}$
\columnbreak
\subsubsection*{Bowley}
$S_q=\dfrac{Q_3-2Q_2+Q_1}{Q_3-Q_1}$
\columnbreak
\subsubsection*{Fisher}
$S_m=\dfrac{M_3}{S^3}=\dfrac{M_3}{\sqrt{{M_2}^3}}$
\end{multicols}
Donde:
\begin{itemize}
\item $S=\sqrt{\dfrac{\displaystyle\sum_{i=1}^{n}(x_i-\overline{x})^2}{n}}$
\end{itemize}
\section{Indice de Gini}

\begin{multicols}{3}
\begin{flushleft}
$G=\frac{\displaystyle\sum_{i=1}^{k-1} (p_i-q_i)}{\displaystyle\sum_{i=1}^{k-1} p_i}$
\end{flushleft}

\columnbreak
\begin{flushleft}
$p_A=\dfrac{f_i}{\sum f_i}$\\
\vspace{0.2cm}
$q_A=\dfrac{y_i\cdot f_i}{\sum y_i\cdot f_i}$
\end{flushleft}
\columnbreak
$p_i=\sum p_a$ (\% Acumulado)\\

\vspace{0.3cm}
\noindent $q_i=\sum q_a$ (\% Acumulado)
\end{multicols}
%\section{Curtosis en Función del Momento}
\end{document}