\section{Media Armónica $(M_h)$}
Es la media aritmética recíproca de los recíprocos de las variables:
\subsection{Cálculo}
\subsubsection{Para datos Originales}
$$M_h=\dfrac{1}{\dfrac{\frac{1}{x_1}+\frac{1}{x_2}+\frac{1}{x_3}+\ldots+\frac{1}{x_n} }{n}}=\dfrac{1}{\dfrac{\displaystyle\sum_{i=1}^{n}\frac{1}{x_i}}{n}}=\dfrac{n}{\displaystyle\sum_{i=1}^{n}\frac{1}{x_i}}$$
\subsubsection{Para datos Agrupados}
$$M_h=\dfrac{1}{\dfrac{\frac{f_1}{y_1}+\frac{f_2}{y_2}+\frac{f_3}{y_3}+\ldots+\frac{f_k}{y_k} }{n}}=\dfrac{1}{\dfrac{\displaystyle\sum_{i=1}^{k}\frac{f_i}{x_i}}{n}}=\dfrac{n}{\displaystyle\sum_{i=1}^{k}\frac{f_i}{y_i}}$$
\subsection{Usos de la Media Armónica}
\begin{enumerate}
\item Es apropiado para promediar velocidades.
\item En economía se usa en los índices de precio.
\item Se usa para promediar precio cuando la cantidad es variable y el monto gastado igual.
\end{enumerate}
\subsection{Ventajas y Desventajas}
\begin{multicols}{2}
\textbf{Ventajas}
\begin{itemize}
\item Esta basada en todas las observaciones.
\item Es capaz de un tratamiento algebraico.
\item No se ve significativamente afectado por la fluctuación del muestreo.
\item Es un promedio apropiado para promediar tasas.
\item No le da mucho peso a los artículos grandes.
\end{itemize}
\columnbreak
\textbf{Desventajas}
\begin{itemize}
\item No se puede calcular si alguno de los elementos es cero.
\item En ciertas situaciones es difícil de calcular.
\item Por lo general, es un valor que no existe en los datos proporcionados.
\end{itemize}
\end{multicols}
\subsection{Ejemplos}