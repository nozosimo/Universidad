\section{Media Geométrica $(M_g)$}
\subsection{Cálculo}
\subsubsection{Para datos no Agrupados}
$$M_g=\sqrt[n]{x_1\cdot x_2 \cdot x_3 \cdot \ldots \cdot x_n}$$
En notación:
$$M_g = \sqrt[n]{\prod_{i=1}^{n}x_i}$$
Usando logarítmos:
$$\log(M_g)=\log\Bigg(\sqrt[n]{\prod_{i=1}^{n}x_i}\Bigg)$$
$$\log(M_g)=\dfrac{1}{n}\log\Bigg(\prod_{i=1}^{n}x_i\Bigg)$$
Aplicando propiedades de logaritmos:
$$M_g=antilog\Bigg[\dfrac{1}{n}\sum_{i=1}^{n}\log(x_i)\Bigg]$$
\subsubsection{Para datos Agrupados}
$$M_g=\sqrt[n]{y_1^{f_1}\cdot y_2^{f_2} \cdot y_3^{f_3} \cdot \ldots \cdot y_n^{f_k}}$$
En notación:
$$M_g = \sqrt[n]{\prod_{i=1}^{k}y_i^{f_i}}$$
Usando logaritmos\footnote{El procedimiento es el mismo que para datos Originales}
$$M_g=antilog\Bigg[\dfrac{1}{n}\sum_{i=1}^{k}f_i \log(y_i)\Bigg]$$

\subsection{Propiedades y Usos}
\begin{enumerate}
\item Los promedios geométricos dan menos importancia a las desviaciones extremas que los promedios aritméticos.
\item No se puede deducir, cuando la serie presenta un valor $0$ o valores negativos.
\item Es apropiado cuando hay que promediar razones y proporciones.
\item La aplicación mas útil es para promediar tazas de cambio.
\end{enumerate}
Cuando el cálculo de la taza media de cambio abarca un número considerable de años se lo hace usando la siguiente formula:
$$P_n=P_o(1+i)^n$$
Donde:
\begin{itemize}
\item $P_o$: Cantidad al Principio del periodo.
\item $P_n$: Cantidad al final del periodo.
\item $i$: Tasa de cambio expresada en tanto por uno.
\item $n$: Número de periodos.
\end{itemize}
\subsection{Ejemplos}