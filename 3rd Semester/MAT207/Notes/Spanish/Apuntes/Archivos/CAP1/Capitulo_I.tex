\chapter{Conceptos Generales}
Se dice que una ecuación es Diferencial si contiene una función desconocida y una o mas de sus derivadas. Si las ecuaciones contienen derivadas de funciones que dependen de una sola variable independiente se tiene una \textit{Ecuación Diferencial Ordinaria.} Si la función depende de varias variables independientes se tienen \textit{Ecuaciones Diferenciales en Derivadas Parciales.}

\subsubsection{Teorema}
Si se tiene la Ecuación:
$$F(x,y,y',y'',\ldots,y^{(n)})=0$$
Si se logra conseguir una función $\alpha(x)$ tal que, al reemplazar en la Ecuación:
$$F(x,\alpha(x),\alpha'(x),\alpha''(x),\ldots,\alpha^{(n)}(x))=0$$
Entonces se dice que $\alpha(x)$ es \textbf{Solución} de la Ecuación Diferencial.
\subsubsection*{Familia de Curvas}
Una ED en su forma mas simple tiene la siguiente apariencia:
$$\dfrac{dy}{dx}=f(x)$$
Para resolver esta ED debemos realizar el proceso inverso de la derivada, es decir integrar:
$$\int\dfrac{dy}{dx}dx=\int f(x) dx\Rightarrow y=F(x)+C$$
Debe notarse que debido a $C$ esta solución es en realidad un número infinito de soluciones. A esto se le llama \textbf{Familia de Curvas}. Una solución particular de esta ecuación, es una sola curva contenida en la familia de curvas.
\subsection*{Orden, Grado y Linealidad de una Ecuación Diferencial}
\begin{itemize}
\item \textbf{Orden:} El Orden de una ED\footnote{ED = Ecuaciones Diferenciales (notación que usaremos de aquí en adelante)} es el orden de la derivada mas alta de la función desconocida (variable dependiente) que aparece en la ecuación.
\item \textbf{Grado:} El grado se expresa mediante el mayor exponente de la derivada de mayor orden.
\item \textbf{Linealidad:} Una ED Lineal de orden $n$ es una ecuación de la forma:
$$a_n(x)\dfrac{d^ny}{dx^n}+a_{n-1}(x)\dfrac{d^{n-1}y}{dx^{n-1}}+\cdots + a_2(x)\dfrac{d^2y}{dx^2}+a_1(x)\dfrac{dy}{dx}+a_0(x)y=h(x)$$
Cuyos coeficientes $a_0(x),a_1(x),\ldots,a_n(x)$ y el segundo miembro $h(x)$ son continuos en un Intervalo $I$ en el que la $a_n(x)\neq 0$. Si $h(x)=0$ la Ecuación se llama \textit{Ecuación Diferencial de Orden $n$ Homogénea.}
\end{itemize}
\section{Formas de Resolución}
\subsection{Ecuaciones con Variables Separables}
Una ED de 1er Orden $\frac{dy}{dx}=f(x,y)$ es separable, si la función $f(x,y)$ se puede escribir como un producto de una función de $x$ y una función de $y$.
$$\dfrac{dy}{dx}=f(x,y)\Rightarrow \dfrac{dy}{dx}=f(x)g(y)$$
\section{Ecuaciones Homogéneas}
Una función $M(x,y)$ es Homogénea de grado $n$ si la suma de las potencias de $x$ y $y$ en cada termino es $n$. Esto es:
$$M(\lambda x,\lambda y)=\lambda^nM(x,y)$$
Una ED Homogénea es una ED de la forma:
$$\dfrac{dy}{dx}=\dfrac{M(x,y)}{N(x,y)}$$
Donde $M$ y $N$ son funciones homogéneas del mismo grado.
\subsection{Método de Solución}
Las ED Homogéneas se resuelven haciendo la sustitución:
$$y=Vx \hspace{1.5cm} V=\dfrac{y}{x}$$
donde $V$ es una función de $x$ por lo tanto:
$$\dfrac{dy}{dx}=V+x\dfrac{dV}{dx}$$
El miembro que corresponde a la función se puede expresar en términos de $\frac{y}{x}$. Esto se hace dividiendo todo entre $x^n$. Donde $n$ es el grado de: $N$ y $M$.

