%Ejemplos del Capitulo 4
\section{Ejemplos}
\begin{enumerate}
\item Consideremos:
\begin{itemize}
\item $x:$ Precios (en miles de Bolivianos)
\item $y:$ Cantidad Demandada.
\end{itemize}
Y contamos con la siguiente tabla:
\begin{center}
\begin{tabular}{|c|c|c|c|c|c|c|c|}
\hline 
$x$ & 5 & 7 & 9 & 12 & 17 & 23 & 30 \\ 
\hline 
$y$ & 100 & 90 & 86 & 72 & 60 & 55 & 43 \\ 
\hline 
\end{tabular} 
 \end{center} 
 \begin{enumerate}
 \item Construir el diagrama de Dispersión.
 \item Hallar la Ecuación de Regresión
 \item ¿Cual es la cantidad demandada cuando el precio es 18?
 \item ¿Cual es la cantidad demandada para un precio de 75?
 \end{enumerate}
\subsubsection{Solución}
\item En el curso de Estadística de la \texttt{UAGRM} se seleccionó una muestra representativa de doce alumnos varones y se registró sus estaturas en metros y sus pesos en kilogramos como se indica en el siguiente cuadro:
\begin{center}
\begin{tabular}{|c|c|c|c|c|c|c|c|c|c|c|c|c|}
\hline 
Estaturas & $1.70$ & $1.68$ & $1.86$ & $1.60$ & $1.68$ & $1.55$ & $1.62$ & $1.68$ & $1.70$ & $1.65$ & $1.82$ & $1.56$ \\ 
\hline 
Pesos & $72$ & $65$ & $82$ & $58$ & $63$ & $65$ & $58$ & $70$ & $69$ & $62$ & $76$ & $60$ \\ 
\hline 
\end{tabular} 
\end{center}
\begin{enumerate}
\item Construir el diagrama de esparcimiento.
\item Determinar la Ecuación de Regresión Lineal.
\item Graficar la regresión lineal obtenida.
\item Estimar el peso de un alumno con estatura $1.75m$
\end{enumerate}
\subsubsection{Solución}
\item Sean:
\begin{itemize}
\item $x:$ Gastos en publicidad de un producto.
\item $y:$ Ventas Conseguidas.
\end{itemize}
Y la siguiente tabla:
\begin{center}
\begin{tabular}{|c|c|c|c|c|c|c|}
\hline 
$x$ & 1 & 2 & 3 & 4 & 5 & 6 \\ 
\hline 
$y$ & 10 & 17 & 30 & 28 & 39 & 47 \\ 
\hline 
\end{tabular} 
\end{center}
\begin{enumerate}
\item Construir el diagrama de esparcimiento.
\item Determinar la Ecuación de regresión, $y$ sobre $x$.
\item Calcular el error Estándar de estimación.
\item Calcular el coeficiente de correlación rectilíneo.
\item Graficar la línea de regresión obtenida.
\end{enumerate}
\subsubsection{Solución}
\end{enumerate}