\section{Ejemplos}
\begin{enumerate}
\item Tres señoritas: María, Marta y Maritza. Compiten en un concurso de belleza. Los premios son solamente otorgados a las que ocupan el primer y segundo lugar.
\begin{enumerate}
\item Listar los elementos del espacio muestral, correspondiente a los experimentos \textit{''Elegir dos Ganadores''}
\item Definir como subconjunto de $S$, los eventos:
\begin{itemize}
\item $A$: María gana el concurso de Belleza.
\item $B$: María gana el segundo lugar.
\item $C$: Maritza y Marta ganan los premios.
\end{itemize}
\end{enumerate}
\subsubsection{Solución}
\begin{enumerate}
\item Correspondiente al experimento, \textit{''Elegir dos Ganadores''}:
\begin{equation*}
S=
\left\lbrace 
\begin{aligned}
(Maria,Marta),(Maria,Maritza), \\
(Marta,Maria),(Marta,Maritza), \\
(Maritza,Maria),(Maritza,Marta)
\end{aligned}
\right\rbrace 
\end{equation*}
\item
\begin{itemize}
\item $A=\left\lbrace (Maria,Marta),(Maria,Maritza),(Marta,Maria),(Maritza,Maria)\right\rbrace $
\item $B=\left\lbrace (Marta,Maria),(Maritza,Maria)\right\rbrace$
\item $C=\left\lbrace (Maritza,Marta),(Marta,Maritza)\right\rbrace$
\end{itemize}
\end{enumerate}
\item Un experimento consiste en lanzar dos dados y observar los números que aparecen en la cara superior:
\begin{enumerate}
\item Listar los elementos del espacio muestral.
\item Listar los elementos tal que:
\begin{itemize}
\item $A:$ La suma de los números es 5.
\item $B:$ La suma de los números es 12.
\item $C:$ El producto de los números es 24.
\item $D:$ La suma de los números es divisibles por 7.
\end{itemize}
\end{enumerate}
\subsubsection{Solución}
\item Un inversionista planea escoger dos de las cinco oportunidades de inversión que le han recomendado. Describa el espacio muestral que represeta las opciones posibles.
\subsubsection{Solución}
Diremos que las cinco oportunidades serán: $A,B,C,D$ y $E$. Como tiene que escoger dos oportunidades, el orden de las parejas no importa, por lo tanto, el espacio muestral será:
\begin{equation*}
S=
\left\lbrace 
\begin{aligned}
(A,B),(A,C),(A,D),(A,E), \\
(B,C),(B,D),(B,E), \\
(C,D),(C,E), \\
(D,E)
\end{aligned}
\right\rbrace 
\end{equation*}
\item Tres artículos son extraídos con reposición de un lote de mercancías, cada artículo ha de ser identificado como defectuoso (D) y no defectuoso (N). Describir todos los puntos posibles del espacio muestral para este experimento.
\subsubsection{Solución}
\begin{equation*}
S=
\left\lbrace 
\begin{aligned}
(D,D,D),(N,D,D),\\
(D,D,N),(N,D,N),\\
(D,N,D),(N,N,D),\\
(D,N,N),(N,N,N)
\end{aligned}
\right\rbrace 
\end{equation*}
\item Una moneda se lanza tres veces. Describir los siguientes eventos:
\begin{itemize}
\item $A:$ \textit{Ocurre por lo menos dos caras.}
\item $B:$ \textit{Ocurre, sello en el tercer lanzamiento.}
\item $C:$ \textit{Ocurre a lo mas, una cara.}
\end{itemize}
Luego hallar:
\begin{enumerate}
\item $A\cup B$
\item $B-C$
\item $A^c \cap B^c$
\item $A$ {\footnotesize $\triangle$} $C$
\end{enumerate}
\subsubsection{Solución}
Describiendo los eventos $A,B$ y $C$:
\begin{multicols}{3}
\noindent
\begin{equation*}
A=
\left\lbrace 
\begin{aligned}
(C,C,C),(C,C,S), \\
(C,S,C),(S,C,C)
\end{aligned}
\right\rbrace 
\end{equation*}
\columnbreak
\begin{equation*}
B=
\left\lbrace 
\begin{aligned}
(C,C,S),(C,S,S),\\
(S,C,S),(S,S,S)
\end{aligned}
\right\rbrace 
\end{equation*}
\columnbreak
\begin{equation*}
C=
\left\lbrace 
\begin{aligned}
(C,S,S),(S,C,S),\\
(S,S,C),(S,S,S)
\end{aligned}
\right\rbrace 
\end{equation*}
\end{multicols}
Haciendo las operaciones de los incisos:
\begin{enumerate}
\item \begin{equation*}
A\cup B=
\left\lbrace 
\begin{aligned}
(C,C,C),(C,C,S),(C,S,C),(S,C,C) \\ 
(C,S,S),(S,S,S),(S,C,S)
\end{aligned}
\right\rbrace 
\end{equation*}
\item \begin{equation*}
B-C=
\left\lbrace 
\begin{aligned}
(C,C,S)
\end{aligned}
\right\rbrace 
\end{equation*}
\item 
\begin{equation*}
(A^c \cap B^c)=(A\cup B)^c = \{ (S,S,C) \}
\end{equation*}
\item Podemos reescribir de la siguiente manera, y aplicamos la unión:
$$
\text{$A$ {\footnotesize $\triangle$} $C$} = (A-C)\cup (C-A)
$$
Hallamos $(A-C)$ y $(C-A)$ respectivamente:
\begin{align*}
A-C &= \{ (C,C,C),(C,C,S),(C,S,C),(S,C,C)\} \\
C-A &= \{ (C,S,S),(S,C,S),(S,S,C),(S,S,S)\} 
\end{align*}
Como podemos ver, al realizar la unión, obtenemos todos los posibles resultados del espacio muestral. Por lo tanto: $A$ {\footnotesize $\triangle$} $C=S$
\end{enumerate}
\item Sea el experimento, \textit{Lanzar una moneda hasta que ocurra cara y contar el número de lanzamientos de la moneda.}  Considerando:
\begin{itemize}
\item $A:$ \textit{Se necesita un número par de lanzamientos}
\item $B:$ \textit{Se necesitan mas de diez lanzamientos.}
\end{itemize}
Hallar:
\begin{enumerate}
\item $A\cap B$
\item $A-B$
\item $B-A$
\item $A^c$
\item $B^c$
\item $A^c \cap B^c$
\end{enumerate}
\subsubsection{Solución}
\item Demostrar que: $P(A/B)+P(A^c/B)=1$
\subsubsection{Solución}
\begin{align*}
P(A/B)+P(A^c/B) &\varstackrel{1}{=} \dfrac{P(A\cap B)}{P(B)} + \dfrac{P(A^c\cap B)}{P(B)} \\
  &\varstackrel{2}{=} \dfrac{P(A\cap B)+P(A^c\cap B)}{P(B)} \\
    &\varstackrel{3}{=} \dfrac{P[(A\cap B)\cup(A^c\cap B)]}{P(B)} \\
      &\varstackrel{4}{=} \dfrac{P[(A\cup A^c)\cap B]}{P(B)} \\
        &\varstackrel{5}{=} \dfrac{P(\Omega\cap B)}{P(B)} \\
           &\varstackrel{6}{=} \dfrac{P(B)}{P(B)} \\
                 &\varstackrel{7}{=} 1 \\
\end{align*}
$\therefore P(A/B)+P(A^c/B)=1$
\item Demostrar que: $P(A/B^c)+P(A^c/B)=1$
\subsubsection{Solución}
\begin{align*}
P(A/B^c)+P(A^c/B) &\varstackrel{1}{=} \dfrac{P(A\cap B^c)}{P(B^c)} + \dfrac{P(A^c\cap B)}{P(B)} \\
&\varstackrel{2}{=} \dfrac{P(A)-P(A\cap B)}{P(B^c)}+ \dfrac{P(B)-P(A\cap B)}{P(B)}
\end{align*}
\item La probabilidad de que llueva en Santa Cruz el 10 de Junio es de 40\%, de que truene es 5\% y de que, llueve y truene es de 3\%. ¿Cual es la probabilidad de que llueva o truene ese día?
\subsubsection{Solución}
Conociendo los siguientes eventos:
\begin{itemize}
\item $A:$ Llueva
\item $B:$ Truene
\item $A\cap B:$ Llueva y truene
\end{itemize}
Sabemos los valores de las probabilidades de cada uno, de acuerdo al enunciado:
\begin{itemize}
\item $P(A)=40\%$
\item $P(B)=5\%$
\item $P(A\cap B)=3\%$
\end{itemize}
Con estos datos nos queda hallar $A\cup B$, por lo tanto calculamos simplemente reemplazando en la igualdad:
\begin{align*}
 P(A\cup B) &= P(A)+P(B)-P(A\cap B)\\
 	      &= 40\% + 5\% - 3\%    \\
 	      	&= 42\%
\end{align*}
\item Sea $A$ y $B$ dos eventos, tal que: $P(A)=0.20\% ; P(B)=0.30\%;P(A\cup B)=0.10\%$ \\ Hallar:
\begin{enumerate}
\item $P(A^c\cap B^c)$
\item $P(A^c\cap B)$
\item $P(A\cap B^c)$
\item $P(A^c\cup B)$
\end{enumerate}


\subsubsection{Solución}
\item Con 7 ingenieros y 4 médicos se formaron comités de 6 miembros. ¿Cuál es la probabilidad que el comité incluya: ?
\begin{enumerate}
\item Exactamente 2 médicos.
\item Al menos 2 ingenieros.
\end{enumerate}
\subsubsection{Solución}
\item En la UAGRM el 30\% de los estudiantes son cruceños, el 10\% estudia Ingeniería Informática, y el 1\% son cruceños y estudian Informática. Si se selecciona al azar un estudiante de la UAGRM. Hallar las siguientes probabilidades:
\begin{enumerate}
\item El estudiante no es cruceño.
\item Sea cruceño o pertenezca a Ingeniería Informática.
\item Sea cruceño y no estudie Ingeniería Informática.
\item No sea cruceño ni estudie Ingeniería Informática.
\end{enumerate}
\subsubsection{Solución}
\item En una encuesta pública se determina que la probabilidad que una persona consuma $A$ es $0.50$ que consuma $B$ es $0.37$, que consuma $C$ es $0.30$, que consuma $A$ y $B$ es $0.12$ que consuma solamente $A$ y $C$ es $0.08$ que consuma solamente $B$ y $C$ es $0.05$ y que consuma solamente $C$ es $0.15$.
\begin{itemize}
\item Calcular la probabilidad que de que alguien consuma $A$ o $B$ pero no $C$.
\end{itemize}
\subsubsection{Solución}
\item Se lanza un dado legal sobre una mesa y se observa el número que aparece en la cara superior.
\begin{enumerate}
\item ¿Cual es la probabilidad de obtener un número par?
\item ¿Cual es la probabilidad de obtener un número impar?
\item Los eventos \textit{''Obtener número par''} y \textit{''Obtener número impar''} son mutuamente excluyentes?¿Son Independientes?
 
\end{enumerate}
\subsubsection{Solución}
\begin{enumerate}
\item Primeramente definamos el espacio muestral, que sería $\Omega = \lbrace 1,2,3,4,5,6 \rbrace$, de este e.m. tomamos en cuenta todos los casos que sean favorables al enunciado, \textit{''Obtener Número Par''}, estos son un evento $A=\lbrace 2,4,6\rbrace$. Finalmente:
$$P(A)=\dfrac{n(A)}{n(\Omega)}=\dfrac{3}{6}=\dfrac{1}{2} (50\%)$$
\item Bajo la misma lógica anterior realizamos para los números impares, teniendo: $B=\lbrace 1,3,5\rbrace$
$$P(B)=\dfrac{n(B)}{n(\Omega)}=\dfrac{3}{6}=\dfrac{1}{2} (50\%)$$
\end{enumerate}
\item Una urna contiene 6 fichas blancas y 4 negras. Se extraen 2 fichas sucesivamente y sin restitución.
\begin{enumerate}
\item ¿Cual es la probabilidad de que ambas sean negras?
\item ¿Cual es la probabilidad de que la primera sea blanca y la segunda negra?
\item ¿Cual es la probabilidad de que la primera sea negra y la segunda blanca?
\item ¿Cual es la probabilidad de que ambas sean blancas?
\end{enumerate}
\subsubsection{Solución}
\item En un grupo de personas hay 3 mujeres y 4 hombres varones. Si se elige una persona al azar ¿Cual es la probabilidad que sea varón?
\subsubsection{Solución}
\item Si $A$ y  $B$ son eventos cualesquiera, demostrar:
$$P(A\cup B\cup C)=P(A)+P(B)+P(C)-P(A\cap C)-P(B\cap C)-P(A\cap B)+P(A\cap B \cap C)$$
\subsubsection{Solución}
\item La probabilidad de que un estudiante apruebe matemáticas es $\frac{2}{3}$ y que apruebe física es $\frac{4}{9}$. Si la probabilidad de aprobar al menos una de estas materias es $\frac{4}{5}$. ¿ Cual es la probabilidad de aprobar ambas materias?
\subsubsection{Solución}
El problema nos plantea los siguientes datos:
\begin{itemize}
\item $A:$ \textit{Aprobar Matemáticas.}
\item $B:$ \textit{Aprobar Física.}
\item $A\cup B :$ \textit{Aprobar al menos una de estas materias.} (Es decir, aprobar Física o Matemáticas)
\end{itemize}
Por lo tanto, las probabilidades ya están dadas:
\begin{itemize}
\item $P(A)= \frac{2}{3}$
\item $P(B)=  \frac{4}{9}$
\item $P(A\cup B)= \frac{4}{5}$
\end{itemize}
Como el problema pide la probabilidad de aprobar ambas materias, esto es: $P(A\cap B)$. Si partimos de la siguiente igualdad:
\begin{align*}
P(A\cup B) = P(A) + P(B) - P(A\cap B)
\end{align*}
Luego, despejamos $P(A\cap B)$ y reemplazamos:
\begin{align*}
P(A\cap B) &= P(A) + P(B) - P(A\cup B) \\
           &= \dfrac{2}{3}+\dfrac{4}{9}-\dfrac{4}{5} \\
           &= \dfrac{14}{45} \approx 31.11 \%
\end{align*}
Obtenemos que la probabilidad de aprobar ambas materias es de $31.11\%$.
\item En los últimos años la Universidad \textbf{ABC} ha estado llevando un registro de sus egresados, en la actualidad tienen empleo, anotando el número de años que utilizaron para terminar su carrera y su posición (alta, media o baja) que tienen como profesionales.
\begin{center}
\begin{tabular}{|c|c|c|c|c|}
  \hline
  \diagbox[innerwidth=2cm,height=1.2cm]{Posición}{Tiempo} & Alta & Media & Baja  & Total \\ \hline
  5 Años & $30$ & $70$ & $20$ & $120$ \\ \hline
  +5 Años & $20$ & $30$ & $30$ & $80$ \\ \hline
  Total & $50$ & $100$ & $50$ & $200$ \\
  \hline
\end{tabular}
\end{center}
\begin{enumerate}
\item Basados en la información anterior.  ¿ Cual es la probabilidad de que si la duración de sus estudios fue de 5 años, tenga una alta posición profesional en su empleo actual?
\item Si el empleado tiene baja posición ¿Cual es la probabilidad de que tal persona haya realizado sus estudios en +5 años?
\item Si el egresado tiene posición profesional media, ¿ Cual es la probabilidad de que tal persona haya realizado la cerrera en 5 años?
\end{enumerate}
\item La urna $A$ contiene 6 fichas grises y 4 rojas, y la urna $B$ contiene 2 fichas grises y 7 rojas. Se saca una ficha de $A$ y se coloca en la $B$, en seguida se saca una ficha de $B$. Dado que la ficha extraída de $B$ es gris. ¿Cual es la probabilidad de que la ficha sacada de $A$ sea gris?
\subsubsection{Solución}
\item La urna $A_1$ contiene 8 fichas blancas y 2 negras. La urna $A_2$ contiene 3 fichas blancas y 7 negras, finalmente, la urna $A_3$ contiene 5 fichas blancas y 5 negras. Se lanza un dado no cargado, si resulta $\left\lbrace 1,2,3\right\rbrace $ se saca una ficha de la urna $A_1$; si resulta $\left\lbrace 4,5 \right\rbrace $ se saca una ficha de $A_2$, si resulta $\left\lbrace 6 \right\rbrace $ se saca de la urna $A_3$. \\${ }$\\ Dado que la ficha extraída es blanca. ¿ Cual es la probabilidad de que venga de $A_2$?
\subsubsection{Solución}
\end{enumerate}
