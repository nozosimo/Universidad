\section{Definición Axiomática de Probabilidad}
Sea $S$ un espacio muestral asociado a un cierto experimento aleatorio y sea $\Gamma$ $\sigma$-algebra de eventos subconjunto de $S$.

\begin{align*}
  P \colon \Gamma &\to [0,1]\\
  A &\mapsto P(A)
\end{align*}
Tal que satisface los siguientes axiomas:
\begin{enumerate}[label=(\roman*)]
\item $0\leq P(A)\leq 1$
\item $P(S)=1$
\item $P\left( \bigcup\limits_{i=1}^{n} A_i\right)=
\displaystyle\sum_{i=1}^{n}P(A_i) \hspace{1cm}A_i \textrm{son eventos mutuamente excluyentes}$
\end{enumerate}
Una consecuencia inmediata del axioma (III) es:
$$P(A\cup B)=P(A)+P(B)$$
Esto es si $A$ y $BB$ son eventos mutuamente excluyentes.
\subsection{Propiedades de la Probabilidad de Eventos}
\begin{enumerate}
\item Si $\phi$ es un evento imposible entonces: $P(\phi)=0$
\item Si $A^c$ es el evento complementario de $A$, entonces:
$$P(A^c)=1-P(A) \hspace{0.5cm} \textrm{ó} \hspace{0.5cm}  P(A)=1-P(A^c)$$
\item Si $A$ y $B$ son eventos tales que: $A\subset B$ entonces:
$$P(A)\leq P(B)$$
\item Si $A$ y $B$ son eventos cualesquiera, entonces:
$$P(A\cup B)=P(A)+P(B)-P(A\cap B)$$
\item Si $A,B$ y $C$ son eventos mutuamente excluyentes, entonces:
$$P(A\cup B \cup C) = P(A)+P(B)+P(C)$$
\item Si $A,B$ y $C$ son eventos cualesquiera, entonces:
$$P(A\cup B \cup C) = P(A)+P(B)+P(C) - P(A\cap B)-P(A\cap C)-P(B\cap C)+P(A\cap B\cap C)$$
\item Si $A_1,A_2,A_3,\ldots,A_n$ son eventos cualesquiera:
$$ P\left( \bigcup\limits_{i=1}^{n} A_i\right)   \leq \sum_{i=1}^{n}P(A_i)$$
\end{enumerate}