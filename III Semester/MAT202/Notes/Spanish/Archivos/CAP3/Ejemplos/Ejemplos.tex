%Ejemplos Capitulo 3
\section{Ejemplos}
\begin{enumerate}
\item En un estudio, las calificaciones obtenidas en un curso interno sobre principios de administración y los años de servicio de los empleados inscritos en el curso, dio como resultado estas estadísticas:


\item Dados los siguientes datos: $2,3,4,3,5,3$
\begin{itemize}
\item Hallar el primer Coeficiente de Asimetría de Pearson $(S_p)$.
\end{itemize}
\subsubsection{Solución}
Para hallar el primer Coeficiente de Asimetría de Pearson necesitamos calcular: $\overline{X}, M_o$ y $S$. De acuerdo a la formula:$$S_p = \dfrac{\overline{X}-M_o}{S}$$
Primeramente podemos obtener la moda, esto es: $M_o=3$. Seguido, obtenemos la media:
$$\overline{X}=\dfrac{2+3+4+3+5+3}{6}=\dfrac{20}{6}=3.\overline{3}$$
Por último hallamos $S$ (es decir, $\sqrt{S^2}$):
$$S^2=\dfrac{(2-3.3)^2+(3-3.3)^2+(4-3.3)^2+(5-3.3)^2}{5}=1.04 \Rightarrow S=\sqrt{1.04}=1.02$$
 
\item Sea la distribución de frecuencias:
\item Sea la distribución de frecuencias:
\begin{center}
\begin{tabular}{|c|c|c|}
\hline 
$y_{i-1}'-y_i'$ & $f_i$ & $y_i$ \\ 
\hline 
$6-10$ & 12 & 8 \\ 
\hline 
$10-14$ & 16 & 12 \\ 
\hline 
$14-18$ & 20 & 16 \\ 
\hline 
$18-22$ & 16 & 20 \\ 
\hline 
$22-26$ & 12 & 24 \\ 
\hline 
\end{tabular} 
\end{center}
\begin{itemize}
\item Hallar $a_4$ (curtosis) e interpretar.
\end{itemize}
\subsubsection{Solución}
\item La siguiente distribución se refiere a los salarios de cien empleados:
\begin{center}
 \begin{tabular}{|c|c|}
\hline 
Salarios & Trabajadores \\ 
\hline 
150 & 10 \\ 
\hline 
250 & 20 \\ 
\hline 
400 & 40 \\ 
\hline 
600 & 20 \\ 
\hline 
1550 & 10 \\ 
\hline 
\end{tabular}
 \end{center} 
\begin{itemize}
\item Hallar el índice de Gini, interpretar y realizar la Curva de Lorenz.
\end{itemize}
\subsubsection{Solución}
\item Las notas de sesenta alumnos en exámenes y trabajos prácticos son las siguientes:
\begin{enumerate}
\item Construir la distribución de frecuencias absoluta, conjuntas, relativas y acumuladas.
\item Calcular las medias de las distribuciones marginales.
\item Hallar la covarianza e interpretar.
\end{enumerate}
\subsubsection{Solución}
\item El siguiente cuadro es la distribución de 100 familias por el número de hijos $(x_i)$ y el número de dormitorios por vivienda $(y_i)$.
\begin{center}
\begin{tabular}{|c|c|c|c|}
  \hline
  \diagbox[innerwidth=1cm]{$x_i$}{$y_i$} & 1 & 2 & 3    \\ \hline
  0 & 9 & 4 & 0\\ \hline
  1 & 10 & 16 & 9 \\ \hline
  2 & 5 & 16 & 12\\ \hline
  3 & 0 & 6 & 13\\ 
  \hline
\end{tabular}
\end{center}
\begin{enumerate}
\item Construir la Distribución Marginal $x,y$
\item Construir las correspondientes distribuciones de frecuencia absoluta y relativa acumulada.
\item  ¿Cuantas familias tienen a lo sumo dos hijos y a lo sumo dos dormitorios?
\end{enumerate}
\subsubsection{Solución}
\end{enumerate}