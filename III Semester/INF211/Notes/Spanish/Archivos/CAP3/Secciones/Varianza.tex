\section{Varianza $(S^2,\hat{S})$}
\subsection{Cálculo}
\subsubsection{Datos Originales}
\begin{itemize}
\item \textbf{Muestras Grandes:}
$$\hat{S}=\dfrac{\displaystyle\sum_{i=1}^{n}(x_i-\overline{x})^2}{n} \hspace{1cm} n\geq 30$$
\item \textbf{Muestras Pequeñas:}
$$\hat{S}=\dfrac{\displaystyle\sum_{i=1}^{n}(x_i-\overline{x})^2}{n-1} \hspace{1cm} n< 30$$
\end{itemize}
\subsection{Propiedades}
\begin{enumerate}
\item $\hat{S}\geq 0$
\item $V(a)=0$ , $a$ constante.
\item $V(x \pm a) = V(x)$
\item $V(ax)=a^2V(x)$
\item $V(ax+b)=a^2V(x)$, $b$ constante.
\end{enumerate}
\subsection*{Observaciones}
\begin{itemize}
\item La diferencia entre $S^2$ y $\hat{S}$ es grande para muestras pequeñas y es mínima para muestras grandes, casi son iguales.
\item La Varianza siempre viene expresada en unidades al cuadrado.
\end{itemize}
\subsection{Componentes de la Varianza}
Si un conjunto de datos se divide en subconjuntos por categoría o estratos, es posible descomponer la Varianza en dos componentes.
\subsubsection{Intervarianza}
Es un estadígrafo que representa la variabilidad entre los estratos y se define como la varianza de la media de los estratos:
$$S_b^2=V(\overline{y}_h)=\dfrac{\displaystyle\sum_{h=1}^{L}(\overline{y}_h-\overline{y})^2 \cdot f_h}{n}$$
Donde:
\begin{itemize}
\item $\overline{y}_h$: Media de los estratos
\item $\overline{y}$: Media General 
\item $f_h$: Frecuencia
\item $L$: Número de Estratos
\end{itemize}
\subsubsection{Intravarianza}
Presenta la variabilidad dentro de los estratos definida como la media de la varianza de los estratos:
$$S_w^2=M(S_h^2)=\dfrac{\displaystyle\sum_{h=1}^{L}S_h^2 \cdot f_h}{n}$$
\begin{itemize}
\item $S_h^2$: Varianza de los estratos.
\end{itemize}
Como ambos componen la Varianza, se cumple la siguiente igualdad:
$$S^2=S_b^2+S_w^2$$
\subsection{Métodos Abreviados para el Cálculo de la Varianza}
\subsubsection{Primer Método Abreviado}
\noindent Se sabe que:
$$\hat{S}=\dfrac{\displaystyle\sum_{i=1}^{n}(x_i-\overline{x})^2}{n}$$
Si desarrollamos el Binomio y distribuimos la sumatoria:
$$\hat{S}=\dfrac{\displaystyle\sum_{i=1}^{n}(x_i^2-2x_i\overline{x}+\overline{x}^2)}{n}=
\dfrac{\displaystyle\sum_{i=1}^{n}x_i^2 - 2\overline{x}\displaystyle\sum_{i=1}^{n}x_i+n\overline{x}^2}{n}
$$
Separamos el denominador:
$$\hat{S}=\dfrac{\displaystyle\sum_{i=1}^{n}x_i^2}{n}-2\overline{x}\cdot\dfrac{\displaystyle\sum_{i=1}^{n}x_i}{n}+\dfrac{n\overline{x}^2}{n} \hspace{1cm} \textrm{Recordemos que: } \dfrac{\displaystyle\sum_{i=1}^{n}x_i}{n}=\overline{x}$$
Reemplazamos y simplificamos:
$$\hat{S}=\dfrac{\displaystyle\sum_{i=1}^{n}x_i^2}{n}-2\overline{x}\cdot\overline{x}+\overline{x}^2=\dfrac{\displaystyle\sum_{i=1}^{n}x_i^2}{n}-2\overline{x}^2+\overline{x}^2$$
Finalmente tendremos:
$$\hat{S}=\dfrac{\displaystyle\sum_{i=1}^{n}x_i^2}{n}-\overline{x}^2$$
\subsubsection{Segundo Método Abreviado}
$$\hat{S}=\dfrac{\displaystyle\sum_{i=1}^{k}z_i'^2 f_i }{n}-\dfrac{\left(\displaystyle\sum_{i=1}^{k}z_i' f_i \right)^2}{n^2}$$
\subsubsection{Tercer Método Abreviado}
$$\hat{S}=C^2 
\begin{bmatrix}\dfrac{\displaystyle\sum_{i=1}^{k}z_i''^2 f_i }{n}-\dfrac{\left(\displaystyle\sum_{i=1}^{k}z_i'' f_i \right)^2}{n^2} \end{bmatrix}$$
\subsection{Ejemplos}